\section{Inateness und Gramatiktheorie}


\frame[shrink=10]{
\frametitle{Inateness und Gramatiktheorie}


\begin{itemize}[<+->]
\item Chomskys Hypothese: Sprachliche Fähigkeiten sind angeboren.
\item Spracherwerb: Prinzipien und Parameter\\
Menschen verfügen über ein vorangelegten Satz grammatischer Kategorien und syntaktischer Strukturen.
\item In Abhängigkeit vom sprachlichen Input, den Kinder bekommen, setzen sie bestimmte Parameter
  und je nach Art der Parametersetzung ergibt sich dann die Grammatik des Deutschen, Englischen oder
  Japanischen.
\item Beispiel: Kopfposition initial oder final.

\item Inateness wird in der Grammatiktheorie missbraucht:\\
Man kann von gewissen Dingen einfach annehmen, dass sie Bestandteil der angeborenen Information sind und so ein einfacheres
  Gesamtsystem enthalten. 
\item Beispiel: Objektkongruenz im Baskischen.\\
 Annahme: Die allgemine Fähigkeit, sowas zu erfassen, ist angeboren. $\to$\\
 Es muss entsprechende Strukturen auch im Deutschen geben.
\item Zur Erklärung der Fakten braucht man die Annahme sprachspezifischen angeborenen Wissens aber nicht.
\end{itemize}

}

\frame{
\frametitlefit{Objektkongruenz im Baskischen motiviert AgrO im Deutschen}

\hfill\scalebox{0.4}{%
\begin{pspicture}(1.6,0)(17.4,16.3)
%\psgrid
\rput[B](3,0){\rnode{die Firma}{die Firma Müller}}
\rput[B](6,0){\rnode{meinem Onkel}{meinem Onkel}}
\rput[B](9,0){\rnode{diese Moebel}{diese Möbel}}
\rput[B](12,0){\rnode{erst gestern}{erst gestern}}
\rput[B](15,0){\rnode{zugestellt}{zugestellt}}
\rput[B](17,0){\rnode{hat}{hat}}
%
\rput[B](14,2){\rnode{tdo}{t\sub{DO}}}
\rput[B](15,2){\rnode{v}{V}}
\rput[B](14.5,3){\rnode{vp1}{VP}}
%
\rput[B](13.5,3){\rnode{tio}{t\sub{IO}}}
\rput[B](14,4){\rnode{vs}{V'}}
%
\rput[B](13,4){\rnode{tsu}{t\sub{SU}}}
\rput[B](13.5,5){\rnode{vp2}{VP}}
%
\rput[B](12,5){\rnode{adv}{Adv}}
\rput[B](13,6){\rnode{vp3}{VP}}
%
\rput[B](15,6){\rnode{agro0}{AgrO$^0$}}
\rput[B](14,7){\rnode{agros}{AgrO'}}
%
\rput[B](9,7){\rnode{do}{DO}}
\rput[B](11.5,9){\rnode{agrop}{AgrOP}}
%
\rput[B](13.5,9){\rnode{agrio0}{AgrIO$^0$}}
\rput[B](12.5,10){\rnode{agrios}{AgrIO'}}
%
\rput[B](6,10){\rnode{io}{IO}}
\rput[B](9.25,12){\rnode{agriop}{AgrIOP}}
%
\rput[B](17,12){\rnode{agrs0}{AgrS$^0$}}
\rput[B](13.125,14){\rnode{agrss}{AgrS'}}
%
\rput[B](3,14){\rnode{su}{SU}}
\rput[B](8.06125,16){\rnode{agrsp}{AgrSP}}
%
\psset{angleA=-90,angleB=90,arm=0pt}
%
\ncdiag{vp1}{v}
\ncdiag{vp1}{tdo}
%
\ncdiag{vs}{tio}
\ncdiag{vs}{vp1}
%
\ncdiag{vp2}{vs}
\ncdiag{vp2}{tsu}
%
\ncdiag{vp3}{vp2}
\ncdiag{vp3}{adv}
%
\ncdiag{agros}{vp3}
\ncdiag{agros}{agro0}
%
\ncdiag{agrop}{agros}
\ncdiag{agrop}{do}
%
\ncdiag{agrios}{agrop}
\ncdiag{agrios}{agrio0}
%
\ncdiag{agriop}{agrios}
\ncdiag{agriop}{io}
%
\ncdiag{agrss}{agriop}
\ncdiag{agrss}{agrs0}
%
\ncdiag{agrsp}{agrss}
\ncdiag{agrsp}{su}
%
\pstriangle(3,0.4)(2.6,13.5)
\pstriangle(6,0.4)(2.3,9.5)
\pstriangle(9,0.4)(1.9,6.5)
\pstriangle(12,0.4)(1.7,4.5)
\ncdiag{v}{zugestellt}
\ncdiag{agrs0}{hat}
\end{pspicture}}\hfill\hfill\mbox{}





}

\frame{
\frametitle{Argumente für Inateness}

\begin{itemize}[<+->]
\item Syntaktische Universalien
\item die Tatsache, dass es eine "`kritische"' Periode für den Spracherwerb gibt
\item Fast alle Kinder lernen Sprache, aber Primaten nicht.
\item Kinder regularisieren spontan Pidgin-Sprachen.
\item Lokalisierung in speziellen Gehirnbereichen
\item Angebliche Verschiedenheit von Sprache und allgemeiner Kognition
\item Williams-Syndrom 
\item KE-Familie mit FoxP2-Mutation 
\item Poverty of the Stimulus
\end{itemize}

Siehe hierzu \citew{Pinker94a} und die Kritik von \citew{Tomasello95a}.

}


\subsection{Syntaktische Universalien}


\frame{
\frametitle{Syntaktische Universalien}

Behauptung: Folgende Dinge sind universal und sprachspezifisch:
\begin{itemize}[<+->]
\item \xbar-Strukturen
\item Grammatische Funktionen wie Subjekt und Objekt
\item Eigenschaften von Fernabhängigkeiten
\item Grammatische Morpheme für Tempus, Modus und Aspekt
\item Wortarten (Nomen und Verb)
\item Rekursion \citep*{HCF2002a}
\end{itemize}

}

\subsubsection{\xbar-Strukturen}

\frame{
\frametitle{\xbar-Strukturen}

\begin{itemize}[<+->]
\item Es gibt Sprachen wie Dyirbal (Australien), für die es \zb nicht sinnvoll erscheint, eine VP anzunehmen.

\item Formal ist die Annahme von \xbar-Strukturen keine Einschränkung der möglichen Grammatiken,
wenn man leere Köpfe zuläßt. \citep{KP90a}

\pause
 Im Rahmen des Minimalistischen Programms gibt es eine Inflation leerer Köpfe.

\end{itemize}

}

\subsubsection{Grammatische Funktionen wie Subjekt und Objekt}

\frame{
\frametitle{Grammatische Funktionen wie Subjekt und Objekt}



}

\subsubsection{Eigenschaften von Fernabhängigkeiten}

\frame{
\frametitle{Eigenschaften von Fernabhängigkeiten}

Eine Kernbeschränkung für Bewegung gilt schon nicht für das Deutsche,\\
kann also auch keine universale Beschränkung sein \citep{Mueller2004d}.

}

\subsubsection{Grammatische Morpheme für Tempus, Modus und Aspekt}

\frame{
\frametitle{Grammatische Morpheme für Tempus, Modus und Aspekt}


}

\subsubsection{Wortarten (Nomen und Verb)}

\frame{
\frametitle{Wortarten (Nomen und Verb)}



}

\subsubsection{Rekursion}

\frame{
\frametitle{Rekursion}

\begin{itemize}
\item \citet*{HCF2002a}: Die einzige domänenspezifische Universalie ist Rekursion.
\item Eventuell gibt es Sprachen, die keine Rekursion verwenden:
      \begin{itemize}
      \item Pirah{\~a} \citep{Everett2005a-u}\\
      (siehe jedoch auch \citew*{NPR2009a-u})
      \item Walpiri \citep{Hale76a}
      \end{itemize}
\item Rekursion gibt es auch im nichtsprachlichen Bereich:
      \begin{itemize}
      \item Planung
      \item Stammbäume
      \end{itemize}
\end{itemize}

}

\subsection{Zusammenfassung}

\frame{
\frametitle{Zusammenfassung}

Es gibt keine linguistischen Universalien, bei denen man sich einig ist, dass man
domänenspezifisches angeborenes Wissen braucht.

}


\subsection{Kritische Periode für den Spracherwerb}

\frame{
\frametitle{Kritische Periode für den Spracherwerb}

\begin{itemize}
\item Bei Enten gibt es eine kritische Phase, in der deren Bezugsverhalten geprägt wird.
\item Kinder erlernen Sprache besser als Erwachsene.
\item Es ist jedoch nicht so, dass nach einer bestimmten Zeit der Spracherwerb unmöglich wird.
\item Kein abrupter Übergang wie bei Enten sondern ein stetiger Abfall.
\item Das ist jedoch auch in anderen Domänen beobachtbar:
      \zb ist Autofahren in hohem Alter schwerer erlernbar.
\item Die geringere Gehirnkapazität von Kindern kann zu einer Vereinfachung der Wahrnehmung der
  Input-Daten führen und somit den Kindern beim Spracherwerb helfen. ("`Weniger ist mehr"'-Hypothese) 
\end{itemize}

}

\subsection{Kein Spracherwerb bei Primaten}

\frame{
\frametitle{Kein Spracherwerb bei nichtmenschlichen Primaten}

\begin{itemize}
\item Nichtmenschliche Primaten verstehen Zeigegesten nicht.
\item Menschen imitieren gern.
\item Nichtmenschlichen Primaten könnten die sozialen/kognitiven Voraussetzungen für Sprache fehlen.
\end{itemize}

}


\subsection{Pidgin-Sprachen}

\frame{
\frametitle{Pidgin-Sprachen}

\begin{itemize}
\item Kinder regulasieren Sprache (\zb Pidgin-Sprachen)
\item Dies kann man jedoch als probability matching erklären,\\
      das auch in anderen Domänen auftritt.
\item Beispiel: Zwei Glühlampen, die blinken. Wenn eine in 70\% der Fälle blinkt, können
  Versuchspersonen das vorhersagen.

Bei drei Glühlampen, von denen eine in 70\% der Fälle blinkt und die anderen in je 15\%, sagen die
Probanden 80--90\% für die häufiger blinkende voraus.
\end{itemize}

}

\frame{
\frametitle{Regularisierung}

\begin{itemize}[<+->]
\item Bringt man Versuchspersonen eine künstliche Sprache bei, in der Determinierer in 60\% der Fälle
  ausgedrückt werden, produzieren sie ebenfalls in ca. 60\% der Fälle Determinierer. 

\item Bringt man den Probanden eine Sprache bei, in der zu 60\% ein Determinierer ausgedrückt wird und
  zusätzlich noch weitere Determinierer, die weniger häufig auftreten, dann erzeugen sie den
  Determinierer in mehr als 80\% der Fälle, \dash sie regularisieren. 

\item Kinder regularisieren stärker als Erwachsene.
\item Erinnern Sie sich an die "`Weniger ist mehr"'-Hypothese.
\end{itemize}

}

\subsection{Lokalisierung in speziellen Gehirnbereichen}

\frame{
\frametitle{Lokalisierung in speziellen Gehirnbereichen}

\begin{itemize}
\item Sprachvermögen ist in speziellen Gehirnbereichen lokalisiert.
\item Allerdings können bei Beschädigung auch andere Gehirnbereiche diese Funktion übernehmen.
\item Beim Lesen wird auch ein bestimmter Gehirnbereich aktiviert.\\
Man würde allerdings daraus nie schließen,\\
dass die Fähigkeit zu lesen angeboren ist.
\end{itemize}

}

\subsection{Verschiedenheit von Sprache und allgemeiner Kognition}

\frame{
\frametitle{Verschiedenheit von Sprache und allgemeiner Kognition}

}

\subsection{Williams-Syndrom}

\frame{
\frametitle{Williams-Syndrom}

}

\subsection{KE-Familie mit FoxP2-Mutation}

\frame{
\frametitle{KE-Familie mit FoxP2-Mutation}

}

\subsection{Poverty of the Stimulus}

\frame{
\frametitle{Poverty of the Stimulus}

}
