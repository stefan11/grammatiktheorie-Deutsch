%% -*- coding:utf-8 -*-

\section{Generalized Phrase Structure Grammar (GPSG)}

\outline{

\begin{itemize}
\item Begriffe
\item Phrasenstrukturgrammatiken
\item Government \& Binding (GB)
\item \blau{Generalisierte Phrasenstrukturgrammatik (GPSG)}
\item Lexikalisch-Funktionale Grammatik (LFG)
%\item Lexical Mapping Theory (LMT)
%\item PATR
\item Kategorialgrammatik (CG)
\item Kopfgesteuerte Phrasenstrukturgrammatik (HPSG)
\item Konstruktionsgrammatik (CxG)
\item Baumadjunktionsgrammatik (TAG)
\end{itemize}
}

\frame{
\frametitle{Generalized Phrase Structure Grammar (GPSG)}

\begin{itemize}
\item Die GPSG wurde als Antwort auf die Transformationsgrammatik Ende der 70er Jahre entwickelt.
\pause
\item Hauptpublikation: das Buch von \citet*{GKPS85a}.
\pause
\item Eine GPSG für das Deutsche wurde von \citet{Uszkoreit87a} entwickelt.
\pause
\item Chomsky hat gezeigt, dass Phrasenstrukturgrammatiken inadäquat sind.\\
      GPSG erweitert PSG so, dass für Chomskys Punkte eine adäquate Behandlung möglich wird:
\begin{itemize}
\item keine Beschränkung auf atomare Kategorien \citep{Harman63a}
\pause
\item andere Behandlung der lokalen Umstellung
\pause
\item Passiv als Meta-Regel
\pause
\item Fernabhängigkeiten als Folge lokaler Abhängigkeiten
\end{itemize}
\pause
\item Wir gucken diese Erweiterungen im folgenden an.
\end{itemize}


}

\subsection{Allgemeines zum Repräsentationsformat}

\frame[shrink=0]{
\frametitle{Allgemeines zum Repräsentationsformat}

\begin{itemize}
\item Kategorien sind Mengen von Merkmal-Wert-Paaren
\pause
\item Lexikoneinträge haben ein Merkmal namens \subcat. Der Wert ist eine Zahl.
      Diese Zahl sagt etwas darüber aus, in welcher Grammatikregel das Wort verwendet werden kann.
\pause
\item Beispiele nach \citew{Uszkoreit87a}:

\begin{tabular}{@{}l@{~$\to$~}ll@{}}
V2  & H[5]                              & (kommen, schlafen)\\
V2  & H[6], N2[Case Acc]                & (kennen, suchen)\\
V2  & H[7], N2[Case Dat]                & (helfen, vertrauen)\\
V2  & H[8], N2[Case Dat], N2[Case Acc]  & (geben, zeigen)\\
V2  & H[9], V3[+dass]                   & (wissen, glauben)\\
\end{tabular}

Diese Regeln lizenzieren VPen, \dash die Kombination eines Verbs mit seinen Komplementen, aber nicht
mit dem Subjekt.

\pause
\item Die Zahlen nach den Kategoriesymbolen (V bzw.\ N) geben die \xbar-Stufe an.\\
      Das Maximum ist bei Uszkoreit drei, nicht wie oft angenommen zwei.
\pause
\item H steht für Head.

\end{itemize}

}

\subsection{Prinzipien: Das Kopf-Merkmalsprinzip}

\frame{
\frametitle{Prinzipien: Das Kopf-Merkmalsprinzip}

Head Feature Convention:\\
Mutterknoten und Head-Tochter müssen in allen Head-Merkmalen übereinstimmen,\\
außer wenn die Merkmale mit explizitem Wert vorgegeben sind.



}

\subsection{Lokale Umstellung}

\frame{
\frametitle{Lokale Umstellung}


\begin{itemize}
\item Im \mf können Argumente in nahezu beliebiger Abfolge angeordnet werden.
\eal
\ex {}[weil] \rot{der Mann} \gruen{der Frau} \blau{das Buch} gibt
\ex {}[weil] \rot{der Mann} \blau{das Buch} \gruen{der Frau} gibt
\ex {}[weil] \blau{das Buch} \rot{der Mann} \gruen{der Frau} gibt
\ex {}[weil] \blau{das Buch} \gruen{der Frau} \rot{der Mann} gibt
\ex {}[weil] \gruen{der Frau} \rot{der Mann} \blau{das Buch} gibt
\ex {}[weil] \gruen{der Frau} \blau{das Buch} \rot{der Mann} gibt
\zl
\pause
\item In (\mex{0}b--f) muß man die Konstituenten anders betonen
und die Menge der Kontexte, in denen der Satz mit der jeweiligen Abfolge
geäußert werden kann, ist gegenüber (\mex{0}a) eingeschränkt \citep{Hoehle82a}. 

Abfolge in (\mex{0}a) = \blau{Normalabfolge} bzw.\ die \blau{unmarkierte Abfolge}.
\end{itemize}
}



\frame{
\frametitle{Motivation für Linearisierungsregeln (I)}

%\smallframe
\savespace
Motivation: Permutation mit Phrasenstrukturregeln $\to$\\
braucht für ditransitive Verben sechs Phrasenstrukturregeln für Verbletztstellung:
\ea
\begin{tabular}[t]{@{}l@{ }l@{ }l@{ }l@{ }l@{ }}
S  & $\to$ NP[nom]& NP[acc] & NP[dat] & V\\
S  & $\to$ NP[nom]& NP[dat] & NP[acc] & V\\
S  & $\to$ NP[acc]& NP[nom] & NP[dat] & V\\
S  & $\to$ NP[acc]& NP[dat] & NP[nom] & V\\
S  & $\to$ NP[dat]& NP[nom] & NP[acc] & V\\
S  & $\to$ NP[dat]& NP[acc] & NP[nom] & V\\
\end{tabular}
\z
}

\frame{
\frametitle{Motivation für Linearisierungsregeln (II)}

Plus sechs Regeln für Verberststellung:

\ea
\begin{tabular}[t]{@{}l@{ }l@{ }l@{ }l@{ }l}
S  & $\to$ V NP[nom]& NP[acc] & NP[dat]\\
S  & $\to$ V NP[nom]& NP[dat] & NP[acc]\\
S  & $\to$ V NP[acc]& NP[nom] & NP[dat]\\
S  & $\to$ V NP[acc]& NP[dat] & NP[nom]\\
S  & $\to$ V NP[dat]& NP[nom] & NP[acc]\\
S  & $\to$ V NP[dat]& NP[acc] & NP[nom]\\
\end{tabular}
\z

Die Regeln erfassen eine Generalisierung nicht.

Genauso für transitive Verben und entsprechende andere Valenzrahmen.

}

\frame{
\frametitle{Abstraktion von linearer Abfolge: Dominanz}

\begin{itemize}
\item \citet*{GKPS85a}:\\
      Trennung von unmittelbarer Dominanz (\emph{immediate dominance} = ID) und
      linearer Abfolge (\emph{linear precedence} = LP)
\pause
\item Dominanzregeln sagen nichts über die Reihenfolge der Töchter.
\ea
\begin{tabular}[t]{@{}l@{ }l}
S  & $\to$ V, NP[nom], NP[acc], NP[dat]\\
\end{tabular}
\z

Regel in (\mex{0}) sagt nur, dass S die anderen Knoten dominiert:
\medskip
\centerline{%
\begin{forest}
sm edges
[S
  [V]
  [{NP[nom]}]
  [{NP[acc]}]
  [{NP[dat]}] ]
\end{forest}}
\medskip

\pause
\item wegen Aufhebung der Anordnungsrestriktion für die rechte Regelseite:\\
      statt zwölf Regeln nur noch eine 

\end{itemize}
}


\frame{
\frametitle{Abstraktion von linearer Abfolge: Lineare Abfolge}

\begin{itemize}
\item LP"=Beschränkungen über lokale Bäume, \dash Bäume der Tiefe eins:


\begin{table}[H]
\centerline{%
\begin{forest}
sm edges
[S
  [V]
  [{NP[nom]}]
  [{NP[acc]}]
  [{NP[dat]}] ]
\end{forest}}
\end{table}

~\medskip

      $\to$ Können etwas über die Abfolge von V, NP[nom], NP[acc] und NP[dat] sagen.


\end{itemize}
}


\frame{
\frametitle{Erneute Formulierung von Restriktionen}


\begin{itemize}
\item ohne Restriktionen für die rechte Regelseite gibt es zu viel Freiheit

\medskip
\begin{tabular}[t]{@{}l@{ }l}
S  & $\to$ V, NP[nom], NP[acc], NP[dat]\\
\end{tabular}

\medskip
Die Regel läßt Abfolgen mit dem Verb zwischen NPen zu:
\ea[*]{
Der Frau der Mann gibt ein Buch.
}
\z
\pause
\item Linearisierungsregeln schließen solche Anordnungen dann aus.

\ea
\begin{tabular}[t]{@{}l@{~$<$~}l@{}}
V[+MC]  & X\\
X       & V[$-$MC]\\
\end{tabular}
\z
{\sc mc} steht hierbei für \emph{main clause}. 

Die LP-Regeln stellen sicher, dass das Verb in Hauptsätzen (+{\sc mc}) vor allen anderen
Konstituenten steht und in Nebensätzen ($-${\sc mc}) nach allen anderen.

\end{itemize}
}

\subsection{Meta-Regeln}

\frame[shrink=10]{
\frametitle{Einführung des Subjekts über eine Meta-Regel}

Bisher sehen unsere Regeln aber so aus:

\ea
\begin{tabular}[t]{@{}l@{~$\to$~}ll@{}}
V2  & H[7], N2[Case Dat]                & (helfen, vertrauen)\\
V2  & H[8], N2[Case Dat], N2[Case Acc]  & (geben, zeigen)\\
\end{tabular}
\z

\pause

Mit (\mex{0}) können wir nur VPen, aber keine Sätze mit Subjekt analysieren.

\pause

Verwenden Meta-Regel, die sagt: Wenn es in der Grammatik eine Regel der Form "`V2 besteht aus irgendwas"' gibt,
dann muß es auch eine Regel\\ "`V3 besteht aus dem, woraus V2 besteht + einer NP im Nominativ"' geben.

\pause

Formal:
\ea
V2  $\to$ W $\mapsto$\\
V3  $\to$ W, N2[Case Nom]
\z

W steht dabei für eine beliebige Anzahl von Kategorien.

}

\frame{
\frametitle{Einführung des Subjekts über eine Meta-Regel}



\ea
V2  $\to$ W $\mapsto$\\
V3  $\to$ W, N2[Case Nom]
\z



\pause

Diese Meta-Regel erzeugt aus den Regeln in (\mex{1}) die Regeln in (\mex{2}):
\ea
\begin{tabular}[t]{@{}l@{~$\to$~}ll@{}}
V2  & H[7], N2[Case Dat]                & (helfen, vertrauen)\\
V2  & H[8], N2[Case Dat], N2[Case Acc]  & (geben, zeigen)\\
\end{tabular}
\z

\ea
\begin{tabular}[t]{@{}l@{~$\to$~}l@{}}
V3  & H[7], N2[Case Dat], N2[Case Nom]                \\
V3  & H[8], N2[Case Dat], N2[Case Acc], N2[Case Nom]  \\
\end{tabular}
\z

\pause

Damit stehen Subjekt und andere Argumente gemeinsam auf der rechten Seite der Regel und können also
beliebig angeordnet werden,\\
so lange die LP-Regeln nicht verletzt sind.

}










\subsection{Passiv als Meta-Regel}

\frame[shrink=15]{
\frametitle{Passiv als Meta-Regel}
\savespace\smallexamples

Beim Passiv passiert folgendes:
\begin{itemize}
\item Das Subjekt wird unterdrückt.
\item Wenn es ein Akkusativobjekt gibt, wird dieses zum Subjekt.
\end{itemize}

Das gilt für alle Verbklassen, die ein Passiv bilden können. Dabei ist es egal,
ob die Verben einstellig, zweistellig oder dreistellig sind:
\eal
\ex weil er noch gearbeitet hat
\ex weil noch gearbeitet wurde
\zl
\eal
\ex weil er an Maria gedacht hat
\ex weil an Maria gedacht wurde
\zl
\eal
\ex weil sie ihn geschlagen hat
\ex weil er geschlagen wurde
\zl
\eal
\ex weil er ihm den Aufsatz gegeben hat
\ex weil ihm der Aufsatz gegeben wurde
\zl

}

\frame[shrink=5]{
\frametitle{Passiv als Meta-Regel}

\begin{itemize}
\item In der PSG müßte man für jedes Satzpaar jeweils zwei Regeln aufschreiben.
\pause
\item In GPSG gibt es Meta-Regeln für das Passiv.

\pause
\item Zu jeder Aktiv-Regel mit Subjekt und Akkusativobjekt wird eine Passiv-Regel mit unterdrücktem
  Subjekt lizenziert. Der Zusammenhang ist also erfaßt.
\pause
\item Unterschied zu Transformationsgrammatik/GB:\\
      Es gibt nicht zwei Bäume, die in Beziehung zueinander gesetzt werden,\\
      sondern jeweils Aktiv-Regeln werden zu Passiv-Regeln in Beziehung gesetzt. 

      Mit den Aktiv- bzw. Passiv-Regeln kann man dann zwei unabhängige Strukturen ableiten, \dash
      (\mex{1}b) ist nicht auf (\mex{1}a) bezogen.

\eal
\ex weil sie ihn geschlagen hat
\ex weil er geschlagen wurde
\zl


      Die Generalisierung in bezug auf Aktiv/Passiv ist aber dennoch erfaßt.
\end{itemize}

}

\frame{
\frametitle{Passiv im Englischen}

\citet*{GKPS85a} schlagen folgende Meta-Regel vor:
\ea
VP  $\to$ W, NP $\mapsto$\\
VP[PAS]  $\to$ W, (PP[\emph{by}])
\z
Diese Regel besagt, dass Verben, die ein Objekt verlangen, in einer Passiv-VP auch ohne dieses Objekt
auftreten können. 

Bei den Passiv-VPen kann eine \emph{by}-PP angeschlossen werden.

(VP entspricht V2)

}

\frame{
\frametitle{Probleme der VP-bezogenen Passivmetaregel}

\begin{enumerate}
\item Regel nimmt nicht auf Verb bezug (nicht alle Verben erlauben Passivierung).
\pause
% \item Es ist unklar, wie die Semantik parallel zur Syntax aufgebaut werden soll:
%       Die Regel in (\mex{0}) unterdrückt ein NP-Argument in der VP. 
%       Dieses Argument ist aber ein Objekt. Rein syntaktisch stellt das System der GPSG-Metaregeln
%       die richtige Menge von ID-Regeln zur Verfügung, in der Semantik muss man aber sicherstellen,
%       dass das durch eine Metaregel eingeführte Subjekt mit dem unterdrückten Objekt übereinstimmt.
% \pause
\item Unpersönliches Passiv kann nicht durch Unterdrückung eines Objekts abgeleitet werden.
      Eine uneinheitliche Behandlung des Passivs scheint unumgänglich zu sein, wenn man (\mex{0})
      aufrechterhalten will.
\end{enumerate}


}


\subsection{Fernabhängigkeiten als Folge lokaler Abhängigkeiten}


\frame{
\frametitle{Fernabhängigkeiten als Folge lokaler Abhängigkeiten}

\begin{itemize}
\item Bisher können wir nur Verberst- und Verbletztstellung erklären.
\eal
\ex {}[dass] der Mann der Frau das Buch gibt
\ex Gibt der Mann der Frau das Buch?
\zl
\item Was ist mit Verbzweitstellung?
\eal
\ex Der Mann gibt der Frau das Buch.
\ex Der Frau gibt der Mann das Buch.
\zl
\item Verbzweitstellung wird als Fernabhängigkeit mittels einer Folge lokaler Abhängigkeiten
 analysiert.

 Eine der großen Leistungen bei der
  Entwicklung der GPSG besteht in der Entwicklung einer Alternative zu Transformationen für die
  Analyse von Fernabhängigkeiten. (siehe aber auch schon \citew{Harman63a})
\end{itemize}

}

\subsubsection{Meta-Regel zur Einführung von Fernabhängigkeiten}


\frame{
\frametitle{Meta-Regel zur Einführung von Fernabhängigkeiten}

Nehmen beliebige Kategorie X aus der Menge der Kategorien auf der rechten Regelseite und
repräsentieren sie auf der linken Seite mit Slash (`/'):
\ea
V3  $\to$ W, X $\mapsto$\\
V3/X  $\to$ W
\z
\pause

Diese Regel erzeugt aus (\mex{1}) die Regeln in (\mex{2}):
\ea
\begin{tabular}[t]{@{}l@{~$\to$~}l@{}}
V3  & H[8], N2[Case Dat], N2[Case Acc], N2[Case Nom] 
\end{tabular}
\z
\ea
\begin{tabular}[t]{@{}l@{~$\to$~}l@{}}
V3/N2[Case Nom] &  H[8], N2[Case Dat], N2[Case Acc]\\
V3/N2[Case Dat] &  H[8], N2[Case Acc], N2[Case Nom]\\
V3/N2[Case Acc] &  H[8], N2[Case Dat], N2[Case Nom]\\
\end{tabular}
\z

}

\subsubsection{Regel zur Abbindung von Fernabhängigkeiten}

\frame{
\frametitle{Regel zur Abbindung von Fernabhängigkeiten}


\ea
V3[+Fin] $\to$ X[+Top], V3[+MC]/X
\z
X steht dabei für eine beliebige Kategorie,\\
die per `/' in V3 als fehlend markiert wurde.

\pause
Die für unsere Beispiele interessanten Fälle zeigt (\mex{1}):
\ea
\begin{tabular}[t]{@{}l@{~$\to$~}l@{~}l@{}}
V3[+Fin] & N2[+Top, Case Nom], & V3[+MC]/N2[Case Nom]\\
V3[+Fin] & N2[+Top, Case Dat], & V3[+MC]/N2[Case Dat]\\
V3[+Fin] & N2[+Top, Case Acc], & V3[+MC]/N2[Case Acc]\\
\end{tabular}
\z

\pause
Linearisierungsregel sorgt dafür, dass X vor dem restlichen Satz steht:
\ea
{}[+Top] $<$ X
\z


}


\subsubsection{Eine Beispielanalyse}

\frame{
\frametitle{Eine Beispielanalyse}


\centerline{%
\scalebox{0.85}{
\begin{forest}
sm edges
[{V3[+\textsc{fin}, $+$\textsc{mc}]}
  [{N2[dat,+\textsc{top}]} [dem Mann,roof] ]
  [{V3[+\textsc{mc}]/N2[dat]}
    [{V[8,+\textsc{mc}]} [gibt] ]
    [{N2[nom]} [er] ] 
    [{N2[acc]} [das Buch, roof] ] ] ]
\end{forest}
}}

\begin{itemize}
\item Metaregel lizenziert Regel, die Dativobjekt in Slash einführt.
\item Diese wird angewendet und lizenziert den Teilbaum für \emph{gibt er das Buch}.
\item Lineariesierungsregel ordnet Verb ganz links an (V[+MC] $<$ X).
\item Im nächsten Schritt wird die Konstituente hinter dem Slash abgebunden.
\end{itemize}

}

\subsubsection{Ein Beispiel mit Fernabhängigkeit}

\frame{
\frametitle{Ein Beispiel mit Fernabhängigkeit (I)}

In (\mex{1}) hängen alle NPen vom selben Verb ab:
\ea
Dem Mann gibt er das Buch.
\z
Man könnte sich ein kompliziertes System von Linearisierungsregeln einfallen
lassen, das es dann erlaubt, (\mex{0}) mit einer ganz flachen Struktur zu analysieren.
\pause

Allerdings braucht man dann immer noch eine Analyse für:

\eal
\ex Wen glaubst du, dass ich getroffen habe?
\ex Gegen ihn falle es den Republikanern hingegen schwerer,\\
    Angriffe zu lancieren.\footnote{
  taz, 08.02.2008, S.\,9
}
\zl

(\mex{0}) kann man nicht über lokale Umstellung erklären, denn \emph{wen} hängt nicht von
\emph{glaubst} sondern von \emph{getroffen} ab und \emph{getroffen} befindet sich in einem anderen Teilbaum.

}

\frame{
\frametitle{Ein Beispiel mit Fernabhängigkeit (II)}

\begin{itemize}
\item Die Analyse von (\mex{1}) besteht aus mehreren Schritten: Einführung, Weitergabe und Abbindung
  der Information über die Fernabhängigkeit. 
\ea
Wen glaubst du, dass ich getroffen habe?
\z

\pause

\item \emph{ich getroffen habe} ist V3/NP[acc]\\
(durch Metaregel lizenzierte Grammatikregel)


\item \emph{dass ich getroffen habe} ist V3/NP[acc]\\
      (Weitergabe der \slasch"=Information)

\pause
\item \emph{glaubst du, dass ich getroffen habe} ist V3/NP[acc]\\
      (Weitergabe der \slasch"=Information)

\pause
\item \emph{Wen glaubst du, dass ich getroffen habe} ist V3\\
      (Abbindung der \slasch"=Information in Grammatikregel)

\end{itemize}

}

\frame{
\frametitle{Ein Beispiel mit Fernabhängigkeit (III)}


\vfill
\centerline{%
\scalebox{0.75}{%
\begin{forest}
sm edges,empty nodes
[{V3[+\textsc{fin},+\textsc{mc}]}
  [{N2[acc,+\textsc{top}]} [wen] ]
  [{V3[+\textsc{mc}]/N2[acc]}
    [{V[9,+\textsc{mc}]} [glaubst] ]
    [{N2[nom]} [du] ] 
    [{V3[+dass,$-$\textsc{mc}]/N2[acc]} 
      [{}[dass] ]
      [{V3[$-$dass,$-$\textsc{mc}]/N2[acc]} 
         [{N2[nom]} [ich] ]
         [{V[6,$-$\textsc{mc}]} [gesehen habe,roof] ] ] ] ] ]
\end{forest}
}
}

\vfill
Hierbei habe ich vereinfachend angenommen,\\
dass \emph{getroffen habe} sich wie ein einfaches transitives Verb verhält.
\vfill

}
