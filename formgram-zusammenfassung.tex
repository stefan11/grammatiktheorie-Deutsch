%% -*- coding:utf-8 -*-


\section{Zusammenfassung}


\frame{
\frametitle{Kopflose, idiosynkratische Strukturen}

\begin{itemize}
\item Es gibt Strukturen, die nicht der \xbart entsprechen.

\citet{Matsuyama2004a} und \citet{Jackendoff2008a}:
\eal
\ex Student after student left the room.
\ex
\label{ex-npn-iteration}
Day after day after day went by, but I never found the courage to talk to
her. \citep{Bargmann2015a}
\zl

\pause

\item Diese sind problematisch für CG, DG und Minimalismus, weil diese Theorien einen Kopf/Funktor
  verlangen.

\pause
\item GPSG, LFG, HPSG, CxG, TAG haben damit kein Problem, weil man beliebige Strukturen mit einer
  Bedeutung kombinieren kann.

\end{itemize}


}

\frame{
\frametitle{Übersicht}


\oneline{%
\begin{tabular}{@{}lllll@{}}
Theorie & V1                & V2                     & Passiv       & Scrambling\\
GB      & Bewegung          & Bewegung               & lexikalisch  & Bewegung/Basisgenerierung\\
GPSG    & ID/LP flach       & \slasch                & Metaregel    & ID/LP flach\\
LFG     & co-heads          & functional uncertainty & Lexikonregel & Unterspezifikation von Kasus\\
CG      & direkt/leerer Kopf & Typanhebung \ldots    & Lexikonregel & direkte Kombination\\
HPSG    & \dsl (Kopfbewegung) & \slasch (X(P)-Bewegung) & Lexikonregel  & direkte Kombination\\     
SBCG    & ist HPSG-Variante\\
CxG     & flach?            & flach?                 & Allostruktion & ? Unterspezifikation ID/LP?\\
Minimalism & Bewegung       & Bewegung               & alternatives \textit{v} & Bewegung/Basisgenerierung\\
\end{tabular}}


}

\frame{
\frametitle{Zusammenfassung der gesamten Veranstaltung}

\begin{itemize}
\item Irgendwie machen wir alle dasselbe.
\item Irgendwie dann aber doch nicht.
\end{itemize}



}
