\subsection{Probleme phrasaler Ansätze}

\outline{

\begin{itemize}
\item Resultativkonstruktionen
\item Phrasale Analysen
\item Interaktion mit anderen Phänomenen
\item Eine lexikalische Analyse
\item Zusammenfassung
\end{itemize}

}


\subsubsection{Resultativkonstruktionen}

\frame{

\frametitle{Resultativkonstruktionen}

Ein \rot{Verb} (meist einstellig) plus \gruen{Akkusativ} + \blau{Prädikat}.\\
Akkusativ muss nicht vom Verb selegiert sein:
%% \eal
%% \ex Heute verzichten die Hooligans vor und beim Fußballspiel auf Alkohol und \rot{trinken} 
%%       erst nach dem Spiel \gruen{ganze Kneipen} \blaubf{leer}\iw{leer}.\label{ex-hooligans-trinken-kneipen-leer}\footnote{
%%         Mannheimer Morgen, 16.07.1998, Politik; Kanther sagt Hooligans den Kampf an
%% %       M98/807.58811 Mannheimer Morgen, 16.07.1998, Politik; Kanther sagt Hooligans den Kampf an
%% }
%% \ex "`Als ich anfing, wollte ich mir eigentlich \gruen{den Hintern} nicht so \blaubf{platt}\rot{sitzen} 
%% wie die älteren Grufties"', sagt Pape.\footnote{
%%         taz-Bremen, 07.02.1997, S.\,21% Nr. 5267 Seite 21 vom 02.07.1997 
%% }

%% \zl

\ea
weil er \gruen{den Teich} \blau{leer} \rot{fischt}.
\z

\pause
Konkurrierende Hypothesen: 
\begin{enumerate}
\item Die resultative Bedeutung ist in keinem der beteiligten Wörter oder Wortgruppen enthalten.\\
      Sie wird vielmehr von der gesamten Konfiguration beigesteuert.\\
\citep{Goldberg95a,Jackendoff97a,GJ2004a}\nocite{FKoC88a,KF99a}
\pause
\item Es gibt einen besonderen Lexikoneintrag für Verben in RKen.\\
      Dieser kommt nur in den RKen vor und steuert die entsprechende Bedeutung bei.
%\oneline{
(\citealp{Verspoor97a,Wechsler97a};\\\citealp{WN2001a,Wunderlich92a-u-kopiert,Mueller2002b})
%}
\end{enumerate}

}

\subsubsection{Phrasale Analysen}

%% \frame{
%% \frametitle{Phrasale Analysen}

%% \begin{itemize}
%% \item Goldberg \& Jackendoff schlagen unabhängig und auch gemeinsam phrasale Analysen
%%       vor.\\
%%       Konsequenzen sind jedoch unterschiedlich, da beide in verschiedenen Frameworks arbeiten:\\
%%       Goldberg: Konstruktionsgrammatik\\
%%       Jackendoff: Government \& Binding 
%% \pause
%% \item Konstruktionsgrammatik:
%%       \begin{itemize}
%%       \item benutzt keine Transformationen \citep[S.\,7]{Goldberg95a},\\
%%             ist monostratal \citep{KF99a}
%% \pause
%%       \item Generalisierungen werden in Typhierarchien erfaßt
%%       \end{itemize}

%% \pause
%% \item Ich diskutiere im folgenden hauptsächlich Goldbergs Vorschlag.
%% \end{itemize}


%% }


\frame{
\frametitle{Phrasale Analysen: \citet{Goldberg95a}}


\begin{itemize}
\item Goldberg schlägt auf S.\,192 folgende Struktur für Resultativkonstruktionen vor: 
\ea
{}[\node{subj}{SUBJ} [\node{v}{V} \node{obj}{OBJ} \node{obl}{OBL}]]
\z

\ea
\node{he}{He} \node{fishes}{fishes} \node{thepond}{the pond} \node{empty}{empty}.
\z
\visible<1>{%
\anodeconnect[t]{he}[b]{subj}%
\anodeconnect[t]{fishes}[b]{v}%
\anodeconnect[t]{thepond}[b]{obj}%
\anodeconnect[t]{empty}[b]{obl}%
}

\pause
\item Braucht noch weitere \emph{Konstruktionen}, \zb für Passiv.
\ea
The pond was fished empty.
\z
\ea
{}[SUBJ [V-pass V OBL]]
\z

\pause
%% Goldberg spricht von "`Inheritance Links"' die Beziehung zwischen Konstruktionen (\zb Aktiv und Passiv)
%% herstellen.

%% Das entspricht aber wohl den Transformationen in der Transformationsgrammatik.

Goldberg nimmt an, daß Aktiv- und Passivvarianten der RK in einer Typhierarchie
beschrieben werden (\citealt{Kay2005a}; G \& J, 2004).

\end{itemize}


}

\subsubsubsection{Typhierarchien}

%\frame{
%\frametitle{Typhierarchien}


%\vfill
%%\oneline{%
%\hfill\begin{tabular}{cccc}
%\multicolumn{4}{c}{\node{ed}{\it elektrisches Gerät}}\\[6ex]
%\node{p}{\it druckendes Gerät} & & \node{sc}{\it scannendes Gerät} & \node{other}{\rule[-0.5ex]{0cm}{2.5ex}\ldots}\\[6ex]
%\node{printer}{\it Drucker}   & \node{copy}{\it Kopierer}  & \node{scanner}{Scanner}\\[6ex]
%\node{l-p}{\it Laserdrucker}  & \node{other-p}{\rule[-0.5ex]{0cm}{2.5ex}\ldots}  & \node{negscan}{\it Negativscanner} & \node{other-sc}{\rule[-0.5ex]{0cm}{2.5ex}\ldots}\\
%\end{tabular}
%\nodeconnect{ed}{p}\nodeconnect{ed}{sc}\nodeconnect{ed}{other}%
%\nodeconnect{p}{copy}\nodeconnect{p}{printer}\nodeconnect{printer}{l-p}\nodeconnect{printer}{other-p}%
%\nodeconnect{sc}{copy}%
%\nodeconnect{sc}{scanner}\nodeconnect{scanner}{negscan}\nodeconnect{scanner}{other-sc}%
%%}
%\hfill\hfill\mbox{}
%\vfill

%}


\frame{
\frametitle{Typhierarchien}


\vfill
\oneline{%
\hfill\begin{tabular}{cccc}
\multicolumn{4}{c}{\node{c}{\it construction}}\\[6ex]
\node{a}{\it active} & \node{p}{\it passive} & \node{m}{\it middle} & \node{r}{\it resultative}\\[6ex]
&\node{ar}{\it active resultative}   & \node{pr}{\it passive resultative}  & \node{mr}{\it middle resultative}\\
\end{tabular}
\nodeconnect{c}{a}\nodeconnect{c}{p}\nodeconnect{c}{m}\nodeconnect{c}{r}%
\nodeconnect{a}{ar}\nodeconnect{p}{pr}\nodeconnect{m}{mr}%
\nodeconnect{r}{ar}\nodeconnect{r}{pr}\nodeconnect{r}{mr}%
}
\hfill\hfill\mbox{}
\vfill
\begin{itemize}
\item Abbildung zeigt Ausschnitt aus einer möglichen Hierarchie
\item weitere Typen für Aktiv-, Passiv- bzw.\ Medial-Konstruktionen:\\
     \zb Heavy-NP-Shift im Aktiv/Passiv usw.
\end{itemize}
\vfill

}

\subsubsection{Interaktion mit anderen Phänomenen}

\outline{

\begin{itemize}
\item Resultativkonstruktionen
\item Phrasale Analysen
\item \blau{Interaktion mit anderen Phänomenen}
\item Eine lexikalische Analyse
\item Zusammenfassung
\end{itemize}

}



\frame{
\frametitle{Goldbergs Analyse für das Deutsche}

%\savespace

\begin{itemize}
\item Goldberg erwähnt nur Interaktionen mit Passiv und der Medialkonstruktion.
\pause
\item Es gibt aber viel mehr interagierende Phänomene.
\pause
\item Das soll im folgenden an Hand der Übertragung der Analyse\\
      auf das Deutsche untersucht werden.
\pause
\item Für das Deutsche würde sie wohl folgende Konstruktion annehmen:
\ea
{}[SUB OBJ OBL V]
\z
\end{itemize}


}


\subsubsubsection{Umstellung von Konstituenten und Verbstellung}

\frame{

\frametitle{Umstellung von Konstituenten und Verbstellung}

Das Argument von \emph{leer} kann vor Argumenten des Matrixverbs stehen:
\eal
\ex weil \blau{niemand} \gruenbf{den Teich} leer fischt.
\ex weil \gruenbf{den Teich} \blau{niemand} leer fischt.
\zl

\pause
Verb kann initial oder final stehen. Man braucht also:
\ea
\begin{tabular}[t]{@{~}l@{~}l@{\hspace{2em}}l@{~}l@{}}
a. & [\blau{SUB} \gruen{OBJ} OBL V] & c. & [V \blau{SUB} \gruen{OBJ} OBL]\\
b. & [\gruen{OBJ} \blau{SUB} OBL V] & d. & [V \gruen{OBJ} \blau{SUB} OBL]\\
\end{tabular}
\z

%bochum
%(Könnte man auch über Trennung von Dominanz und Präzedenz machen,
%\hspaceThis{(}es gibt aber informationsstrukturelle Unterschiede,\\
%\hspaceThis{(}weshalb wohl \Konstruktionen angenommen würden.)

}

\subsubsubsection{Fokusumstellung}

\frame{

\frametitle{Fokusumstellung}

\citet{Neeleman94a} für das Niederländische:
\eal
\ex daß \gruenbf{so   grün} selbst Jan die Tür nicht streicht
\ex daß \gruenbf{so grün} die Tür selbst Jan nicht streicht
\ex daß Jan \gruenbf{so grün} selbst die Tür nicht streicht
\ex daß eine solche Tür \gruenbf{so grün} niemand streicht
\zl
Siehe auch \citew{Luedeling2001a}.

\pause
Man braucht:
\ea
\begin{tabular}[t]{@{~}l@{~}l@{\hspace{2em}}l@{~}l@{}}
a. & [OBL SUB OBJ V] & e. & [V OBL SUB OBJ]\\
b. & [OBL OBJ SUB V] & f. & [V OBL OBJ SUB]\\
c. & [SUB OBL OBJ V] & g. & [V SUB OBL OBJ]\\
d. & [OBJ OBL SUB V] & h. & [V OBJ OBL SUB]\\
\end{tabular}
\z

}

\subsubsubsection{Voranstellung}

\exewidth{(33))}

\frame{
\frametitle{Voranstellung}

\eal
\ex \blaubf{Er} fischt den Teich schnell leer. \hfill  (Subjekt)
\pause
\ex \blaubf{Den Teich} fischt er schnell leer. \hfill  (Objekt)
\pause
\ex \blaubf{Leer} fischt er den Teich nicht.   \hfill  (Resultativum = OBL)
\pause
\ex \blau{Schnell} fischt er den Teich leer. \hfill (Adjunkt)
\zl
\pause
Voranstellungen haben andere Eigenschaften als lokale Umstellungen.\\
Will man leere Elemente vermeiden,
sind weitere Konstruktionen nötig.\nocite{Haugereid2004a}


\pause
Berücksichtigt man Umstellungen (inkl.\ Fokusumstellung) braucht man:
\ea
\begin{tabular}[t]{@{~}l@{~}l@{\hspace{1em}}l@{~}l@{}}
a. & [V SUB OBL] (OBJ vorang.)   & d. & [V OBL SUB] (OBJ vorang.)\\
b. & [V OBJ  OBL] (SUB vorang.)  & e. & [V OBL OBJ]  (SUB vorang.)\\
c. & [V SUB OBJ] (OBL vorang.)   & f. & [V OBJ SUB] (OBL vorang.)\\
\end{tabular}
\z

% bochum
%\pause
%Lexikalische Analysen der Extraktion wie die von \citet*{BMS2001a} sind
%nicht mehr möglich, da Argumente auf der phrasalen Ebene eingeführt werden.

}

\subsubsubsection{Relativsätze und Interrogativsätze}

\frame{
\frametitle{Relativsätze und Interrogativsätze}


Voranstellung von Relativ- bzw. Interrogativphrasen ähnelt\\
der Voranstellung einer Wortgruppe vor das Finitum.

\eal
\ex der Mann, \blaubf{der} den Teich leer fischt
\ex den Teich, \blaubf{den} Richard leer fischt
\ex Er hat gefragt, \blaubf{wie platt} Max das Metall gehämmert hat.
\zl

Berücksichtigt man Umstellungen (inkl.\ Fokusumstellung) braucht man:
\ea
\small
\begin{tabular}[t]{@{~}l@{~}l@{\hspace{1em}}l@{~}l@{}}
a. & [SUB OBL V] (OBJ vorang.)  & d. & [OBL SUB V] (OBJ vorang.)\\
b. & [OBJ  OBL V] (SUB vorang.) & e. & [OBL  OBJ V] (SUB vorang.)\\
c. & [SUB OBJ V] (OBL vorang.)  & f. & [OBJ SUB V] (OBL vorang.)\\
\end{tabular}
\z

}

\subsubsubsection{Passiv und modale Infinitive}%, Medialkonstruktionen}

\frame{
\frametitle{Passiv und modale Infinitive}

Argument von \emph{leer} kann zum Subjekt der gesamten Konstruktion werden:

\eal
\ex weil er \blaubf{den Teich} leer fischt. \hfill (Aktiv)
\ex weil \blaubf{der Teich} leer gefischt wurde. \hfill (Passiv)
%\pause
%\ex weil der Teich leer gefischt ist. \hfill (Zustandspassiv)
\pause
\ex weil der Teich bis Montag leer zu fischen ist. \hfill (modaler Infinitiv)
\zl

%\mode<handout>{
%\eal
%\ex Stunden später sind meine Füße plattgelaufen,\footnote{taz, 02.01.1999, S.\,9}
%\ex Erinnern Sie sich an A Fish Called Wanda, 
%wo genußvoll ein Hündchen nach dem anderen plattgefahren wurde?\label{ex-huendchen-platt-fahren}\footnote{
%        taz-Bremen, 03.03.1990, S.\,27% Nr. 3048 Seite 27 vom 03.03.1990
%        }
%\zl
%}

\pause
Für Passiv braucht man:
\ea
\begin{tabular}[t]{@{~}l@{~}l@{\hspace{1em}}l@{~}l@{}}
a. & [ SUB OBL V]              & e. & [ V SUB OBL ] \\
b. & [ OBL SUB V] (Fokusumst.) & f. & [ V OBL SUB ] (Fokusu.)\\
c. & [ OBL V ] (SUB vorang.)   & g. & [ V OBL ] (SUB vorang.)\\
d. & [ SUB V ] (OBL vorang.)   & h. & [ V SUB ] (OBL vorang.)\\
\end{tabular}
\z
\pause
Zusätzlich weitere Konstruktionen für %Zustandspassiv, 
modale Infinitive und 
Medialkonstruktion.\nocite{Wunderlich97c,KR95a}


}


\subsubsubsection{usw.}

\frame{
\frametitle{usw.}

\begin{itemize}
\item Für bisher erwähnte Phänomene braucht man 50 Konstruktionen.
\pause
\item Freie Dative 
      \ea
      weil sie \blaubf{ihm} den Teich leer fischen.
      \z
\pause
      \begin{itemize}
      \item Umstellungen (2x3x4 x 2 = 48), 
\pause
      \item Voranstellungen (2x3x4 x 2 = 48), 
\pause
      \item Passiv + Passivvarianten (3x24 = 72)\\
            bisherige Konstruktionen + Dativpassiv:
            \ea
            weil \blaubf{er} den Teich leer gefischt bekommt.
            \z
\item Insgesamt 218 \Konstruktionen
      \end{itemize}
% Vorgangspassiv
% Permutation:   2x3 x 2 
% Voranstellung: 2x3 x 2 
% = 24
%
% Dativpassiv ist parallel, nur eben mit dem Dativargument
% = 24
%
% modaler Infinitiv:
% Permutation:   2x3 x 2 
% Voranstellung: 2x3 x 2 + 2 für PVP
%
% 
%
% Medialkonstruktion:
%
% 

\end{itemize}

}

\subsubsubsection{Offene Fragen: Syntax}

%\subsubsubsubsection{Adjunkte}

\frame{
\frametitle{Adjunkte}

\savespace
\begin{itemize}
\item Adjunkte können im Deutschen irgendwo im Mittelfeld stehen:
\ea
daß \visible<5->{(}\visible<2,5->{\blau<2>{schnell}}\visible<5->{)} jemand \visible<5->{(}\visible<3,5->{\blau<3>{schnell}}\visible<5->{)} den Teich \visible<5->{(}\visible<4->{\blau<4>{schnell}}\visible<5->{)} leer fischt.
\z
\pause\pause
\pause\pause
\item Adjunktanalysen in der Konstruktionsgrammatik:
      \begin{itemize}
      \item \citew{KF99a} \pause{}formal falsch (Mengenunifikation, \citew{Mueller2006d})
\pause
      \item \citew{Kay2005a}\pause{}, wenn man es repariert, entspricht es \citew{NB94}\\
            = lexikalische Einführung von Adjunkten\\
\pause
            $\to$ Skopusproblem, da Resultativbedeutung nicht im Lexikon eingeführt wird
      \end{itemize}
\pause
\item Es bleibt nur eine phrasale Adjunkt-Resultativkonstruktion anzunehmen.\\
\pause
      Reguläre Ausdrücke:\\
      {}[\blau{Adjunct*} SUBJ \blau{Adjunct*} OBJ \blau{Adjunct*} OBL \blau{Adjunct*} V]

\pause
      Und die Semantik? Relationale Beschränkungen a la \citew{Kasper94a}?
\end{itemize}

}

%\subsubsubsubsection{Prädikatskomplexe}

\frame{
\frametitle{Umstellungen im Zusammenhang mit Prädikatskomplexen}

\savespace
\begin{itemize}
\item Soll man für (\mex{1}) eine AcI-Resultativkonstruktion annehmen?
\ea
weil    ihn       den Teich      \blaubf{niemand}      leer  fischen \blaubf{sah}.
\z
\pause
(Und die Semantik?)
\pause
\item Annahme von diskontinuierlichen Konstruktionen \citep{Reape94a} führt zu Problemen
      bei Subjekt-Verb-Kongruenz und Fernpassiv \citep{Kathol98b}.

\bigskip
\pause
\item Schlußfolgerung:\\
      Man braucht noch mehr Grundmuster und entsprechende Permutationen.
\end{itemize}

}

\subsubsubsection{Automatische Hüllenberechnung}

\frame{
\frametitle{Automatische Hüllenberechnung}

\vfill
\oneline{%
\hfill\begin{tabular}{cccc}
\multicolumn{4}{c}{\node{c}{\it construction}}\\[6ex]
\node{a}{\it active} & \node{p}{\it passive} & \node{m}{middle} & \node{r}{resultative}\\[6ex]
&\visible<2>{\blau<2>{\node{ar}{\it active resultative}}}   & \visible<2>{\blau<2>{\node{pr}{\it passive resultative}}}  & \visible<2>{\blau<2>{\node{mr}{middle resultative}}}\\[6ex]
\end{tabular}
\nodeconnect{c}{a}\nodeconnect{c}{p}\nodeconnect{c}{m}\nodeconnect{c}{r}%
\visible<2>{\blau<2>{%
\mode<handout>{
\dashlength 4pt%
}
\nodeconnect{a}{ar}\nodeconnect{p}{pr}\nodeconnect{m}{mr}%
\nodeconnect{r}{ar}\nodeconnect{r}{pr}\nodeconnect{r}{mr}%
}}}
\hfill\hfill\mbox{}
\vfill
Idee: Nur die Kernkonstruktionen werden spezifiziert\pause{},\\
die Interaktionen werden automatisch berechnet.
\vfill

}

%\subsubsubsubsection{Kays Vorschlag}

\frame[shrink=4]{

\frametitle{Automatische Hüllenberechnung: Kays Vorschlag}


Unlike LFG phrase structure rules and lexical items and unlike HPSG maximal
types, distinct maximal constructions can span the same (piece of) FT [\emph{Feature Structure Tree}, St.\ Mü.]. 
For example, the English VP construction, which provides for a lexical verb followed
by an arbitrary number of constituents (subject to valence restrictions), can unify
with a construction specifically licensing a VP displaying the `heavy NP shift'
property. In order to specify an explicit recursive licensing procedure for sentences,
we need some way to deal with this overlap of constructions. 
We wish to reduce the set of constructions of a grammar to a set of construction"=like
objects (let's call them CLOs) with the property that in licencing a given sentence,
exactly one CLO licences each node. To obtain the set of CLOs from the set of constructions
$C$: \blau{(1) form the power set of the set of constructions $\wp(C)$; (2) for each set of
constructions in  $\wp(C)$, attempt to unify all the members, matching the root nodes;
(3) throw away all the sets that don't unify; (4) the remainder is the set of CLOs}.
%\citep[Subsubsection~7.1]{Kay2002a}%
\citep{Kay2002a}

~

}

\frame{

\frametitle{Ergebnis der Anwendung des Algorithmus}

\eal
\ex C = \{ VP construction, Heavy NP Shift construction \}
\ex $\wp(C)$ = \{ \begin{tabular}[t]{@{}l}
                 \{\},\\
                 \{ VP construction \},\\
                 \{ Heavy NP Shift construction \},\\
                 \{ VP construction, Heavy NP Shift construction \} \}\\
                 \end{tabular}
\pause
\ex Erwünscht:\\
    CLOs = \{ VP construction $\wedge$ Heavy NP Shift construction \}
\pause
\ex Ergebnis nach Kays Algorithmus:\\
    CLOs = \{ \begin{tabular}[t]{@{}l}
              \rot{VP construction},\\
              \rot{Heavy NP Shift construction},\\
              VP construction $\wedge$ Heavy NP Shift construction \}\\
                 \end{tabular}
\zl

}

%\subsubsubsubsection{Automatische Hüllenberechnung mit Subsumptionstest}

\frame{
\frametitle{Reparatur}

\begin{itemize}
\item Man müßte einen Subsumptionstest einbauen,\\
      \dash alle die CLOs aus der Menge entfernen,\\
      die allgemeiner sind als ein anderes CLO.

\pause
\bigskip
\item Es entstehen Probleme mit Idiomen,\\
      da diese immer spezieller sind als nicht-idiomatische  \Konstruktionen.
%% \item Man müßte die Eigenschaft, daß \emph{kick the bucket} eine VP ist, also bewußt aus
%%       der \Konstruktion ausklammern, damit sie dann bei der Hüllenberechnung nicht zu Problemen führt.
\pause
\item Alle \Konstruktionen, die idiomatische Unterkonstruktionen haben, müssen zusätzliche
      nicht-idiomatische Unterkonstruktionen kriegen,\\
      da sonst die Unifikationsergebnisse der Idiom-Konstruktionen mit den allgemeinen \Konstruktionen zur Eliminierung der allgemeinen
      \Konstruktionen führen würde.

\pause
      \emph{kick the bucket} würde sonst dafür sorgen,\\
      daß es keine reguläre transitive VP-Konstruktion gibt.
\end{itemize}

}

%\subsubsubsubsection{Idioms und Hüllenberechnung mit Subsumptionstest}

\frame{

\frametitle{Idioms und Hüllenberechnung}

\smallexamples
\eal
\ex C = \{ VP, Transitive, \emph{kick the bucket}  \}
\ex $\wp(C)$ = \{ \begin{tabular}[t]{@{}l}
                 \{\},\\
                 \{ VP \},\\
                 \{ Transitive \},\\
                 \{ \emph{kick the bucket}  \},\\
                 \{ VP, Transitive  \},\\
                 \{ VP, \emph{kick the bucket}  \},\\
                 \{ Transitive, \emph{kick the bucket}  \},\\
                 \{ VP, Transitive, \emph{kick the bucket}  \} \}\\
                 \end{tabular}
\pause
\ex Ergebnis bei Berücksichtigung von Subsumptionsverhältnissen:\\
    CLOs = \{ \begin{tabular}[t]{@{}l}
              VP $\wedge$ Transitive $\wedge$ \emph{kick the bucket} \}\\
                 \end{tabular}
\pause
\ex Erwünscht:\\
    CLOs = \{ \begin{tabular}[t]{@{}l}
              VP $\wedge$ Transitive,\\
              VP $\wedge$ Transitive $\wedge$ \emph{kick the bucket} \}\\
                 \end{tabular}
\zl

}

%\subsubsubsubsection{Idioms und Hüllenberechnung mit Hilfskonstruktionen}

\frame[shrink=30]{

\frametitle{Idioms und Hüllenberechnung mit Hilfskonstruktionen}


\eal
\ex C = \{ VP, Transitive \blau<1>{Idiomatic}, Transitive \blau<1>{Non-Idiomatic}, \emph{kick the bucket}  \}
\pause
\ex $\wp(C)$ = \{ \begin{tabular}[t]{@{}l@{}l@{}}
                 \{\},\\[1ex]
%
                \blau<4-5>{\{ VP \}},                                            & \visible<beamer| 3-5>{\rot{Unifikationsfehler}}\\
                \blau<4-5>{\{ Transitive Idiomatic  \}},                         & \visible<beamer| 4-5>{\blau{Subsumptionstest}}\\
                \blau<4-5>{\{ Transitive Non-Idiomatic  \}},                     & \visible<beamer| 5-6>{\gruen{Rest}}\\
                \blau<4-5>{\{ \emph{kick the bucket}    \}},\\[1ex]
%
                \blau<4-5>{\{ VP, Transitive Idiomatic  \}},\\
                \gruen<5-6>{\{ VP, Transitive Non-Idiomatic  \}},\\
                \blau<4-5>{\{ VP, \emph{kick the bucket}  \}},\\
                 \{ \rot<3-5>{Transitive Idiomatic}, \rot<3-5>{Transitive Non-Idiomatic}  \},\\
                \blau<4-5>{\{ Transitive Idiomatic, \emph{kick the bucket}  \}},\\
                 \{ \rot<3-5>{Transitive Non-Idiomatic}, \rot<3-5>{\emph{kick the bucket}}  \},\\[1ex]
%
                 {\{ VP, \rot<3-5>{Transitive Idiomatic}, \rot<3-5>{Transitive Non-Idiomatic}  \},}\\
                 {\gruen<5-6>{\{ VP, Transitive Idiomatic, \emph{kick the bucket}  \}},}\\
                 {\{ VP, \rot<3-5>{Transitive Non-Idiomatic}, \rot<3-5>{\emph{kick the bucket}}  \},}\\
                 {\{ \rot<3-5>{Transitive Idiomatic}, \rot<3-5>{Transitive Non-Idiomatic}, \emph{kick the bucket}  \},}\\[1ex]
%
                 \multicolumn{2}{@{}l@{}}{\{ VP, \rot<3-5>{Transitive Idiomatic}, \rot<3-5>{Transitive Non-Idiomatic}, \emph{kick the bucket}  \} \}}\\

                 \end{tabular}
\pause\pause\pause\pause
\ex CLOs = \{ \begin{tabular}[t]{@{}l}
              VP $\wedge$ Transitive Non-Idiomatic,\\
              VP $\wedge$ Transitive Idiomatic $\wedge$ \emph{kick the bucket} \}\\
                 \end{tabular}
\zl

}

\frame{

\frametitle{Fragen}

\begin{itemize}[<+->]
\item Zu welchen \Konstruktionen brauchen wir\\
      die Idiomatic/Non-Idiomatic Unterkonstruktionen?\\
      Zu allen \Konstruktionen,\\
      die mit idiomatischen \Konstruktionen kompatibel sind.
\item Das könnte man automatisch berechnen,\\
      vorausgesetzt, die idiomatischen \Konstruktionen sind gekennzeichnet.
\item Welchen theoretischen Status haben die neuen Konstruktionen?\\
      Bisher wurden nur CLOs berechnet,\\
      aber jetzt berechnen wir auch \Konstruktionen!
\end{itemize}

}


\subsubsubsection{Offene Fragen: Morphologie}

\frame{
\frametitle{Morphologie}
%\footnotesize
\savespace

%\eal
%\ex \ungn:\\ 
%\emph{Leerfischung}\footnote{
%        taz, 20.06.1996, S.\,6.%
%},
%\emph{Kaputterschließung}\footnote{
%        taz, 02.09.1987, S.\,8.%  % 02.09.1987, S.\,8     
%      },
%\emph{Kaputtmilitarisierung}\footnote{
%        taz, 19.04.1990, S.\,5.% %19.04.1990, S.\,5
%      },
%\emph{Gelbfärbung}\footnote{
%taz, 14.08.1995, S.\,3.% %TAZ Nr. 4695 Seite 3 vom 14.08.1995
%}
%%
%\ex \emph{-er} nominalizations:\\
%\emph{Totschläger}\footnote{
%        taz, bremen, 24.05.1996, S.\,24 und taz, hamburg 21.07.1999, S.\,22%
%},
%\emph{SFB-Gesundbeter}\footnote{
%        taz, 25.08.1989, S.\,20.% %TAZ Nr. 2893 Seite 20 vom 25.08.1989
%        },
%\emph{Ex"=Bierflaschenleertrinker}\footnote{
%        taz, 13./14.01.2001, S.\,32.%
%}
%%
%\ex  marginal auch \geen:\\
%\emph{Totgeschlage}\footnote{
%  \citew[S.\,208]{FB95a}.
%}
%\zl

\eal
\ex \ungn:\\ 
\emph{Leerfischung} (taz, 20.06.96),
\emph{Kaputterschließung} (taz, 02.09.87),
\emph{Kaputtsanierung}, (FR, 24.10.98),\\
\emph{Kaputtmilitarisierung} (taz, 19.04.90),
\emph{Gelbfärbung} (MM, 27.05.88)
%
\pause
\ex \ern:\\
\emph{Totschläger} (FR, 08.01.98 und ZEIT, 03.10.86)\\ % (taz, bremen, 24.05.96 und taz, hamburg 21.07.99)
\emph{SFB-Gesundbeter} (taz, 25.08.89),\\
\emph{Ex"=Bierflaschenleertrinker} (taz, 13.01.01)
%
\pause
\ex  marginal auch \geen:\\
\emph{Totgeschlage} \citep[S.\,208]{FB95a}
\zl

\pause
Eine übergeordnete Konstruktion,\\
von der die phrasalen und die morphologischen Konstruktionen erben?


}

\frame{
\frametitle{Derivationelle Morphologie mit einfacher Vererbung?}

\savespace
\eal
\ex allgemeine Resultativkonstruktion:\\
    \begin{tabular}[t]{|l|}\hline
    syn val \{ NP$_{\#1}$, NP$_{\#2}$, Pred$_{\#3}$, V$_{\#4}$ \} \\
    sem cause-become( \#1, \#2, \#3 ) by \#4\\\hline
    \end{tabular}
%\vspace{\itemsep}

\pause
~
\ex Nominalisierungsresultativkonstruktion:\\
    \begin{tabular}[t]{|l|}\hline
    syn val \{ Det, NP$_{\#2}$, Pred$_{\#3}$, V$_{\#4}$ \} \\
    sem \rot<3>{nominal-semantics(}cause-become( \#1, \#2, \#3 ) by \#4\rot<3>{)}\\\hline
    \end{tabular}

~
\zl

\pause

Mit normaler Vererbung kann ein Objekt nicht sowohl vom Typ (\mex{0}a)
als auch vom Typ (\mex{0}b) sein, da die {\sc sem}-Werte verschieden sind.

Das ginge nur mit zusätzlichen Merkmalen und "`Umkopieren"'. \nocite{Koenig99a,MuellerLehrbuch3}


}


\frame{

\frametitle{Derivationelle Morphologie und Vererbung}


%\begin{itemize}
%\item 
Derivationelle Morphologie nicht über Vererbungshierarchien modellierbar:
\begin{itemize}
\item Rekursion ist nicht erfaßbar \citep{KN93a}:
\ea
Vorvorvorvorvorversion\footnote{
\url{http://forum.geizhals.at/t393036,3147329.html}. 10.07.2007.
}
\z
\pause
\item Vererbung ist nicht asymmetrisch:
\eal
\ex {}[un- [do -able]]  \jambox{(nicht machbar)}
\ex {}[[un- do] -able]  \jambox{(kann rückgängig gemacht werden)} 
\zl
\end{itemize}

%Siehe auch \citew{Riehemann:93}.
%


}

\frame{
\frametitle{Derivationelle Morphologie mit Einbettung}

\ea
einbettende \emph{-ung}-\emph{Nominalisierungskonstruktion}:\\
\begin{tabular}{@{}l@{\hspace{5ex}}l@{}}
\begin{tabular}[t]{|l|}\hline
syn N\\
sem nominal-semantics(\#1)\\
phon \#2 $\oplus$ \phonliste{ ung }\\\\
\hspace{6em}%\colorbox{blue!30}{
\node{da}{\begin{tabular}{|l|}\hline
syn V\\
sem \#1\\
phon \#2\\\hline
\end{tabular}}%}
\\[-1ex]
\\\hline
\end{tabular} & \visible<3->{\begin{tabular}[t]{@{}l@{}}
\\\\
                Resultativkonstruktion\\
                muss \node{hier}{hier} rein\\
                \end{tabular}
\anodeconnect{hier}[tr]{da}}%
\end{tabular}
\z

\pause
Bei der Vererbung sagt man etwas über die äußere Box,\\
nicht über die innere.

}

\begin{vortrag}

\frame{
\frametitle{Und Defaults?}

\savespace
\begin{itemize}
\item \citet{MR2001a}: \emph{be-}Verben über Default-Vererbung
\pause
\item Die Details der Formalisierung bleiben im Dunkeln.
\pause
\item Man kann mit Defaults Listen verlängern. \citep{Villavicencio2000a-u}\\
\pause
      $\to$ Mit Default-Zeigern und Pfadungleichungen ist es möglich,
      \begin{itemize}
      \item eine Liste phonologischer Information zu verlängern,
\pause
      \item die syntaktische Kategorie zu verändern,
\pause
      \item semantische Einbettung zu modellieren,\\
            wenn die Repräsentation listenbasiert ist, wie \zb bei MRS.\nocite{CFPS2005a}
      \end{itemize}
\pause
\item \emph{undoable} und \emph{Vorvorversion} kann analysiert werden, wenn man
\pause
      \begin{itemize}
      \item diverse Hilfsmerkmale verwendet (für Phonologie, syn.\ Kategorie, Semantik)
\pause
      \item Relationale-Beschränkungen hat, die aus der Liste der phonologischen
            Repräsentationen den \phonw ausrechnet.
\pause
      \item unendlich viele Affixe bzw.\ reguläre Ausdrücke in Merkmalstrukturen hat.
      \end{itemize}
\pause
\item Das ist unglaublich häßlich! Details siehe \citew{Mueller2005g}.
\end{itemize}

%\begin{bochum}
%\pause
%\item Manche Sachen kann man mit Erweiterungen des Formalismus\\
%um reguläre Ausdrücke mit Hilfe von Default-Vererbung\\
%auf eine sehr, sehr technische Weise machen \citep{Mueller2005g}.
%\end{bochum}
%\end{itemize}

}

\end{vortrag}


\subsubsection{Eine lexikalische Analyse}

\outline{

\begin{itemize}
\item Resultativkonstruktionen
\item Phrasale Analysen
\item Interaktion mit anderen Phänomenen
\item \blau{Eine lexikalische Analyse}
\item Zusammenfassung
\end{itemize}

}




%\frame{
%\frametitle{Eine lexikalische Analyse}

%Beispiel: Analyse im Rahmen der HPSG
%      \begin{itemize}
%      \item Argumente eines Kopfes werden in einer Liste repräsentiert.
%\pause
%      \item Lexikoneinträge können zu anderen über Lexikonregeln in Beziehung gesetzt werden.
%\pause
%      \item Prädikatskomplexe können über Argumentkomposition analysiert werden.
%      \end{itemize}


%}

\subsubsubsection{Einige Eigenschaften der Resultativkonstruktion}

\frame{
\frametitle{RKen im Vergleich zur restlichen deutschen Syntax}


RK gleichen Kopulakonstruktionen, \emph{gut finden}-Prädikationen, Verbalkomplexen,
und Partikelverben in folgender Hinsicht:
\begin{itemize}[<+->]
\item Permutation von Argumenten beteiligter Köpfe
\item Voranstellung unvollständiger Teilphrasen,\\
      die einen linken Präfix des Prädikatskomplexes bilden,\\
      möglicherweise mit Argumenten und Adjunkten
\item Keine Voranstellung aus der Mitte des Prädikatskomplexes
\item Skopus von Adjunkten über alle Bestandteile des Komplexes möglich
\end{itemize}

}


\subsubsubsection{Grundannahmen}

\frame{

\frametitle{Grundannahmen: Konstituentenstellung und Valenz}

\vfill
\hfill%
%\resizebox{!}{0.8\textheight}{
%\small
\psset{xunit=1cm,yunit=5.4mm}%
%
% node labels for moving elements will be typeset by the \tmove command
% here we have to provide invisible boxes to get the line drawing right.
\begin{pspicture}(4.8,1)(14.4,7.6)

%\rput[B](1,1){\rnode{speccp}{\visible<1->{diesen Mann$_i$}}}
\rput[B](5,1){\rnode{jeder}{niemand}}
\rput[B](7,1){\rnode{ihnmf}{ihn}}
\rput[B](9,1){\rnode{kennt}{kennt}}

\rput[B](7,3){\rnode{np1}{NP[\textit{acc}]}}
\rput[B](9,3){\rnode{v}{V}}\nput[labelsep=2pt]{0}{v}{\only<2->{\nliste{NP[\textit{nom}], NP[\textit{acc}]}}}

\rput[B](5,5){\rnode{np2}{NP[\textit{nom}]}}
\rput[B](8,5){\rnode{vs1}{V'}}\nput[labelsep=2pt]{0}{vs1}{\only<3->{\nliste{NP[\textit{nom}]}}}

\rput[B](6.5,7){\rnode{vp}{VP}}\nput[labelsep=2pt]{0}{vp}{\only<3->{\nliste{ }}}



\psset{angleA=-90,angleB=90,arm=0pt}

\ncdiag{v}{kennt}
\ncdiag{vs1}{np1}\ncdiag{vs1}{v}
\ncdiag{vs2}{np2}\ncdiag{vs2}{vs1}
\ncdiag{vp}{vs2}

%\ncdiag{np3}{t1}

\ncdiag{i}{t2}
\ncdiag{is}{i}\ncdiag{is}{vp}
\ncdiag{vp}{np2}\ncdiag{ip}{is}

\ncdiag{np2}{jeder}
\ncdiag{np1}{ihnmf}
\ncdiag{vp}{vs1}

%\psgrid

\end{pspicture}%
\hfill\hfill\mbox{}
\vfill
\pause
\begin{itemize}
\item Valenzanforderung ist in einer Liste repräsentiert
\pause
\item Ein beliebiges Element der Liste kann mit Kopf kombiniert werden.\\
      $\to$ auch Abfolge Acc $<$ Nom analysierbar.\\
      Liste mit restlichen Elementen wird nach oben gegeben.
\end{itemize}

}


%% \subsubsubsubsection{Der Verbalkomplex}


%% \frame[t]{
%% \frametitle{Der Verbalkomplex}

%% \hfill\scalebox{0.85}{%
%% \psset{xunit=1cm,yunit=5.4mm}%
%% %
%% % node labels for moving elements will be typeset by the \tmove command
%% % here we have to provide invisible boxes to get the line drawing right.
%% \begin{pspicture}(2.25,1)(15.8,9.6)
%% %\psgrid

%% \rput[B](3,1){\rnode{er}{niemand}}
%% \rput[B](5,1){\rnode{ihn}{ihn}}
%% \rput[B](7,1){\rnode{zu reparieren}{zu reparieren}}
%% \rput[B](11.5,1){\rnode{versucht}{versucht}}

%% %% \rput[B](7,0){leer}
%% %% \rput[B](11.5,0){fischt}

%% %% \rput[B](7,-1){an}
%% %% \rput[B](11.5,-1){lacht}

%% \rput[B](7,3){\rnode{vzu reparieren}{\blau<3>{V}}}\nput[labelsep=2pt]{0}{vzu reparieren}{\only<2->{\rnode{sczu reparieren}{\nliste{ NP[\textit{nom}], \blau<2>{NP[\textit{acc}] }}}}}
%% \rput[B](11.5,3){\rnode{vversucht}{V}}\nput[labelsep=2pt]{0}{vversucht}{\only<2->{\nliste{ NP[\textit{nom}],\blau<2>{ NP[\textit{acc}]}, \blau<3>{V} }}}


%% \rput[B](5,5){\rnode{np2}{NP[\textit{acc}]}}
%% \rput[B](8,5){\rnode{vzu reparierenversucht}{V}}\nput[labelsep=2pt]{0}{vzu reparierenversucht}{\only<4->{\nliste{ NP[\textit{nom}], NP[\textit{acc}] }}}

%% \rput[B](6.5,7){\rnode{vs}{V'}}\nput[labelsep=2pt]{0}{vs}{\only<5->{\nliste{ NP[\textit{nom}] }}}



%% \rput[B](3,7){\rnode{np1}{NP[\textit{nom}]}}

%% \rput[B](4.75,9){\rnode{vp}{VP}}\nput[labelsep=2pt]{0}{vp}{\only<5->{\nliste{  }}}




%% \psset{angleA=-90,angleB=90,arm=0pt}

%% \ncdiag{vversucht}{versucht}
%% \ncdiag{vzu reparieren}{zu reparieren}
%% \ncdiag{vzu reparierenversucht}{vversucht}\ncdiag{vzu reparierenversucht}{vzu reparieren}
%% \ncdiag{vs}{np2}\ncdiag{vs}{vzu reparierenversucht}
%% \ncdiag{vp}{vs}\ncdiag{vp}{np1}

%% \ncdiag{np2}{ihn}
%% \ncdiag{np1}{er}




%% \end{pspicture}
%% }
%% \hfill\hfill\mbox{}%

%% \begin{itemize}
%% \item Verben bilden einen Komplex
%% \pause
%% \item Argumente des eingebetteten Verbs werden angezogen\\\citep{HN89a,HN94a,Kiss95a}
%% \pause
%% \pause
%% \item Argumente in Liste werden nacheinander abgearbeitet
%% %\pause
%% %\item Kasus ist im Lexikon unterspezifiziert (Kasusprinzipien)
%% \end{itemize}

%% }

\subsubsubsection{Resultativkonstruktionen}




\frame[t]{
\frametitle{Resultativkonstruktionen}

\savespace

\hfill\scalebox{0.85}{%
\psset{xunit=1cm,yunit=5.4mm}%
%
% node labels for moving elements will be typeset by the \tmove command
% here we have to provide invisible boxes to get the line drawing right.
\begin{pspicture}(2.25,1)(15.8,9.6)
%\psgrid

\rput[B](3,1){\rnode{er}{niemand}}
\rput[B](5,1){\rnode{ihn}{ihn}}
%\rput[B](7,1){\rnode{zu reparieren}{zu reparieren}}
%\rput[B](11.5,1){\rnode{versucht}{versucht}}

\rput[B](7,1){\rnode{zu reparieren}{leer}}
\rput[B](11.5,1){\rnode{versucht}{fischt}}

%% \rput[B](7,-1){an}
%% \rput[B](11.5,-1){lacht}

\rput[B](7,3){\rnode{vzu reparieren}{Adj}}\nput[labelsep=2pt]{0}{vzu reparieren}{\only<1->{\rnode{sczu reparieren}{\nliste{ \blau<2>{NP[\textit{acc}] }}}}}
\rput[B](11.5,3){\rnode{vversucht}{\blau<1>{V}}}\nput[labelsep=2pt]{0}{vversucht}{\only<1->{\blau<1>{\nliste{ NP[\textit{nom}],\blau<2>{ NP[\textit{acc}]}, Adj }}}}


\rput[B](5,5){\rnode{np2}{NP[\textit{acc}]}}
\rput[B](8,5){\rnode{vzu reparierenversucht}{V}}\nput[labelsep=2pt]{0}{vzu reparierenversucht}{\only<1->{\nliste{ NP[\textit{nom}], NP[\textit{acc}] }}}

\rput[B](6.5,7){\rnode{vs}{V'}}\nput[labelsep=2pt]{0}{vs}{\only<1->{\nliste{ NP[\textit{nom}] }}}



\rput[B](3,7){\rnode{np1}{NP[\textit{nom}]}}

\rput[B](4.75,9){\rnode{vp}{VP}}\nput[labelsep=2pt]{0}{vp}{\only<1->{\nliste{  }}}




\psset{angleA=-90,angleB=90,arm=0pt}

\ncdiag{vversucht}{versucht}
\ncdiag{vzu reparieren}{zu reparieren}
\ncdiag{vzu reparierenversucht}{vversucht}\ncdiag{vzu reparierenversucht}{vzu reparieren}
\ncdiag{vs}{np2}\ncdiag{vs}{vzu reparierenversucht}
\ncdiag{vp}{vs}\ncdiag{vp}{np1}

\ncdiag{np2}{ihn}
\ncdiag{np1}{er}




\end{pspicture}
}
\hfill\hfill\mbox{}%

\begin{itemize}
\item Lexikoneintrag für Verb in RK über Lexikonregel lizenziert\\
      Diese Lexikonregel steuert auch die Kausativbedeutung bei.
\pause
\item Subjekt des eingebetteten Adj wird zum Objekt des Verbs
\pause
\item Analyse der Resultativkonstruktion ist parallel zu der von Verbalkomplexen \citep{HN89a,HN94a,Kiss95a}
\end{itemize}


}



\mode<beamer>{\beamertemplatebackfindforwardnavigationsymbolshorizontal}

\subsubsection{Zusammenfassung}

\frame[label=lastframe]{
\frametitle{Zusammenfassung: Vorteile der lexikalischen Analyse}

%\savespace
~\vspace{-\baselineskip}
\begin{itemize}
\item Passiv, modale Infinitive und Medialkonstruktionen sind von der RK unabhängig.
%
\pause
\item Konstituentenstellung ist von der RK unabhängig.
%
\pause
\item Adjunktsyntax, freie Dative und Interaktionen sind von der RK unabhängig.
%
\pause
\item Parallelität zw.\ Verbalkomplexbildung \citep{HN89a},
Partikelverbsyntax \citep{Hoehle82b,Mueller2003a} und RK %\citep{Mueller2002b}
ist erfaßt.
%
\pause
\vspace{1ex}
%\bigskip
\item Morphologische Prozesse nehmen Lexikoneinträge von Verben für RK als Eingabe.
%
\pause
\vspace{1ex}
%\bigskip
\item Sprachübergreifende Gemeinsamkeiten werden erfaßt.\\
      Unterschiede folgen aus Unterschieden in der jeweiligen Syntax.

%% \pause
%% \vspace{1ex}
%% \item Aufsatz unter \url{http://www.cl.uni-bremen.de/~stefan/Pub/phrasal.html}
\end{itemize}



}

\subsection{Grammatiktheoretische Einordnung}

\frame{
\frametitle{Grammatiktheoretische Einordnung}

\begin{itemize}
\item Mittels Typhierarchien kann man phrasale Muster kategorisieren.
\item Das ist jedoch nicht genug.\\
      Man braucht Regeln, die komplexe Einheiten kombinieren.
\item Nicht alle argumentstrukturverändernden Prozesse lassen sich über Vererbung erfassen.
\item Wenn man keine Transformationen annehmen will und von lexikalischer Integrität ausgeht, muss man alle Phänomene, die mit
derivationeller Morphologie und argumentstrukturverändernden Prozessen interagieren, lexikalisch
behandeln. (\citealp[S.\,412]{Dowty78a}; \citealp[S.\,21]{Bresnan82a}; \citealp{Mueller2006d,Mueller2007d})
\item Die skizzierte lexikalische Analyse ist mit GPSG, LFG, HPSG und auch CxG kompatibel.\\
(siehe \citew{Simpson83a} für eine lexikalische Analyse in LFG.)
\end{itemize}





}
