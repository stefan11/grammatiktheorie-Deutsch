\section{Kategorialgrammatik (CG)}

\outline{

\begin{itemize}
\item Begriffe
\item Phrasenstrukturgrammatiken
\item Government \& Binding (GB)
\item Generalisierte Phrasenstrukturgrammatik (GPSG)
\item Lexikalisch-Funktionale Grammatik (LFG)
%\item Lexical Mapping Theory (LMT)
%\item PATR
\item \blaubf{Kategorialgrammatik (CG)}
\item Kopfgesteuerte Phrasenstrukturgrammatik (HPSG)
\item Konstruktionsgrammatik (CxG)
\item Baumadjunktionsgrammatik (TAG)
\end{itemize}
}



\frame{
\frametitle{Kategorialgrammatik}

\begin{itemize}
\item Kategorialgrammatik ist die älteste der hier besprochenen Theorien \citep{Ajdukiewicz35a-u}.
\item Heute hauptsächlich in Edinburgh, Uetrecht und Amsterdam betrieben.
\item Wichtige Artikel und Bücher von \citet{Steedman91a,Steedman2000a-u,SB2006a-u}
\end{itemize}

}

\subsection{Allgemeines zum Repräsentationsformat}


\frame{
\frametitle{Motivation der Kategorialgrammatik}


\begin{itemize}
\item komplexe Kategorien ersetzen das \subcat"=Merkmal der GPSG

\medskip
\begin{tabular}[t]{@{}l@{\hspace{1cm}}l}
Regel                              & Kategorie~im~Lexikon\\
vp $\to$ v(ditrans) np~np          & (vp/np)/np  \\
vp $\to$ v(np\_and\_pp) np~pp(to)  & (vp/pp)/np  \\
\end{tabular}
\medskip


%
\pause
\item Es gibt nur noch sehr wenige, sehr abstrakte Regeln:

\ea
\blaubf{Multiplikationsregel}\\
\emph{Regel1}: X/Y * Y = X 
\z

Diese Regel sagt: Kombiniere ein X, das ein Y sucht, mit dem Y,\\
wenn es rechts von X/Y steht.
\pause

\item Valenz ist nur noch einmal kodiert, nämlich im Lexikon.

Bisher Valenz in den Grammatikregeln und im \subcat-Merkmal des Lexikoneintrags.
\end{itemize}
}

\frame{
\frametitle{Ein Beispiel}


\ea
\blaubf{Multiplikationsregel}\\
\emph{Regel1}: X/Y * Y = X 
\z

Diese Regel sagt: Kombiniere ein X, das ein Y sucht, mit dem Y,\\
wenn es rechts von X/Y steht.

%% \begin{center}
%%   \deriv{3}{
%%      \ccggf{I}  & \ccggf{saw} & \ccggf{the~man} \\
%%      \uline{1}  & \uline{1} & \uline{1}\\
%%      \ccgcf{NP} & \ccgcf{(S\bs NP)/NP} & \ccgcf{NP}\\
%%       &  \fapply{2}  \\
%%        & \gccmc{2}{\ccgcf{S\bs NP}} \\
%%        \bapply{3}  \\
%%       \cgmc{3}{\ccgcf{S}}
%%      }
%% \end{center}

\deriv{2}{
%\begin{tabular}{@{}cc@{}}
chased       & Mary\\
%\uline{1}    & \uline{1} \\
\hr & \hr\\
vp/np   & np\\
\multicolumn{2}{@{}c@{}}{\visible<2->{\forwardapp}} \\
\multicolumn{2}{@{}c@{}}{\visible<2->{vp}}\\
%\cgmc<2->{2}{vp}\\
%\end{tabular}
}


\pause
%\item für `/' meist Linksassoziativität \dh (vp/pp)/np = vp/pp/np

\pause

Kategorie v wird nicht mehr benötigt.

}

\frame{
\frametitle{Kategorialgrammatik}

\begin{itemize}
\item vp kann auch eleminiert werden: vp = s$\backslash$np

\ea
\emph{Regel2}: Y * X$\backslash$Y = X 
\z

\pause
\deriv{4}{
the  & cat & chased         & Mary\\
\hr  & \hr & \hr            & \hr\\
np/n & n   & (s\bs np)/np   & np\\
\multicolumn{2}{@{}c}{\visible<3->{\forwardapp}} & \multicolumn{2}{c@{}}{\visible<4->{\forwardapp}}\\
\multicolumn{2}{c}{\visible<3->{np}}             & \multicolumn{2}{c@{}}{\visible<4->{s\bs np}}\\
\multicolumn{4}{@{}c@{}}{\visible<5->{\backwardapp}}\\
\multicolumn{4}{c@{}}{\visible<5->{s}}\\
}

\pause\pause\pause
\pause
\item kein expliziter Unterschied zwischen Phrasen und Wörtern:
\begin{itemize}
\item intransitives Verb = Verbphrase = $(s \backslash np)$
\item genauso Eigennamen = Nominalphrasen = np
\end{itemize}
\end{itemize}
}


\frame{

\frametitle{Modifikation}

\begin{itemize}
\item optionale Modifikatoren:

vp $\to$ vp~pp \\
noun $\to$ noun~pp

beliebig viele PPen nach einer VP bzw.\ einem Nomen
\pause
\item Modifikatoren allgemein haben Form: $X \backslash X$ bzw.\ $X / X$
\pause
\item Prämodifikator für Nomina:

noun $\to$ adj~noun\\

Adjektive = $n/n$
\pause
\item Postmodifier für Nomina: $n \backslash n$
\pause
\item vp-Modifikator $\to$  X = $s \backslash np$
\pause
\item vp-Modifikator: $(s \backslash np) \backslash (s \backslash np)$. 

\end{itemize}
}

\frame{
\frametitle{Ableitung mit einer Kategorialgrammatik}
\vfill
% Eine typische Kategorialgrammatik benutzt nur s, np und n als Hauptkategorien. Manchmal werden noch pps als
% Komplementpräpositionalphrasen zugelassen.

% Eine Ableitung in der CG ist im wesentlichen ein binär verzweigender Baum, wird aber meistens wie folgt repräsentiert:
% Ein Pfeil unter einem Paar von Kategorien zeigt an, daß diese mit einer Kombinationsregel kombiniert werden.
% Die Richtung des Pfeils gibt die Richtung der Kombination an. Das Ergebnis wird unter den Pfeil geschrieben.
% Ein Beispiel zeigt Abbildung \ref{abb-cg}.

\oneline{%
\deriv{9}{
The  & small & cat & chased       & Mary & quickly                & round                     & the & garden\\
\hr  & \hr   & \hr & \hr          & \hr  & \hr                    & \hr                       & \hr & \hr\\
np/n & n/n   & n   & (s\bs np)/np & np   & (s\bs np)\bs (s\bs np) & (s\bs np)\bs (s\bs np)/np & np/n & n\\
     & \multicolumn{2}{c}{\forwardapp} & \multicolumn{2}{c@{}}{\forwardapp}\\
     & \multicolumn{2}{c}{n}           & \multicolumn{2}{c@{}}{s\bs np}\\
\multicolumn{3}{@{}c}{\forwardapp}        & \multicolumn{3}{c@{}}{\backwardapp}\\
\multicolumn{3}{@{}c}{np}                 & \multicolumn{3}{c@{}}{s\bs np}\\
&&&&&&&\multicolumn{2}{c@{}}{\forwardapp}\\
&&&&&&&\multicolumn{2}{c@{}}{np}\\
&&&&&&\multicolumn{3}{c@{}}{\forwardapp}\\
&&&&&&\multicolumn{3}{c@{}}{(s\bs np)\bs (s\bs np)}\\
&&&\multicolumn{6}{c@{}}{\backwardapp}\\
&&&\multicolumn{6}{c@{}}{s\bs np}\\
\multicolumn{9}{@{}c@{}}{\backwardapp}\\
\multicolumn{9}{@{}c@{}}{s}\\
}}
% Eine Kategorialgrammatik mit den beiden Multiplikationsregeln, die oben angegeben wurden, ist schwach äquivalent
% zu einer kontextfreien Grammatik. Solche Grammatiken wurden zuerst von Ajdukiewicz diskutiert. Die Äquivalenz
% zu den CFG wurde von Bar-Hillel bewiesen. Deshalb wird ein solches System auch AB genannt. 

% Obwohl AB schwach äquivalent zu kontextfreien Grammatiken ist, ermöglicht das System die Beschreibung
% verschiedener linguistischer Phänomene auf elegante Weise. In der CFG müßte man für entsprechende
% Beschreibungen Merkmale benutzen.
\vfill
}


\subsection{Verbstellung}

\subsubsection{Variable Verzweigung}

\frame{
\frametitle{Verbstellung}

\begin{itemize}
\item \citet[S.\,159]{Steedman2000a-u} für das Niederländische:

\eal
\ex gaf (`geben') mit Verbletztstellung: (S\sub{+SUB}$\backslash$NP)$\backslash$NP
\ex gaf (`geben') mit Verberststellung: (S\sub{$-$SUB}/NP)/NP
\zl

Das eine \emph{geben} verlangt Argumente rechts und das andere links von sich.

\pause
\item Die beiden Lexikoneinträge werden über Lexikonregeln zueinander in Beziehung gesetzt.

\end{itemize}


}
\frame{
\frametitle{Anmerkung zu dieser Analyse der Verbstellung}

Man beachte: Die NPen müssen in verschiedenen Reihenfolgen mit dem Verb kombiniert werden. Die
  Normalstellung entspricht:

\eal
\ex mit Verbletztstellung: (S\sub{+SUB}$\backslash$NP[nom])$\backslash$NP[acc]
\ex mit Verberststellung: (S\sub{$-$SUB}/NP[acc])/NP[nom]
\zl

Zur Kritik an solchen Analysen mit variabler Verzweigung siehe \citew{Mueller2005c}.


}


\subsubsection{Verbstellung mit Spur}

\frame{
\frametitle{Verbstellung mit Spur}

\citet{Jacobs91a} schlägt eine Spur in Verbletztstellung vor,\\
die die Argumente des Verbs und das Verb in Erststellung selbst als Argument verlangt.


}


\subsection{Konstituentenstellung}

\frame{
\frametitle{Konstituentenstellung}


\begin{itemize}
\item Bisher haben wir Kombination nach links und Kombination nach rechts gesehen.
      Die Abbindung der Argumente erfolgte immer in einer festen Reihenfolge (von außen nach
      innen).

\pause
\item \citet{SB2006a-u} unterscheiden: 
      \begin{itemize}
      \item Sprachen, in denen die Reihenfolge der Abbindung egal ist
\pause
      \item Sprachen, in denen die Richtung der Kombination egal ist
      \end{itemize}

\pause
\vfill
\begin{tabular}{@{}lll@{}}
Englisch   & (S$\backslash$NP)/NP     & S(VO)\\
Latein     & S\{$|$NP[nom], $|$NP[acc] \} & freie Stellung\\
Tagalog    & S\{/NP[nom], /NP[acc] \} & freie Stellung, verbinitial\\
Japanisch  & S\{$\backslash$NP[nom], $\backslash$NP[acc] \} & freie Stellung, verbfinal\\
\end{tabular}

\vfill
Elemente in Klammern in beliebiger Reihenfolge abbindbar

Steht `$|$' statt `$\backslash$' oder `/', dann ist die Abbindungsrichtung egal.

\end{itemize}

}

\subsection{Passiv}

\frame{
\frametitle{Passiv}

Lexikonregel:
\eal
\ex lieben: S\sub{+SUB} \{ $\backslash$NP[nom]$_i$, $\backslash$NP[acc]$_j$ \}
\ex geliebt: S\sub{pas} \{ $\backslash$NP[nom]$_j$, $\backslash$PP[von]$_i$ \}
\zl

}


\subsection{Fernabhängigkeiten}


\frame{
\frametitle{Fernabhängigkeiten}

\citet[S.\,614]{SB2006a-u}:
\ea
the man that Manny says Anna married
\z
\pause

Lexikoneintrag für Relativpronomen:
\ea
(N$\backslash$N)/(S/NP)
\z
Wenn ich rechts von mir einen Satz finde, dem noch eine NP fehlt,\\
dann kann ich mit dem zusammen einen N-Modifikator (N$\backslash$N) bilden. 

Das Relativpronomen ist in dieser Analyse der Kopf (Funktor).



}


\subsubsection{Type Raising}

\frame{
\frametitle{Type Raising}



Die Kategorie np kann durch {\em type raising} in die Kategorie $(s/(s\backslash np))$
umgewandelt werden. Kombiniert man diese Kategorie mit $(s\backslash np)$ erhält man dasselbe Ergebnis
wie bei einer Kombination von np und $(s\backslash np)$ mit Regel 2.
\eal
\ex np * s $\backslash$ np $\to$ s 
\ex s / (s $\backslash$ np) * s $\backslash$ np $\to$ s
\zl

\pause

Man dreht durch Type Raising die Selektionsrichtung um.

In (\mex{0}a) selegiert ein Verb (bzw.\,s) links von sich eine NP,\\
in (\mex{0}b) selegiert ein Nomen rechts von sich ein Verb (s),\\
das links von sich eine NP erwartet.

Das Ergebnis der Kombination ist in beiden Fällen ein Satz.




}


\subsubsection{Vorwärts- und Rückwärtskomposition}

\frame{
\frametitle{Vorwärts- und Rückwärtskomposition}

\ea
\begin{tabular}[t]{@{}l@{\hspace{1cm}}l}
X/Y * Y/Z = X/Z & Vorwärtskomposition~(fc) \\
Y$\backslash$Z * X$\backslash$Y = X$\backslash$Z & Rückwärtskomposition~(bc)
\end{tabular}
\z 

Beispiel Vorwärtskomposition:

Wenn ich Y finde, bin ich ein vollständiges X.

Ich habe ein Y, dem aber noch ein Z fehlt.

Wenn ich dieses Element mit X/Y verbinde, bekomme ich etwas,\\
das ein X ist, wenn es noch mit einem Z verbunden wird.

%% \begin{tabular}{@{}llll@{}}
%% Fido                & chased  & Mary\\
%% S/(S$\backslash$NP) & (s$\backslash$np)/np   & np\\
%% \multicolumn{2}{c}{\visible<3->{np}}        & \multicolumn{2}{c}{\visible<4->{s$\backslash$np}}\\
%% \multicolumn{4}{c}{\visible<5->{s}}\\
%% \end{tabular}

%% \pause\pause\pause
%% \pause
}


\subsubsection{Relativsätze mit Fernabhängigkeiten}


\frame{
\frametitle{Relativsätze mit Fernabhängigkeiten}


\deriv{5}{
that                                & Manny                                                   & says                              & Anna                 & married\\
\hr                                 & \forwardt                                               & \hr                               & \forwardt            & \hr \\
%
%
(N\bs N)/\braun<5->{(S/NP)} & S/\rot<2->{(S\bs NP)}                           & \rot<2->{(S\bs NP)}/S     & S/\gruen<3->{(S\bs NP)} & \gruen<3->{(S\bs NP)}/NP\\
                                    & \multicolumn{2}{c}{\visible<2->{\forwardc}} & \multicolumn{2}{c@{}}{\visible<3->{\forwardc}}\\
%
%
                                    & \multicolumn{2}{c@{}}{\visible<2->{S/\blau<4->{S}}}                    & \multicolumn{2}{c@{}}{\visible<3->{\blau<4->{S}/NP}}\\
                                    & \multicolumn{4}{c@{}}{\visible<4->{\forwardc}}\\
                                    & \multicolumn{4}{c@{}}{\visible<4->{\braun<5->{S/NP}}}\\
\multicolumn{5}{@{}c@{}}{\visible<5->{\forwardapp}}\\
\multicolumn{5}{@{}c@{}}{\visible<5->{N\bs N}}\\
}



}


\frame{
\frametitle{Anmerkung zu dieser Relativsatzanalyse}


Die Annahme, dass das Relativpronomen der Kopf ist, ist problematisch, da Rattenfängerkonstruktionen
wie (\mex{1}) nicht einfach erklärt werden können \citep{Pollard88a}.


\eal
\ex Here's the minister [[in [the middle [of [whose sermon]]]] the dog barked].\footnote{
\citet[S.\,212]{ps2}
}
\ex Reports [the height of the lettering on the covers of which] the government prescribes should be
abolished.\footnote{
\citet[S.\,109]{Ross67}\nocite{Ross86a-u}
}
\zl

Zu Analysen siehe \citep{Morrill95a,Steedman97a}.

}

\subsection{Übungsaufgabe}

\frame{
\frametitle{Übungsaufgabe}


Analysieren Sie den Satz:
\ea
Die Kinder im Zimmer lachen laut.
\z





}
