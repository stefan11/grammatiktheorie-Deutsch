\section{Transformationsgrammatik bis Government \& Binding (GB)}

\outline{

\begin{itemize}
\item Begriffe
\item Phrasenstrukturgrammatiken
\item \blaubf{Government \& Binding (GB)}
\item Generalisierte Phrasenstrukturgrammatik (GPSG)
\item Lexikalisch-Funktionale Grammatik (LFG)
%\item Lexical Mapping Theory (LMT)
%\item PATR
\item Kategorialgrammatik (CG)
\item Kopfgesteuerte Phrasenstrukturgrammatik (HPSG)
\item Konstruktionsgrammatik (CxG)
\item Baumadjunktionsgrammatik (TAG)
\end{itemize}
}

\iftoggle{gb-intro}{
\section{Government \& Binding}

%\if 0


\frame{
\frametitle{Gliederung}

\begin{itemize}
%\item Ziele
%\item Wiederholung von Grundbegriffen
%\item Grundfragen der Sprachwissenschaft
\item Grammatikmodelle
\begin{itemize}
\item Phrasenstrukturgrammatik
\item Transformationsgrammatik und deren Nachfolger
\begin{itemize}
\item \blaubf{Geschichtliches und Motivation}
\item Das T-Modell im Überblick
\item Grundbegriffe: Theta-Rollen, Externes Argument, \ldots
\item Lexikoneinträge
\item Syntaktische Kategorien
\item \xbar-Schemata
\item Die Struktur des deutschen Satzes
\item Kasus
\item NP-Bewegung
\item Bindungstheorie
\item \emph{w}-Bewegung
\item Inkorporation
\end{itemize}
\end{itemize}
\end{itemize}

}

}%\end{gb-intro}

\subsection{Geschichtliches und Motivation}

\frame{
\frametitle{Transformationsgrammatik}

\begin{itemize}
\item Noam Chomsky, MIT, argumentiert für Transformationen \citep{Chomsky57a}

\pause
\item Manfred Bierwisch (Dissertation Leipzig, 1960) ist der erste Deutsche, der
  transformationsgrammatische Ansätze verfolgt \citep{Bierwisch63a}

\pause
\item später an der Akademie der Wissenschaften der DDR
\pause
\item nicht immer einfach \ldots

\pause
\item Interessantes Gespräch mit Carla Umbach und Annette Leßmöllmann

\url{http://www.gespraech-manfred-bierwisch.de/}

\pause
\item West-Deutschland Lehrbuch 1971 \citep{BechertClementThummel1971a-u} 
%  und Gründung der Linguistischen
%  Berichte durch Peter Hartmann und Arnim von Stechow.

\pause
\item Transformationsgrammatik (heute Government \& Binding bzw. Minimalismus) ist weit verbreitet
      mit eigenen Tagungen, Zeitschriften, Buchreihen

\end{itemize}


}

\frame{
\frametitle{Phrasenstrukturgrammatiken und natürliche Sprachen}

Chomsky: Zusammenhänge zwischen bestimmten Sätzen (\zb Aktiv und Passiv) können nicht erfaßt
      werden. $\to$
      Transformationen:

\begin{table}[H]
\begin{tabular}{@{}l@{~}l@{~}l@{~}l}
NP& V &NP &$\to$ 3 [\sub{AUX} be] 2en [\sub{PP} [\sub{P} by] 1]\\
1 & 2 &3\\
\end{tabular}
\end{table}

\eal
\ex John loves Mary.
\ex Mary is loved by John.
\zl

Ein Baum mit der Symbolfolge auf der linken Regelseite wird auf einen Baum mit der Symbolfolge
auf der rechten Regelseite abgebildet.

}


\frame{
\frametitle{Transformation eines Aktivbaumes in einen Passivbaum}

\vfill

%\oneline{%
\hfill
\begin{forest}
%sm edges
[S, for tree={parent anchor=south, child anchor=north}
  [NP [John] ]
  [VP
    [V [loves] ]
    [NP [Mary] ] 
  ]]
\end{forest}
\hspace{1em}
\raisebox{6\baselineskip}{$\leadsto$}
\hspace{1em}
  \begin{forest}
  %sm edges
  [S, for tree={parent anchor=south, child anchor=north}
  	[NP[Mary]]
	[VP
	[Aux[is, tier=word]]
	[V[loved, tier=word]]
	[PP
	[P[by, tier=word]]
	[NP[John, tier=word]]]]]
\end{forest}
\hfill\mbox{}
%}

\vfill

\begin{tabular}{@{}l@{~}l@{~}l@{~}l}
NP& V &NP &$\to$ 3 [\sub{AUX} be] 2en [\sub{PP} [\sub{P} by] 1]\\
1 & 2 &3\\
\end{tabular}

\vfill

}


\iftoggle{einfsprachwiss-exclude}{
\frame{
\frametitle{Komplexität, Transformationen und natürliche Sprachen}

\small
\begin{itemize}
\item Es gibt Ersetzungsgrammatiken verschiedener Komplexität.\\(Chomsky-Hierarchie, Typ 3--0)
\pause
\item bisher behandelte so genannte kontextfreie Grammatiken sind vom Typ 2.
\pause
\item höchste Komplexitätsstufe (Typ 0) ist zu mächtig für natürliche Sprachen.\\
$\to$ Einschränkung nötig.
\pause
\item Transformationsgrammatiken entsprechen Typ-0-Phrasenstrukturgrammatiken hinsichtlich ihrer
      Komplexität \citep{PR73a-u}.
\pause
\item Transformationen sind nicht genügend restringiert,\\Interaktionen nicht überschaubar,\\
      Probleme mit Transformationen, die Material löschen (siehe \citew{Klenk2003a})
\pause
\item $\to$ neue theoretische Ansätze, Government \& Binding: Einschränkungen für Form der Grammatikregeln,
      Wiederauffindbarkeit von Elementen in Bäumen, allgemeine Prinzipien zur Beschränkung von Transformationen

\end{itemize}

}

\frame{
\frametitle{Hypothese zum Spracherwerb: Prinzipien \& Paramater}

\begin{itemize}
\item<+-> Ein Teil der sprachlichen Fähigkeiten ist angeboren.\\
          (Wird nicht von allen Linguisten geteilt! Diskussion: \citew{MuellerGTBuch2,MuellerGT-Eng1})
\item<+-> Prinzipien, die von sprachlichen Strukturen nicht verletzt werden dürfen.
\item<+-> Die Prinzipien sind parametrisiert, \dash es gibt Wahlmöglichkeiten.\\
      Ein Parameter kann für verschiedene Sprachen verschieden gesetzt sein.

\medskip
\pause
      Beispiel: \\

Prinzip: Ein Kopf steht in Abhängigkeit vom Parameter {\sc stellung}\\
vor oder nach seinen Komplementen.

\medskip
%% \begin{tabular}[t]{@{}l@{ }l@{}}
%%                 Englisch  &$\to$ Verb steht vor Komplementen\\
%%                 Japanisch &$\to$ Verb steht nach Komplementen\\
%%                 \end{tabular}


\eal
\ex 
\gll be showing pictures of himself\\
     ist zeigend Bilder von sich\\\jambox{(English)}
\ex
\gll zibun -no syasin-o mise-te iru\\
     sich  von Bild     zeigend sein\\\jambox{(Japanisch)}
\zl

\end{itemize}

}
}%\end{einfsprachwiss-exclude}

\lecture{T-Modell}{t-modell-lec}

\subsection{Das T-Modell}


\iftoggle{gb-intro}{%
\frame{

\frametitle{Gliederung}

\begin{itemize}
%% \item Ziele
%% \item Wiederholung von Grundbegriffen
%% \item Grundfragen der Sprachwissenschaft
\item Grammatikmodelle
\begin{itemize}
\item Phrasenstrukturgrammatik
\item Transformationsgrammatik und deren Nachfolger
\begin{itemize}
\item Geschichtliches und Motivation
\item \blaubf{Das T-Modell im Überblick}
\item Grundbegriffe: Theta-Rollen, Externes Argument, \ldots
\item Lexikoneinträge
\item Syntaktische Kategorien
\item \xbar-Schemata
\item Die Struktur des deutschen Satzes
\item Kasus
\item NP-Bewegung
\item Bindungstheorie
\item \emph{w}-Bewegung
\item Inkorporation
\end{itemize}
\end{itemize}
\end{itemize}

}

}%\end{gb-intro}

\frame{
\frametitle{Tiefen- und Oberflächenstruktur}

\begin{itemize}[<+->]
\item Chomsky hat behauptet, dass man mit einfachen PSGen gewisse Zusammenhänge nicht adäquat erfassen kann.\\
      \ZB den Zusammenhang zwischen Aktivsätzen und Passivsätzen.

\item Er nimmt deshalb eine zugrundeliegende Struktur an,\\
      die sogenannte \blaubf{Tiefenstruktur}.

\item Eine Struktur kann auf eine andere Struktur abgebildet werden.

Dabei können \zb Teile gelöscht oder umgestellt werden. 

Als Folge solcher Transformationen gelangt man
von der Tiefenstruktur zu einer neuen Struktur, der \blaubf{Oberflächenstruktur}.

\medskip
\begin{tabular}{@{}l@{ = }l@{}}
\emph{Surface Structure} & S-Struktur\\
\emph{Deep Structure} & D-Struktur\\
\end{tabular}
\end{itemize}

}

%\beamertemplatebackfindforwardnavigationsymbolshorizontal


\frame[label=t-modell]{
\frametitle{Das T-Modell}


%% \centerline{%
%% \resizebox{0.8\linewidth}{!}{
%% \begin{tabular}{@{}ccc@{}}
%% \xbar-Theorie der \\
%% \node{psr}{Phrasenstrukturregeln} & & \hyperlink{lexikon}{\node{lex}{Lexikon}}\\[6ex]
%% &\hyperlink{ds}{\node{ds}{D-Struktur}}\\[2ex]
%% \mc{3}{@{}c@{}}{\hyperlink{move-alpha}{move-$\alpha$}}\\[4ex]
%% &\hyperlink{ss}{\node{ss}{S-Struktur}}\\[6ex]
%% \node{tilg}{Tilgungsregeln},             && \node{ana}{Regeln des anaphorischen Bezugs,}\\
%% \node{filter}{Filter, phonol.\ Regeln}  && \node{quant}{der Quantifizierung und der Kontrolle}\\[6ex]
%% \hyperlink{pf}{\node{pf}{Phonetische}}             && \hyperlink{lf}{\node{lf}{Logische}}\\
%% \hyperlink{pf}{Form (PF)}               && \hyperlink{lf}{Form (LF)}\\
%% \end{tabular}
%% \anodeconnect{psr}{ds}\anodeconnect{lex}{ds}
%% \anodeconnect{ds}{ss}
%% \anodeconnect{ss}{tilg}\anodeconnect{ss}{ana}
%% \anodeconnect{filter}{pf}\anodeconnect{quant}{lf}
%% }}

\vfill
\centerline{%
\begin{forest}
for tree = {edge={->},l=4\baselineskip}
[D-structure
     [S-structure,edge label={node[midway,right]{move $\alpha$}} 
            [Deletion rules{,}\\Filter{,} phonol.\ rules
                    [Phonetic\\Form (PF)]]
            [Anaphoric rules{,}\\rules of quantification and control
                    [Logical\\Form (LF)]]]]
\end{forest}}

\vfill

}



\gotobuttonleft{t-modell}{T-Modell}

\frame[label=lexikon]{
\frametitle{Das T-Modell: Das Lexikon}
\showsingleitemframe
\savespace
\begin{itemize}
\item<+> Enthält zu jedem Wort einen Lexikoneintrag mit Information zu:
\begin{itemize}\itemsep0pt
\item morphophonologischer Struktur
\item syntaktischen Merkmalen
\item Selektionsraster (=~Valenzrahmen)
\item \ldots
\end{itemize}
Beinhaltet Wortformen- und Morphemliste sowie eine Wortbildungskomponente (=~Morphologie)

\item<+> Lexikon bildet Schnittstelle zur semantischen Interpretation\\
der einzelnen Wortformen.

\item<+> Wortschatz ist nicht von Universalgrammatik bestimmt,\\
nur bestimmte strukturelle Anforderungen sind prädeterminiert

\item<+> Morphosyntaktische Merkmale (\zb Genus) ebenfalls nicht vorbestimmt:
Universalgrammatik gibt nur Bandbreite von Möglichkeiten vor.
% und setzt einige Minimalanforderungen.
\end{itemize}

}

\frame{
\frametitle{Das T-Modell: D-Struktur, Move-$\alpha$ und S-Struktur (I)}

\begin{itemize}
\item<+-> Phrasenstruktur $\to$\\
Beschreibung der Beziehungen zwischen einzelnen Elementen mögl.

\item<+-> Gewisses Format für Regeln ist vorgegeben (\xbar-Schema).

Lexikon + Strukturen der \xbar-Syntax = Basis für die D-Struktur

\hypertarget{ds}{D-Struktur} = syntaktische Repräsentation der im Lexikon festgelegten
Selektionsraster (Valenzrahmen) der einzelnen Wortformen.
\end{itemize}


}

\frame{
\frametitle{Das T-Modell: D-Struktur, Move-$\alpha$ und S-Struktur (II)}

\begin{itemize}
\item<+-> Konstituenten stehen an der Oberfläche nicht unbedingt an der Stelle,
die der Valenzrahmen vorgibt:
\eal
\ex {}[dass] der Mann der Frau das Buch gibt
\ex Gibt der Mann der Frau das Buch?
\ex Der Mann gibt der Frau das Buch.
\zl
\item<+-> deshalb Transformationsregel für Umstellungen:\\
\hypertarget{move-alpha}{Move-$\alpha$} = "`Bewege irgendetwas irgendwohin!"'

Was genau wie und warum bewegt werden kann,\\
wird von Prinzipien geregelt.

\end{itemize}


}

\frame[label=ss]{
\frametitle{Das T-Modell: D-Struktur, Move-$\alpha$ und S-Struktur (III)}

\begin{itemize}
\item Von Lexikoneinträgen bestimmte Relationen zwischen einem Prädikat
und seinen Argumenten müssen für semantische Interpretation auf allen Repräsentationsebenen auffindbar sein.

\item
$\to$ Ausgangspositionen bewegter Elemente durch Spuren markiert.

\eal
\ex {}[dass] der Mann der Frau das Buch gibt
\ex Gibt$_i$ der Mann der Frau das Buch \_$_i$?
\ex {}[Der Mann]$_j$ gibt$_i$ \_$_j$ der Frau das Buch \_$_i$.
\zl

Verschiedene Spuren werden mit Indizes markiert.\\
Andere Darstellung: \emph{e} oder \emph{t}.

\item
\hypertarget{ss}{S-Struktur} ist eine oberflächennahe Struktur,\\
darf aber nicht mit real geäußerten Sätzen gleichgesetzt werden.
\end{itemize}
}


\frame[label=pf]{
\frametitle{Das T-Modell: Die Phonetische Form}

Auf PF werden phonologische Gesetzmäßigkeiten eines Satzes repräsentiert. 

Sie stellt den Output zum Sprechmodul dar.

\pause
Beispiel: \emph{wanna}"=Kontraktion

\eal
\ex The students want to visit Paris.
\ex The students wanna visit Paris.
\zl
Die Kontraktion in (\mex{0}) wird durch die optionale Regel in (\mex{1}) lizenziert:
\ea
want + to $\to$ wanna
\z


}

\frame[label=lf]{
\frametitle{Das T-Modell: Die Logische Form (I)}

\begin{itemize}
\item Logische Form ist eine syntaktische Ebene, die zwischen der S-Struktur und der
semantischen Interpretation eines Satzes vermittelt.

anaphorischer Bezug (Bindung): Worauf bezieht sich ein Pronomen?
\eal
\ex Peter kauft einen Tisch. Er gefällt ihm.
\ex Peter kauft eine Tasche. Er gefällt ihm.
\ex Peter kauft eine Tasche. Er gefällt sich.
\zl

\pause
\item Quantifizierung:
\ea
Every man loves a woman.
\z
$\forall x \exists y (man(x) \to (woman(y) \wedge love(x,y))$\\
$\exists y \forall x (man(x) \to (woman(y) \wedge love(x,y))$
\end{itemize}

}

\frame{
\frametitle{Das T-Modell: Die Logische Form (II)}

Kontrolltheorie:\\
Wodurch wird die semantische Rolle des Infinitiv-Subjekts gefüllt?

\eal
\ex Der Professor schlägt dem Studenten vor,\\die Klausur noch mal zu schreiben.
\pause
\ex Der Professor schlägt dem Studenten vor,\\die Klausur nicht zu bewerten.
\pause
\ex Der Professor schlägt dem Studenten vor,\\gemeinsam ins Kino zu gehen.
\zl

}

%\beamertemplatenavigationsymbolsempty 
\mode<beamer>{\beamertemplatebackfindforwardnavigationsymbolshorizontal}

\subsection{Das Lexikon}

\frame{
\frametitle{Lexikon: Grundbegriffe (I)}

\small
\begin{itemize}
\item Bedeutung von Wörtern $\to$ Phrasenbildung mit bestimmten Rollen ("`handelnde Person"'
      oder "`betroffene Sache"')

      Beispiel: semantischer Beitrag von \emph{kennen} in (\mex{1}a) ist (\mex{1}b):
      \eal
\ex Maria kennt den Mann.
\ex kennen'(x,y)
\zl
\pause
\item Solche Beziehungen werden mit dem Begriff \blaubf{Selektion} bzw.\ \blaubf{Valenz} erfaßt.

Achtung:\\
Logische Valenz kann sich von syntaktischer Valenz unterscheiden!
\pause
\item Man spricht auch von \blaubf{Subkategorisierung}:

\emph{kennen} ist für ein Subjekt und ein Objekt subkategorisiert.

Das Wort \emph{subkategorisieren} hat sich verselbständigt, auch wie folgt gebraucht:

\emph{kennen} subkategorisiert für ein Subjekt und ein Objekt.
\end{itemize}


}

\lecture{Lexikon}{lexikon}

\frame{
\frametitle{Lexikon: Grundbegriffe (II)}
\savespace
\begin{itemize}
\item \emph{kennen} wird auch \blaubf{Prädikat} genannt\\
(weil \emph{kennen'} das logische Prädikat ist).

Vorsicht:\\
entspricht nicht der Verwendung des Begriffs in der Schulgrammatik.
\pause
\item Subjekt und Objekt sind die \blaubf{Argumente} des Prädikats.
\pause
\item Spricht man von der Gesamtheit der Selektionsanforderungen, verwendet man
Begriffe wie \blaubf{Argumentstruktur}, \blaubf{Valenzrahmen}, \blaubf{Selektionsraster},
\blaubf{Subkategorisierungsrahmen}, \blaubf{thematisches Raster} oder
\blaubf{Theta-Raster} = \blaubf{$\theta$-Raster}
(\emph{thematic grid}, \emph{Theta-grid})
\pause
\item \blaubf{Adjunkte} (oder Angaben) modifizieren semantische Prädikate,\\
man spricht auch von Modifikatoren.\\
Adjunkte sind in Argumentstrukturen von Prädikaten nicht vorangelegt.
\end{itemize}



}



\frame{
\frametitle{Das Theta-Kriterium}

\label{theta-kriterium}%
Argumente nehmen im Satz typische Positionen (Argumentpositionen) ein.

Theta-Kriterium:
\begin{itemize}
\item Jede Theta-Rolle wird an genau eine Argumentposition vergeben.
\item Jede Phrase an einer Argumentposition bekommt genau eine Theta-Rolle.
\end{itemize}

}


\frame{
\frametitle{Externes Argument und interne Argumente}

\savespace
\begin{itemize}[<+->]
\item Argumente stehen in Rangordnung, \dash, man kann zwischen ranghohen und rangniedrigen Argumenten unterscheiden.

\item Ranghöchstes Argument von V und A hat besonderen Status. 

Weil es oft (im manchen Sprachen: immer) an einer Position außerhalb der Verbal- bzw.\ Adjektivphrase steht,\\
wird es auch als \blaubf{\hypertarget{ext-arg}{externes Argument}} bezeichnet. 

\item Die übrigen Argumente stehen an Positionen\\
      innerhalb der Verbal- bzw. Adjektivphrasen.

Bezeichnung: \blaubf{internes Argument} oder \blaubf{Komplement}

\item Faustregel für einfache Sätze: Externes Argument = Subjekt.
\end{itemize}

}

\frame{
\frametitle{Einzelne Theta-Rollen}
%\savespace
\begin{itemize}
\item<+-> Wenn es sich bei den Argumenten um Aktanten handelt,\\
kann man drei Klassen von Theta-Rollen unterscheiden. 

\item<+-> Wenn ein Verb mehrere Theta-Rollen dieser Art vergibt,\\
hat Klasse 1 gewöhnlich den höchsten Rang, Klasse 3 den niedrigsten:   

\begin{itemize}
\item Klasse 1: \blaubf{Agens} (handelnde Person), Auslöser eines Vorgangs oder Auslöser einer Empfindung (Stimulus), \blaubf{Träger einer Eigenschaft}
\item<+-> Klasse 2: \blaubf{Experiencer} (wahrnehmende Person), \blaubf{nutznießende Person} (Benefaktiv) (oder auch das Gegenteil: vom einem Schaden betroffene Person), \blaubf{Possessor} (Besitzer, Besitzergreifer   oder auch das Gegenteil: Person, der etwas abhanden kommt oder fehlt)
\item<+-> Klasse 3: \blaubf{Patiens} (betroffene Person oder Sache), \blaubf{Thema}
\end{itemize}

\item<+-> Vorsicht!\\
großes Wirrwarr bei Zuordnungen von semantischen Rollen zu Verben
\nocite{Gruber65a-u,Fillmore68,Fillmore71a-u,Jackendoff72a-u,Dowty91a}
\end{itemize}
}

\iftoggle{gb-intro}{
\frame{
\frametitle{Theta-Rolle und syntaktische Kategorie (I)}


\begin{itemize}
\item Argumente meist als \blaubf{Nominalphrasen} realisiert:
\ea
{}[\sub{NP} Der Beamte] verlangte [\sub{NP} einen schriftlichen Antrag].
\z
\pause
\item bei passender Theta-Rolle auch als \blaubf{Nebensatz}:
\eal
\ex Der Beamte verlangte,\\
    {}[dass der Antrag schriftlich eingereicht wird].
\ex Der Beamte verlangte,\\
    {}[den Antrag schriftlich einzureichen].
\zl
\pause
\item oder als \hyperlink{sc}{\blaubf{Small Clause}} = satzwertige Fügung ohne Verb,\\
      bestehend aus Nominalphrase + Prädikativ
\eal
\ex Anna findet, [dass das Häschen niedlich ist]. (Nebensatz)
\ex Anna findet [das Häschen niedlich]. (Small Clause)
\zl
\eal
\ex Otto macht, [dass der Tisch sauber wird]. (Nebensatz)
\ex Otto macht [den Tisch sauber]. (Small Clause)
\zl
\end{itemize}

}

\frame{
\frametitle{Theta-Rolle und syntaktische Kategorie (II)}

Angabe der syntaktischen Kategorie eines Arguments (NP, Nebensatz%
%\iftoggle{gb-intro}{, Small Clause}
) 
im Theta-Raster meist unnötig ($\to$ freie Wahl).

\medskip
Allerdings:
\eal
\ex[]{
Er weiß, dass Peter kommt.
}
\ex[*]{
Er kennt, dass Peter kommt.
}
\ex[]{
Er kennt die Vermutung, dass Peter kommt.
}
\zl

}
} % gb-intro


\frame{
\frametitle{Ein Lexikoneintrag (I)}

\small
Über welche Information muss man verfügen, um ein Wort sinnvoll anzuwenden? 

Antwort: Das mentale Lexikon enthält Lexikoneinträge (englisch: \emph{lexical entries}),\\ 
in denen die spezifischen Eigenschaften der syntaktischen Wörter aufgeführt sind:  

\begin{itemize}
\item Form
\item Bedeutung (Semantik)
\item Grammatische Merkmale:\\
      syntaktische Wortart + morphosyntaktische Merkmale   
\item Theta-Raster
\end{itemize}



}

\frame{
\frametitle{Ein Lexikoneintrag (II)}


{\footnotesize
\begin{tabular}{|l|ll|}
\hline
Form     & \emph{hilft}&\\\hline
Semantik & helfen'     &\\\hline
Grammatische Merkmale & \multicolumn{2}{l|}{Verb, 3.\ Person Singular Indikativ Präsens Aktiv}\\\hline\hline
%\setlength{\arrayrulewidth}{9pt}
Theta-Raster                &&\\\hline
Theta-Rollen                & \underline{Agens} & Benefaktiv\\[2mm]\hline
Grammatische Besonderheiten &                   & Dativ\\\hline
\end{tabular}
}

\bigskip
Argumente sind nach dem Rang geordnet:\\
ranghöchstes Argument steht ganz links. 

In diesem Fall ist das ranghöchste Argument das externe Argument.

Das externe Argument wird unterstrichen.

}

\iftoggle{gb-intro}{
\frame{
\frametitle{Ein Lexikoneintrag (III)}


\begin{itemize}[<+->]
\item Bei den grammatischen Besonderheiten wird nur angegeben,\\
was nicht von allgemeinen Regeln abgeleitet werden kann,\\
also besonders zu lernen ist. 

\item \emph{helfen} $\to$ Kasus des internen Arguments = Dativ\\
(Zweiwertige Verben haben sonst internes Argument im Akkusativ)

\item Kasus des externen Arguments folgt aus allgemeinen Regeln:\\
In einfachen Sätzen erscheint es als Subjekt und erhält Nominativ. 

\item Argumente von \emph{helfen} werden gewöhnlich als NPen realisiert.
\end{itemize}
}
}% gb-intro


\lecture{\xbar-Theorie}{x-bar}
\subsection{\xbar-Theorie}


\iftoggle{gb-intro}{
\frame{
\frametitle{Gliederung}

\begin{itemize}
%% \item Ziele
%% \item Wiederholung von Grundbegriffen
%% \item Grundfragen der Sprachwissenschaft
\item Grammatikmodelle
\begin{itemize}
\item Phrasenstrukturgrammatik
\item Transformationsgrammatik und deren Nachfolger
\begin{itemize}
\item Geschichtliches und Motivation
\item Das T-Modell im Überblick
\item Grundbegriffe: Theta-Rollen, Externes Argument, \ldots
\item Lexikoneinträge
\item Syntaktische Kategorien
\item \blaubf{\xbar-Schemata}
\item Die Struktur des deutschen Satzes
\item Kasus
\item NP-Bewegung
\item Bindungstheorie
\item \emph{w}-Bewegung
\item Inkorporation
\end{itemize}
\end{itemize}
\end{itemize}

}
}%\end{gb-intro}


\frame{

\frametitle{Anmerkung zur Verbreitung von \xbar-Regeln}

\xbar-Theorie wird auch in vielen anderen Frameworks angenommen:\\
\begin{itemize}
%\item Government \& Binding (GB):\\\citew*{Chomsky93a}
\item Lexical Functional Grammar (LFG):\\\citew{Bresnan82a-ed,Bresnan2001a,BF96a-ed,Berman2003a}
\item Generalized Phrase Structure Grammar (GPSG):\\ \citew*{GKPS85a}
\end{itemize}

Es wird nicht unbedingt dasselbe Kategorieninventar benutzt.\\
Insbesondere bei sogenannten funktionalen Kategorien (\zb INFL).

}

\subsubsection{Köpfe}


\frame{
\frametitle{\xbar-Theorie: Köpfe}
\pause
Kopf bestimmt die wichtigsten Eigenschaften\\
einer Wortgruppe/Phrase/Projektion
\eal
\ex Karl \blaubf{schläft}.
\ex Karl \blaubf{liebt} Maria.
\ex \blaubf{in} diesem Haus
\ex ein \blaubf{Mann}
\zl


}

\subsubsection{Lexikalische Kategorien}

\frame{
\frametitle{\xbar-Theorie: Lexikalische Kategorien}
\label{slide-lex-kat-gb}%

Untereinteilung in lexikalische und funktionale Kategorien\\ 
($\approx$ Unterscheidung zwischen offenen und geschlossenen Wortklassen)

Lexikalische Kategorien: 
\begin{itemize}
\item V = Verb
\item N = Nomen
\item A = Adjektiv
\item P = Präposition
\item Adv = Adverb
\end{itemize}
% Abney87a:64--65 Meinunger2000a:38--39

}

\iftoggle{einfsprachwiss-exclude}{
\frame[shrink=10]{
\frametitle{\xbar-Theorie: Lexikalische Kategorien (Kreuzklassifikation)}
%\label{slide-lex-kat-gb}%


Versuch, die lexikalischen Kategorien mittels Kreuzklassifikation auf elementarere Merkmale zurückzuführen:

\bigskip

\centerline{\renewcommand{\arraystretch}{1.5}
\begin{tabular}{|c|c|c|}\hline
 & $-$ V & + V \\\hline
$-$ N & P = [ $-$ N, $-$ V] &  V = [ $-$ N, + V] \\\hline
  + N & N = [+ N, $-$ V]    &  A = [+ N, + V]\\\hline
\end{tabular}
}
\pause

\bigskip

Kreuzklassifikation $\to$ einfach auf Adjektive und Verben referieren:\\
Alle lexikalischen Kategorien, die [ + V] sind,\\
sind entweder Adjektive oder Verben.

Generalisierungen mgl.\ \zb: [ + N]-Kategorien können Kasus tragen 
% [ $-$ N]-Kategorien können Kasus regieren (allerdings auch A)
\medskip

Anmerkung: Adverbien können als einstellige Präpositionen behandelt werden.\nocite{Chomsky70a}

}



\frame{
\frametitle{Kopfposition in Abhängigkeit von Kategorie}

Bei Präpositionen und Nomina steht der Kopf vorn:
\eal
\ex \gruen{für} Marie
\ex \gruen{Bild} von Maria
\zl

Bei Adjektiven und Verben steht er hinten:
\eal
\ex dem König \gruen{treu}
\ex dem Mann \gruen{helfen}
\zl

\pause

$\to$ \begin{tabular}[t]{@{}l@{}}
      {}[+ V] $\equiv$ Kopf hinten\\
      {}[$-$ V] $\equiv$ Kopf vorn\\
      \end{tabular}

\pause
Problem: Postpositionen (P = [$-$ V])
\ea
des Geldes wegen
\z

}
}%\end{einfsprachwiss-exclude}

\iftoggle{gb-intro}{
\frame{

\frametitle{Kasuszuweisung in Abhängigkeit von Kategorie}

\citet[S.\,48]{Chomsky93a}:
Im Englischen haben nur Präpositionen und Verben kasustragende Argumente.

Nur [$-$ N]-Kategorien weisen im Englischen Kasus zu.

\pause
Wie ist das im Deutschen?
\pause
\ea
dem König treu
\z
\pause
Trotzdem will man mitunter Teilklassen herausgreifen,\\
um etwas über sie zu sagen.
\pause
Oder eventuell: [$-$ N]-Kategorien weisen in allen Sprachen Kasus zu,
in manchen Sprachen kommen noch weitere Kategorien hinzu.

}

\frame{
\frametitle{Anmerkung zu Kreuzklassifikationen}

\begin{itemize}
\item Beschreibungen mittels Merkmalen, die +/$-$ als Wert haben,\\
machen Vorhersagen über mögliche Wertbelegungen.

Zwei Merkmale $\to$ vier Kombinationen,\\
Drei Merkmale $\to$ acht Kombinationen

\item Wenn Wertkombinationen nicht belegt sind,\\
sind binäre Merkmale nicht geeignet.

\end{itemize}

}
}%\end{gb-intro}

\subsubsection{Funktionale Kategorien}

\frame{
\frametitle{\xbar-Theorie: Funktionale Kategorien}

%\begin{gb-intro}
Keine Kreuzklassifikation:\bigskip
%\end{gb-intro}


\begin{tabular}{lp{\linewidth}@{}}
C   & COMP = complementizer\\[\baselineskip]
I   & Finitheit (sowie Tempus und Modus);\\
    & in älteren Arbeiten auch INFL (engl. inflection = Flexion),\\
    & in neueren Arbeiten auch T (Tempus) \\[\baselineskip]
\iftoggle{gb-intro}{
Agr & Agreement (Übereinstimmung, Kongruenz)\\[\baselineskip]
}
D   & Determinierer (Artikelwort)\\[\baselineskip]
\end{tabular}


}



\subsubsection{Annahmen und Regeln}


\frame{
\frametitle{\xbar-Theorie: Annahmen (I)}

\begin{itemize}
\item \blaubf{Endozentrizität}:\\
Jede Phrase hat einen Kopf,\\
und jeder Kopf ist in eine Phrase eingebettet.\\
(fachsprachlich: Jeder Kopf wird zu einer Phrase projiziert.)\\
Phrase und Kopf haben die gleiche syntaktische Kategorie. 

\iftoggle{gb-intro}{
\pause
\item \blaubf{Binarität} als heute vorherrschende Annahme:\nocite{Kayne84a-u}\\
Phrasenstrukturen verzweigen binär,\\
\dash, es gibt keine Drei- oder Mehrfachverzweigungen. 
}
\pause
\item Die Äste von Baumstrukturen können sich nicht überkreuzen.\\
(\emph{Non-Tangling Condition})
\end{itemize}

}

% \frame{
% \frametitle{\xbar-Theorie: Annahmen (II)}

% Phrasen sind mindestens dreistöckig:
% \begin{itemize}
% \item X$^0$ = Kopf
% \item X' = Zwischenebene (X-Bar, X-Strich; $\to$ Name des Schemas) 
% \item XP = oberster Knoten (=~X'' = $\overline{\overline{\mbox{X}}}$), auch Maximalprojektion genannt
% \end{itemize}
% Neuere Analysen $\to$ teilweise Verzicht auf nichtverzweigende X'-Knoten
% \nocite{Muysken82a}


% }

% \frame[shrink]{
% \frametitle{Minimaler und maximaler Ausbau von Phrasen}

% \bigskip

% \small\centerline{\begin{tabular}[t]{c}
% \node{xp}{XP}\\[5ex]
% \node{xs}{X'}\\[5ex]
% \node{x}{X}\\
% \end{tabular}%
% \nodeconnect{xp}{xs}\nodeconnect{xs}{x}\hspace{10ex}%
% \begin{tabular}[t]{cccc}
% \multicolumn{2}{c}{\hspace{18mm}\node{xp2}{XP}}\\[5ex]
% \node{spec}{Spezifikator} & \multicolumn{2}{c}{\node{xs2}{X'}}\\[5ex]
%                           & \node{adj}{Adjunkt} & \multicolumn{2}{c}{~~~~~\node{xs22}{X'}}\\[5ex]
%                           &                     & \node{comp}{Komplement} & \node{x2}{X}\\
% \end{tabular}
% \nodeconnect{xp2}{spec}\nodeconnect{xp2}{xs2}%
% \nodeconnect{xs2}{adj}\nodeconnect{xs2}{xs22}%
% \nodeconnect{xs22}{comp}\nodeconnect{xs22}{x2}}


% \begin{itemize}
% \item Adjunkte sind optional\\
% $\to$ muss nicht unbedingt ein X' mit Adjunkttochter geben.
% \pause
% \item Für manche Kategorien gibt es keinen Spezifikator, oder er ist optional.\\
% %(Zusätzliche Regel nötig $\overline{\overline{\mbox{X}}} \rightarrow \xbar$)
% \pause
% \item zusätzlich mitunter: Adjunkte an XP und Kopfadjunkte an X. 
% \ifthenelse{\boolean{gb-intro}}{
% (dazu \hyperlink{inkorporation}{später})
% }{}
% \end{itemize}

% }


% \frame{
% \frametitle{\xbar-Theorie: Regeln nach \citep{Jackendoff77a}}\nocite{KP90a}\nocite{Pullum85a}



% \oneline{\(
% \begin{array}{@{}l@{\hspace{1cm}}l@{\hspace{1cm}}l}
% \xbar\mbox{-Regel} & \mbox{mit Kategorien} & \mbox{Beispiel}\\[2mm]
% \overline{\overline{\mbox{X}}} \rightarrow \overline{\overline{\mbox{Spezifikator}}}~~\xbar & \overline{\overline{\mbox{N}}} \rightarrow \overline{\overline{\mbox{DET}}}~~\overline{\mbox{N}} & \mbox{das [Bild von Maria]} \\
% \xbar \rightarrow \xbar~~\overline{\overline{\mbox{Adjunkt}}}             & \overline{\mbox{N}} \rightarrow \overline{\mbox{N}}~~\overline{\overline{\mbox{REL\_SATZ}}} & \mbox{[Bild von Maria] [das alle kennen]}\\
% \xbar \rightarrow \overline{\overline{\mbox{Adjunkt}}}~~\xbar             & \overline{\mbox{N}} \rightarrow \overline{\overline{\mbox{ADJ}}}~~\overline{\mbox{N}} & \mbox{schöne [Bild von Maria]}\\
% \xbar \rightarrow \mbox{X}~~\overline{\overline{\mbox{Komplement}}}*               & \overline{\mbox{N}} \rightarrow \mbox{N}~~\overline{\overline{\mbox{P}}} & \mbox{Bild [von Maria]}\\\\
% \end{array}
% \)}

% X steht für beliebige Kategorie, X ist Kopf,\\
% `*' steht für beliebig viele Wiederholungen

% \medskip
% X kann links oder rechts in Regeln stehen

% }

%\fi

\iftoggle{gb-intro}{
\subsubsection{Verbalphrasen}


\frame{
\frametitle{Verbalphrasen: Externes Argument und Spezifikator}
\savespace
\begin{itemize}
\item ranghöchstes Verbargument hat besonderen Status.\\
      (in einfachen Sätzen = Subjekt)
\begin{itemize}
\item Standardannahme: ranghöchstes Argument steht immer außerhalb der VP ($\to$~\hyperlink{ext-arg}{externes Argument}).
VP hat keinen Spezifikator.
\pause
\item neuere Arbeiten: Subjekt wird zunächst als VP-Spezifikator generiert.\nocite{FS86a-u,KS91a-u} 
In einigen Sprachen wird es von dort aus immer an eine Position außerhalb der VP angehoben, in anderen Sprachen, so auch im Deutschen, zumindest unter bestimmten Bedingungen (zum Beispiel bei Definitheit).\nocite{Diesing92a}
\end{itemize}
\pause
Wir folgen der Standardannahme.
%\iftoggle{{gb-intro}{\\
%Einzelheiten und Besonderheiten (Passiv, nichtakkusativische V)  später.}
\pause
\item Übrige Argumente sind Komplemente der VP (=~interne Argumente).

Wenn Verb ein einziges Komplement verlangt,\\
ist dieses nach \xbar-Schema Schwester des Kopfes V$^0$ und Tochter von V'.\\
Prototypisches Komplement: Akkusativobjekt.
\end{itemize}


}



\frame[shrink=13]{
\frametitle{Verbalphrasen: Adjunkte}

Adjunkte (freie Angaben) zweigen entsprechend dem \xbar-Schema oberhalb der Komplemente von V' ab.

\ea
weil der Mann morgen den Jungen trifft
\z

{\small\begin{tabular}{ccc}
\multicolumn{2}{c}{\node{vp}{VP}}\\[4ex]
\multicolumn{2}{c}{\node{vs}{V'}}\\[4ex]
\node{adv}{Adv} & \multicolumn{2}{c}{\node{vs2}{V'}}\\[4ex]
                & \node{np}{NP} & \node{v}{V}\\[5ex]
\node{nat}{morgen} & de\node{dj}{n Jung}en    & \node{kennt}{trifft}\\
\end{tabular}
\nodeconnect{vp}{vs}%
\nodeconnect{vs}{adv}\nodeconnect{vs}{vs2}%
\nodeconnect{adv}{nat}%
\nodeconnect{vs2}{np}\nodeconnect{vs2}{v}%
\nodetriangle{np}{dj}%
\nodeconnect{v}{kennt}
}



}



\frame{
\frametitle{Verbalphrasen: Dreistellige Verben (I)}


Was passiert mit dreistelligen Verben (Verben mit zwei Komplementen)?
\ea
als Anna ihrer Freundin den Brief zeigte
\z
\pause
{\small\begin{tabular}{ccc}
\multicolumn{3}{c}{\node{vpd}{VP}}\\[4ex]
\multicolumn{3}{c}{\node{vsd}{V'}}\\[6ex]
\node{npd}{NP} & \node{npd2}{NP} & \node{vd}{V}\\[5ex]
ihr\node{if}{er Freun}din & de\node{db}{n Bri}ef    & \node{zeigte}{zeigte}\\
\end{tabular}
\nodeconnect{vpd}{vsd}%
\nodeconnect{vsd}{npd}\nodeconnect{vsd}{npd2}\nodeconnect{vsd}{vd}%
\nodetriangle{npd}{if}%
\nodetriangle{npd2}{db}%
\nodeconnect{vd}{zeigte}}

\vfill

Struktur ist nicht binär verzweigend.


}

\frame{
\frametitle{Verbalphrasen: Dreistellige Verben (II)}


Alternative: binär verzweigende Strukturen
%% \ea
%% als Anna ihrer Freundin den Brief zeigte.
%% \z

\begin{tabular}{ccc}
\multicolumn{2}{c}{\node{vpd3}{VP}}\\[4ex]
\multicolumn{2}{c}{\node{vsd3}{V'}}\\[4ex]
\node{npd3}{NP} & \multicolumn{2}{c}{\node{vsd32}{V'}}\\[4ex]
                & \node{npd32}{NP} & \node{vd3}{V}\\[5ex]
ih\node{if3}{rer Freund}in & de\node{db3}{n Bri}ef    & \node{zeigte3}{zeigte}\\
\end{tabular}
\nodeconnect{vpd3}{vsd3}%
\nodeconnect{vsd3}{npd3}\nodeconnect{vsd3}{vsd32}%
\nodetriangle{npd3}{if3}%
\nodeconnect{vsd32}{npd32}\nodeconnect{vsd32}{vd3}%
\nodetriangle{npd32}{db3}%
\nodeconnect{vd3}{zeigte3}

}

\frame{
\frametitle{Verbalphrasen: Binarität}

Binär verzweigende Strukturen:
\begin{itemize}
\item brauchen zusätzliche Regeln für Rekursion mit X' und Argumenten\\
brauchen Mechanismus, der sicherstellt,\\
dass Adjunkte nach Komplementen mit Köpfen verbunden werden.
\pause
\item alternativ:\\
      zusätzliche Kategorien, die helfen, das \xbar-Schema einzuhalten\\
      Stichwort VP-Shell-Analyse \citep{Larson88a-u}
\end{itemize}

}

\frame{
\frametitle{Binär verzweigende vs.\ flache Strukturen und Lernbarkeit}

Die Argumentation in \citew[Kapitel~2.5]{Haegeman94a-u} ist dubios.

Manche der betrachteten Strukturen scheiden schon aus semantischen Gründen aus.

Mit Lernbarkeitsargumenten wird viel Schindluder getrieben.

%% \pause
%% Beispiel für Lernbarkeitsargumentation:\\

%% Daten X, Y, Z sind Evidenz dafür, dass Sprache A eine VP hat.\\
%% Sprecher der Sprache A haben keine für sie zugängliche Evidenz im Input,
%% die es ermöglicht, diese Tatsache zu lernen. $\to$ Existenz der VP
%% ist Bestandteil der Universalgrammatik und damit angeboren. $\to$
%% alle Sprachen besitzen eine VP.
%%
%% Fanselow87a:Chapter 1




}



\subsubsection{Nominalphrasen}

\frame{
\frametitle{Nominalphrasen: Spezifikatoren und Adjunkte}

\begin{columns}

\column{80mm}
Artikelwörter (Determinierer) sind Spezifikatoren der NP. 

\bigskip
Etikettierung der Kategorie in der Fachliteratur:\\
D (Determinierer) oder Art (Artikel). 

\bigskip
Attributive Adjektivphrasen (AP) sind Adjunkte; sie werden normalerweise links angefügt:

\bigskip
\uncover<2->{
Relativsätze sind ebenfalls Adjunkte, stehen aber rechts vom Kopf N$^0$.
}
\column{40mm}
\only<1->{
{\small%
\begin{tabular}{cccc}
\multicolumn{3}{c}{\node{np}{NP}}\\[4ex]
\node{dp}{DP}  & \multicolumn{3}{c}{\node{ns}{N'}~~~~}\\[4ex]
\node{ds}{D'}  & \node{ap}{AP}    & \multicolumn{2}{c}{\node{ns2}{N'}}\\[4ex]
\node{d}{\dnull}    & \node{as}{A'}    & \node{ap2}{AP}   & \node{ns3}{N'}\\[4ex]
               & \node{a}{\anull}      & \node{as2}{A'}   & \node{n}{\nnull}   \\[4ex]
               &                  & \node{a2}{\anull}     &               \\[4ex]
\node{das}{das} & \node{dicke}{dicke} & \node{alte}{alte} & \node{buch}{Buch}\\
\end{tabular}%
\nodeconnect{np}{dp}\nodeconnect{np}{ns}%
\nodeconnect{dp}{ds}\nodeconnect{ds}{d}%
\nodeconnect{ns}{ap}\nodeconnect{ns}{ns2}%
\nodeconnect{ap}{as}\nodeconnect{as}{a}%
\nodeconnect{ns2}{ap2}\nodeconnect{ns2}{ns3}\nodeconnect{ns3}{n}%
\nodeconnect{ap2}{as2}\nodeconnect{as2}{a2}%
\nodeconnect{d}{das}\nodeconnect{a}{dicke}\nodeconnect{a2}{alte}\nodeconnect{n}{buch}%
}%
}
\end{columns}
\pause

}

\frame{
\frametitle{Nominalphrasen: Genitivattribute (I)}

\begin{itemize}
\item<+-> Nominalphrasen im Genitiv (Genitivattribute): 
verschiedene Typen: 
\begin{itemize}
\item Genitivus possessivus
\item Genitivus subjectivus
\item Genitivus  objectivus 
\end{itemize}
In (\mex{1}) Subjekt- und Objekt-Lesart möglich:
\ea
Die aus Angst um ihre Sicherheit und die ihrer Familie zurückgetretenen Journalisten berichten von \blauit{Einschüchterungen Pekinger Politiker und prochinesischer Kreise} in Hongkong.\footnote{taz, 29.05.2004, S.\,11}
\z
Durch Kontext klar: Subjektlesart
\item<+-> erscheinen im Deutschen an zwei Positionen:\\
          vor dem Nomen (=~pränominal) und danach (=~postnominal). 

\eal
\ex {}[des Kaisers] neue Kleider
\ex {}die neuen Kleider [des Kaisers] 
\zl
\end{itemize}

}

\frame{
\frametitle{Nominalphrasen: Genitivattribute (II)}

\begin{itemize}
\item<+-> pränominaler Genitiv tritt nie zusammen mit Artikelwort auf

\item<+-> pränominaler Genitiv legt übergeordnete NP in Definitheit fest,\\
verhält sich also in dieser Hinsicht wie ein definiter Artikel. 

$\to$ Annahme: solche NPs nehmen ebenfalls die Spezifikatorposition ein.

\item<+-> Umstritten ist, ob dies die "`Originalposition"' der Genitivphrase ist (Genitiv-NP dort basisgeneriert) oder ob sie dorthin bewegt worden ist. 

\bigskip
\item<+-> Postnominale Genitivphrasen sind teils Komplemente (=~valenzbedingt), teils Adjunkte.
\eal
\ex der Vater des Jungen (Komplement)
\ex die Konstruktion einer Vertikalsonnenuhr (Komplement)
\ex der Mantel des Jungen (Adjunkt)
\zl
\end{itemize}

}

\frame{
\frametitle{Nominalphrasen: NP (III)}

Präpositionalphrasen als Attribute von Nomen sind Komplement (\mex{1}) oder Adjunkt (\mex{2}): 
\eal
\ex die Freude [über den Erfolg]
\ex der Anteil [am Erfolg] 
\zl
\eal
\ex die Brücke [über die Weser] 
\ex die Sitzung [am Freitag]
\zl

}

\frame{
\frametitle{Nominalphrasen: DP (I)}

\begin{columns}

\column{95mm}
Alternativer Ansatz (sehr oft in neueren Arbeiten):\\
Nominalphrasen sind in eine funktionale Kategorie DP eingebettet;
sie sind Komplement des Kopfes D.\nocite{Abney87a-u}

\pause
\bigskip
Der Artikel wird entweder mit D identifiziert:

\column{25mm}
{\small
\begin{tabular}{cc}
\multicolumn{2}{c}{\node{dp}{DP}}\\[4ex]
\multicolumn{2}{c}{\node{ds}{D'}}\\[4ex]
\node{d}{\dnull} & \node{np}{NP}\\[4ex]
  & \node{ns}{N'}\\[4ex]
  & \node{n}{\nnull}\\[4ex]
\node{das}{das} & \node{buch}{Buch}\\
\end{tabular}
\nodeconnect{dp}{ds}%
\nodeconnect{ds}{d}\nodeconnect{ds}{np}%
\nodeconnect{np}{ns}\nodeconnect{ns}{n}%
\nodeconnect{d}{das}%
\nodeconnect{n}{buch}%
}
\end{columns}

}

\frame{
\frametitle{Nominalphrasen: DP (II)}

\begin{columns}

\column{85mm}
Oder Artikel wird als Spezifikator der DP analysiert (syntaktische Kategorie: Art):

\column{35mm}
{\small
\begin{tabular}{ccc}
\multicolumn{3}{c}{\node{dp2}{DP}}\\[4ex]
\node{artp}{ArtP} & \multicolumn{2}{c}{\node{ds2}{D'}}\\[4ex]
\node{arts}{Art'} & \node{d2}{\dnull} & \node{np2}{NP}\\[4ex]
\node{art}{Art$^0$}  &  & \node{ns2}{N'}\\[4ex]
     &  & \node{n2}{\nnull}\\[4ex]
\node{das2}{das} & \node{e}{[~~]} & \node{buch2}{Buch}\\
\end{tabular}
\nodeconnect{dp2}{ds2}%
\nodeconnect{dp2}{artp}\nodeconnect{artp}{arts}\nodeconnect{arts}{art}%
\nodeconnect{ds2}{d2}\nodeconnect{ds2}{np2}%
\nodeconnect{np2}{ns2}\nodeconnect{ns2}{n2}%
\nodeconnect{d2}{e}
\nodeconnect{art}{das2}%
\nodeconnect{n2}{buch2}%
}

\end{columns}


}

\subsubsection{Adjektivphrasen}

\frame{
\frametitle{Adjektivphrasen: Komplemente (I)}


Adjektive können interne Argumente (Komplemente) bei sich haben:\\
\begin{itemize}
\item Genitivobjekt\pause
\ea
Er ist [\sub{AP} [\sub{A'}  [\sub{NP} des kalten Wetters] überdrüssig]].
\z
\pause
\item Dativobjekt\pause
\ea
Er ist [\sub{AP} [\sub{A'}  [\sub{NP} seinem Vater] ähnlich]].
\z
\pause
\item Akkusativobjekt\pause
\ea
Er ist [\sub{AP} [\sub{A'}  [\sub{NP} den Lärm] gewohnt]].
\z
\pause
\item Präpositionalobjekt\pause
\ea
Er ist [\sub{AP} [\sub{A'}  [\sub{PP} auf ihre Tochter] stolz]].
\z
\end{itemize}

}

\frame{
\frametitle{Adjektivphrasen: Komplemente (II)}


\begin{itemize}
\item Prädikative
\ea
Sie ist [\sub{AP} [\sub{A'}  [\sub{KonjP} als Geschäftsführerin] tätig]].
\z
\pause
\item Adverbialien
\pause
\ea
Er ist [\sub{AP} [\sub{A'}  [\sub{PP} in Bremen] ansässig]].
\z
\end{itemize}

}

\frame{
\frametitle{Adjektivphrasen: Spezifikatoren}


\begin{itemize}
\item Adverbiale Akkusative und andere Gradausdrücke sind Spezifikatoren:
\ea
Der Sack ist [\sub{AP} [\sub{NP} einen Zentner] [\sub{A'}  [\sub{A} schwer]]]. 
\z

Alternative: eine spezielle funktionale Hülle DegP für Gradausdrücke\\
(DegP =~Degree Phrase, Gradphrase),\\
analog zur DP-Hypothese: Deg$^0$ ist leer und \emph{einen Zentner} steht in SpecDegP.
\end{itemize}

}

\frame{
\frametitle{Adjektivphrasen: Externes Argument}

\begin{itemize}
\item<+-> Adjektive haben wie Verben ein externes Argument.
\item<+-> Wenn die AP als Adjunkt ein Nomen modifiziert (=~attributives Adjektiv), ist der Schwesterknoten N' der AP mit dem externen Argument koreferent. 

\item<+-> Bei prädikativen Adjektiven fungiert meist das Subjekt oder das Akkusativobjekt als externes Argument der AP: 
\eal
\ex {}[\sub{NP} Der Tisch] ist [\sub{AP} sauber]. 
\ex Otto macht [\sub{NP} den Tisch] [\sub{AP} sauber].
\zl

\end{itemize}

}

\frame[label=sc]{
\frametitle{Adjektivphrasen -- Exkurs: Small Clauses (I)}


\begin{description}
\item[Theta-Kriterium]
Jedes Argument bekommt genau eine Theta-Rolle.
\end{description}

Theta-Kriterium $\to$\\
Objekt in (\mex{1}) bekommt nur von der AP eine Theta-Rolle.
\ea
Otto macht [\sub{NP} den Tisch] [\sub{AP} sauber].
\z
\pause
Zwei Möglichkeiten für den Umgang mit (\mex{0})
\begin{enumerate}
\item Wir verwerfen das Theta-Kriterium\\
      und lassen zu, dass es Argumente gibt, die keine Theta-Rolle bekommen. (\zb in LFG, HPSG)
\item 
Generativen Grammatik Chomsky'scher Prägung:\\
Beziehung zwischen dem Verb einerseits und dem Objekt und der prädikativen AP 
andererseits kommt über eine satzartige Zwischenschicht zustande, 
ein sogenannter Small Clause.%(hier relativ theorieneutral als SC (=~Small Clause) klassifiziert) 
\end{enumerate}
}

\frame{
\frametitle{Adjektivphrasen -- Exkurs: Small Clauses (II)}

Zumindest in Fällen wie (\mex{1}a) ist Paraphrase mit finitem Satz möglich (\mex{1}b):

\eal
\ex Otto macht [\sub{SC} [\sub{NP} den Tisch] [\sub{AP} sauber] ].
\ex Otto macht, [dass [\sub{NP} der Tisch] [\sub{AP} sauber] wird].
\zl

\emph{machen} selegiert in (\mex{0}a--b) 
ein Argument mit Theta-Rolle PROPOSITION, das je nachdem als Small Clause oder als finiter Nebensatz 
realisiert wird.

Small Clauses sind nicht unproblematisch (\citew[Kapitel~7.4]{Mueller2002b} und \compare{sc-probleme}{SC-Probleme})\\
und es gibt Alternativen (auch im GB-Framework).


}

\subsubsection{Präpositionalphrasen}

\frame{
\frametitle{Präpositionalphrasen: Komplemente}

Sieht man von als P kategorisierten Adverbien \compare{slide-lex-kat-gb}{Kreuzklassifikation} ab,
haben Präpositionen immer ein Komplement.

\vfill

\centerline{\begin{tabular}{cc}
\multicolumn{2}{c}{\node{dp}{PP}~~~~}\\[4ex]
\multicolumn{2}{c}{\node{ds}{P'}~~~~}\\[4ex]
\node{d}{P} & \node{np}{NP}\\[6ex]
\node{mit}{mit} & de\node{dk}{n Kinde}rn\\
\end{tabular}%
\nodeconnect{dp}{ds}%
\nodeconnect{ds}{d}\nodeconnect{ds}{np}%
\nodetriangle{np}{dk}%
\nodeconnect{d}{mit}%
%
\hspace{4cm}%
\begin{tabular}{cc}
\multicolumn{2}{c}{\node{pp}{PP}~~~~}\\[4ex]
\multicolumn{2}{c}{\node{ps}{P'}~~~~}\\[4ex]
\node{np2}{NP} & \node{p}{P}\\[6ex]
de\node{dk2}{n Kinde}rn & \node{zuliebe}{zuliebe}\\
\end{tabular}%
\nodeconnect{pp}{ps}%
\nodeconnect{ps}{p}\nodeconnect{ps}{np2}%
\nodetriangle{np2}{dk2}%
\nodeconnect{p}{zuliebe}%
}

\vfill

Manchmal unterteilt man in Prä- und Postpositionen,\\
manchmal faßt man alles unter Präposition zusammen.

}

\frame{
\frametitle{Spezifikatoren der PP und Verwendeung von PPen}

\begin{itemize}
\item Als Spezifikatoren lassen sich Gradausdrücke auffassen,\\
\zb NPs im Akkusativ (=~adverbialer Akkusativ) oder Adverbphrasen: 

\eal
\ex {}[\sub{PP} [\sub{NP} einen Tag] [\sub{P'}  vor [\sub{NP} der Abreise]]]
\ex {}[\sub{PP} [\sub{AdvP} kurz] [\sub{P'}  vor [\sub{NP} der Abreise]]] 
\zl
\pause
\item Präpositionalphrasen, die von Verben und Adjektiven abhängen,\\
können als Objekte, Prädikative oder Adverbialien fungieren;\\
in diesen Funktionen können sie vom Verb oder Adjektiv verlangte Komplemente sein 
oder als Adjunkte die Verbalphrase bzw.\ den Satz modifizieren.
\end{itemize}

}

}%\end{gb-intro}


%%%%%%%%%%%%%%%%%%%%%%%%%%%%%%%%%%%%%%%%%%%%%%%%%%%%%%%%%
%%   $RCSfile: 3-gengram-grundfragen.tex,v $
%%  $Revision: 1.1 $
%%      $Date: 2004/04/26 14:10:21 $
%%    Authors: Stefan Mueller (CL Uni Bremen)
%%    Purpose: course slides
%%   Language: LaTeX
%%%%%%%%%%%%%%%%%%%%%%%%%%%%%%%%%%%%%%%%%%%%%%%%%%%%%%%%%

\subsection{Die Struktur des deutschen Satzes}

%\if 0

\iftoggle{gb-intro}{

\frame{

\frametitle{Gliederung}

\begin{itemize}
%% \item Ziele
%% \item Wiederholung von Grundbegriffen
%% \item Grundfragen der Sprachwissenschaft
\item Grammatikmodelle
\begin{itemize}
\item Phrasenstrukturgrammatik
\item Transformationsgrammatik und deren Nachfolger
\begin{itemize}
\item Geschichtliches und Motivation
\item Das T-Modell im Überblick
\item Grundbegriffe: Theta-Rollen, Externes Argument, \ldots
\item Lexikoneinträge
\item Syntaktische Kategorien
\item \xbar-Schemata
\item \blaubf{Die Struktur des deutschen Satzes}
\item Kasus
\item NP-Bewegung
\item Bindungstheorie
%\item \emph{w}-Bewegung
\item Inkorporation
\end{itemize}
\end{itemize}
\end{itemize}

}
}%\end{gb-intro}

\subsubsection{Exkurs: CP und IP im Englischen}

\frame[shrink]{
\frametitle{Die Struktur des deutschen Satzes}


\begin{itemize}
\item In früheren Arbeiten zum Englischen gab es für Sätze Regeln wie: 
\eal
\ex S $\to$ NP VP
\ex S $\to$ NP Infl VP
\zl

\begin{forest}
sm edges
[S
  [NP
  	[Ann,roof]]
  [INFL
  	[will]]
  [VP
  	[V$'$
		[V$^0$[read]]
		[NP[the newspaper, roof]]]]]
\end{forest}

\pause
\item Diese Regeln entsprechen nicht dem \xbar-Schema.


\end{itemize}

}



\subsubsubsection{C und I}

% \frame{
% \frametitle{C und I}

% \begin{itemize}
% \item Anmerkung: Die Benennung der Kategorien C und I ist teilweise nur noch wissenschaftsgeschichtlich motivierbar.
% \begin{itemize}
% \item<+-> Älterer Terminus für C$^0$: COMP = englisch \emph{complementizer},\\
% unterordnende Konjunktion, wie sie prototypisch in Objektnebensätzen (\emph{complement clauses}) auf"|tritt.
% \item<+-> Neuerer Terminus für I$^0$: T = Tempus (englisch \emph{tense}). 
% \item<+-> Älterer Terminus für I$^0$: INFL = englisch \emph{inflection};\\
% gemeint: Flexion des Verbs (Konjugation); \\
% Vorgänger: AUX = englisch \emph{auxiliary} = Hilfsverb.
% \end{itemize}
% \ifthenelse{\boolean{gb-intro}}{
% \item<+-> Bevor wir uns CP und IP im Deutschen zuwenden\\
% betrachten wir das Englische.
% }{}
% \end{itemize}


% }


\subsubsubsection{IP und VP im Englischen}

\frame{
\frametitle{Exkurs: IP und VP im Englischen: Hilfsverben}

\savespace\small\parskip0pt


\centerline{
\begin{forest}
sm edges
[IP
  [NP
  	[Ann,roof]]
  [I$'$
  	[I$^0$
  		[will]]
	[VP
  	[V$'$
		[V$^0$[read]]
		[NP[the newspaper, roof]]]]]]
\end{forest}}


\begin{itemize}
\item Statt dessen INFL als Kopf, der eine VP als Komplement nimmt.
\item Hilfsverben stehen in \inull (=~Aux).
\item Satzadverbien können zwischen Hilfsverb und Vollverb stehen.
\end{itemize}

}


\frame{
\frametitle{IP und VP im Englischen: Sätze ohne Hilfsverb}

\savespace\small\parskip0pt

\centerline{%
\begin{forest}
sm edges
[IP
	[NP[Ann,roof]]
	[I$'$
		[I$^0$[-s]]
		[VP
			[V$'$
				[V$^0$[read-]]
				[NP[the newspaper, roof]]]]]]
\end{forest}
}

\begin{itemize}
\item Hilfsverben stehen in \inull (=~Aux).
\item Position kann leer bleiben.\\
Wird dann mit der flektierten Form des finiten Verbs verknüpft.

Früher: In \inull stand das Affix \suffix{s}, das Verb bewegt sich in die \inull-Position.
\end{itemize}

}


\frame{
\frametitle{Englische Sätze mit Komplementierer}

\psset{xunit=1cm,yunit=5.4mm}
\psset{nodesep=4pt,linewidth=0.8pt,arrowscale=2}


\hfill\scalebox{0.73}{%
\begin{forest}
sm edges
[CP
[C$'$
	[C$^0$[that]]
	[IP
		[NP[Ann,roof]]
		[I$'$
			[I$^0$[will]]
			[VP
				[V$'$
					[V$^0$[read]]
					[NP[the newspaper, roof]]]]]]]]
\end{forest}}\hfill\hfill\mbox{}

\begin{itemize}
\item Der Komplementierer (\emph{that}, \emph{because}, \ldots) verlangt eine IP.
\end{itemize}

}

\subsubsubsection{CP, IP und VP im Englischen}

\frame[shrink]{
%\frame{
\frametitle{CP, IP und VP im Englischen: Fragesätze}

\savespace\small\smallexamples\parskip0pt\itemsep0pt


\centerline{%
\scalebox{.8}{
\begin{forest}
sm edges
[IP
		[NP[Ann,roof]]
		[I$'$
			[I$^0$[\trace$_k$]]
			[VP
				[V$'$
					[V$^0$[read]]
					[NP[the newspaper,roof]]]]]]
\end{forest}}}

\begin{itemize}
\item Ja/nein-Fragen werden durch Voranstellung des Hilfsverbs gebildet:
\ea
Will Ann read the newspaper?
\z
\pause
\item \emph{wh}-Fragen werden durch zusätzliche Voranstellung vor das Hilfsverb gebildet:
\ea
What will Ann read?
\z
\item Umstellung des Hilfsverbs erfolgt in Position, die sonst der Komplementierer innehat.
\end{itemize}

}


\frame[shrink]{
\frametitle{CP, IP und VP im Englischen: Fragesätze}

\vfill

\hfill
\scalebox{.9}{
\begin{forest}
sm edges
[CP
[XP [\trace]]
[C$'$
	[C$^0$[will$_k$]]
	[IP
		[NP[Ann,roof]]
		[I$'$
			[I$^0$[\trace$_k$]]
			[VP
				[V$'$
					[V$^0$[read]]
					[NP[the newspaper,roof]]]]]]]]
\end{forest}}
\hfill
\only<2->{\scalebox{.9}{
\begin{forest}
sm edges
[CP
[NP[what$_i$]]
[C$'$
	[C$^0$[will$_k$]]
	[IP
		[NP[Ann,roof]]
		[I$'$
			[I$^0$[\trace$_k$]]
			[VP
				[V$'$
					[V$^0$[read]]
					[NP[\trace$_i$]]]]]]]]
\end{forest}}}
\hfill\mbox{}

\vfill

\pause

}



\iftoggle{gb-intro}{
\frame{
\frametitle{Grundannahmen zu CP, IP und VP im Englischen}

\begin{itemize}
\item Die \blaubf{D-Struktur} ist die Phrasenstruktur,\\
die sich aus den \blaubf{Theta-Rastern} der beteiligten \blaubf{lexikalischen Einheiten} ergibt.
\item Die \blaubf{S-Struktur} berücksichtigt
zusätzlich die Anforderungen der \blaubf{funktionalen Kategorien}.

Besonders wichtig:\\
Funktionale Kategorien können \blaubf{Bewegungen} auslösen. 
\end{itemize}

}



\subsubsubsection{Strukturelle Regeln für die Satzstruktur des Englischen}


\frame{
\frametitle{Strukturelle Regeln für die Satzstruktur des Englischen (I)}


\begin{itemize}
\item Vollverben stehen immer an Position \vnull.
\pause
\item Satzadverbialien stehen am linken Rand der VP (Einzelheiten umstritten).
\pause
\item In einfachen affirmativen (nicht verneinten) Sätzen mit einer
einzigen Verbform ist \inull lexikalisch leer. 

\pause
\item Für Kongruenz zwischen Subjekt und Verb ist die Kategorie I zuständig 
(sogenanntes Spec-Head-Agreement, hier Kongruenz zwischen der Subjektposition
SpecIP und dem Kopf \inull) $\to$ 
muss eine "`unsichtbare"' Beziehung zwischen \inull und \vnull geben.

\pause 
Heute geht man von Verkettung von \vnull mit der leeren \inull-Position aus. 

\pause
Früher: "`Affix-Hopping"':\nocite{Chomsky75a} \inull war nur
mit grammatischen Merkmalen (insbesondere Tempus und Modus sowie Person und
Numerus) und allenfalls mit einem Affix (\zb \emph{-s}, \emph{-ed}) besetzt.
Affix musste aus phonologischen Gründen zum Verb "`hinunterhüpfen"'.


\end{itemize}

}

\frame{
\frametitle{Strukturelle Regeln für die Satzstruktur des Englischen (II)}

\begin{itemize}[<+->]
\item Hilfsverben (Auxiliare) \emph{be}, \emph{have}, \emph{will}, \emph{shall}, \emph{can}, \emph{may}, \emph{must}\\
haben syntaktische Kategorie I.

\item In verneinten Sätzen tritt das semantisch leere Hilfsverb \emph{do} auf (Kategorie: auch I). 
\item Die CP-Schicht wird nur bei bestimmten Satzarten gefüllt, und zwar mit \blaubf{Bewegung} von
Elementen aus IP und VP. 

Dabei gilt das universelle \blaubf{Prinzip der
Strukturerhaltung}: An eine Phrasenposition (zum Beispiel Spec-Position) kann
nur eine Phrase bewegt werden, an eine Kopfposition nur ein Kopf.

\end{itemize}

}

\frame{
\frametitle{Regeln: Interrogativ-Hauptsätze (I)}

%\setlength\leftmargini{1em}
In Interrogativ-Hauptsätzen finden die folgenden Bewegungen statt: 
\begin{itemize}
\item Interrogative XP $\to$ SpecCP 
\item \inull $\to$ \cnull
\end{itemize}


{\footnotesize

\begin{tabular}{|l|l|l|l|l|l|l|}\hline
SpecCP    & \cnull     & SpecIP & \inull   & Satzadv       & \vnull    & Rest der VP\\\hline\hline
What$_i$  & will$_k$   & Ann    & t$_k$   & usually       & read      & t$_i$ in the pub?\\
What$_i$  & does$_k$   & Ann    & t$_k$   & usually       & read      & t$_i$ in the pub?\\\hline
\end{tabular}

}
}

\frame{
\frametitle{Regeln: Interrogativ-Hauptsätze (II)}

Bei Entscheidungsfragen sowie Frage nach dem Subjekt\\
keine Bewegung nach SpecCP:


{\footnotesize
\begin{tabular}{|l|l|l|l|l|l|l|}\hline
SpecCP    & \cnull     & SpecIP & \inull   & Satzadv       & \vnull    & Rest der VP\\\hline\hline
          & Will$_k$   & Ann    & \_$_k$   & usually       & read      & the newspaper in the pub?\\
          & Did$_k$    & Ann    & \_$_k$   & usually       & read      & the newspaper in the pub?\\
          & Didn't$_k$ & Ann    & \_$_k$   & usually       & read      & the newspaper in the pub?\\
          &            & Who    & read$_k$ s  & usually       & \_$_k$ & the newspaper in the pub?\\
          &            & Who    & will     & usually       & read      & the newspaper in the pub?\\\hline
\end{tabular}
}


\pause
\inull $\to$ \cnull (außer bei Frage nach dem Subjekt).\\
Bewegung von leerem \inull ist ausgeschlossen;\\
stattdessen wird in \cnull das semantisch leere Hilfsverb \emph{do} eingeführt. 

}

\frame{
\frametitle{Regeln: Interrogativnebensätze}


\begin{itemize}
\item<+-> In \emph{wh}-Interrogativnebensätzen und in \emph{wh}-Relativsätzen wird nur SpecCP
besetzt, und zwar mit einer Phrase, die ein \emph{wh}-Wort (\emph{what}, \emph{who}, \emph{which}, \emph{where}, \ldots)
enthält. 
\eal
\ex I wonder whether he will come.
\ex the man who you saw
\zl
Man spricht hier von \emph{wh}-Bewegung.
\item<+-> (\emph{wh}-Interrogativnebensätze sind nicht die einzige Option für Relativsätze im Englischen!)
\end{itemize}

}

\frame{
\frametitle{Regeln: Konjunktionalnebensätze}


\begin{itemize}

\item<+-> Konjunktionalnebensätze: unterordnende Konjunktion (Complementizer) steht in \cnull. 
\ea
He believes that he will win.
\z
\item<+-> In älteren Sprachstufen und dialektal noch heute wird \cnull auch bei
\emph{wh}-Interrogativ-Nebensätzen mit der unterordnenden Konjunktion \emph{that} besetzt
(vgl.\ die analoge Erscheinung im Frühneuhochdeutschen und in deutschen Dialekten):
\ea
Men shal knowe what that I am \citep[S.\,122]{Haegeman94a-u} 
\z
\end{itemize}

}

\subsubsubsection{Zwei Arten von Bewegung}
\frame{
\frametitle{Zwei Arten von Bewegung}

Fazit: Es treten zwei Arten von Bewegungen auf:
\begin{itemize}
\item Bewegung von Phrasen an Spezifikatorpositionen und 
\item Bewegung von Köpfen an die jeweils unmittelbar übergeordnete Kopfposition
\end{itemize}


}

\frame{
\frametitle{\normalsize Übersicht: CP, IP und VP im Englischen}
%
\savespace
{\scriptsize
\hspace{\leftmargin}\begin{tabular}{|l|l|l|l|l|l|l|}\hline
SpecCP    & \cnull     & SpecIP & \inull   & Satzadv       & \vnull    & Rest der VP\\\hline\hline
          &            & Ann    & read$_k$ s   & usually      & \_$_k$ & the newspaper in the pub.\\
          &            & Ann    & will     & usually       & read      & the newspaper in the pub.\\
          &            & Ann    & does not & usually       & read      & the newspaper in the pub.\\ 
          & that       & Ann    & read$_k$ s  & usually       & \_$_k$ & the newspaper in the pub.\\
          & Will$_k$   & Ann    & t$_k$   & usually       & read      & the newspaper in the pub?\\
          & Did$_k$    & Ann    & t$_k$   & usually       & read      & the newspaper in the pub?\\
          & Didn't$_k$ & Ann    & t$_k$   & usually       & read      & the newspaper in the pub?\\
          &            & Who    & read$_k$ s  & usually       & \_$_k$ & the newspaper in the pub?\\
          &            & Who    & will     & usually       & read      & the newspaper in the pub?\\
What$_i$  & will$_k$   & Ann    & t$_k$   & usually       & read      & t$_i$ in the pub?\\
What$_i$  & does$_k$   & Ann    & t$_k$   & usually       & read      & t$_i$ in the pub?\\\hline\hline
%
what$_i$  & [ \_ ]     & Ann    & read$_k$ s   & usually       & \_$_k$      & t$_i$ in the pub.\\
what$_i$  & [ \_ ]     & Ann    & will     & usually       & read      & t$_i$ in the pub.\\
which$_i$ & [ \_ ]     & Ann    & read$_k$ s  & usually       & \_$_k$ & t$_i$ in the pub.\\
which$_i$ & [ \_ ]     & Ann    & will     & usually       & read      & t$_i$ in the pub.\\\hline
\end{tabular}

\begin{itemize}\itemsep0pt
\item Leere Köpfe, sofern von Anfang an leer: \_ oder e (englisch empty = leer)
\item Wegbewegte Köpfe und Phrasen = t (englisch \emph{trace} = Spur)
\item Die Indexe (i, j, k, \ldots) machen klar, welche Phrase zu welcher leeren Position gehört.
\end{itemize}
}

}

}%\end{gb-intro}

\subsubsection{Topologie des deutschen Satzes}

\frame{
\frametitle{Topologie des deutschen Satzes (I)}

Bevor wir uns dem CP/IP-System für das Deutsche zuwenden, müssen einige deskriptive Begriffe geklärt
werden:

\begin{itemize}
\item Die Abfolge der Konstituenten im Deutschen wird unter Bezugnahme auf topologische
Felder erklärt.

\pause
\item Wichtige Arbeiten zum Thema topologische Felder sind:\\
\citew{Drach37},  \citew{Reis80a} und \citew{Hoehle86}.

\pause
\item Im folgenden werden die Begriffe \emph{Vorfeld}, \emph{linke/rechte Satzklammer},
\emph{Mittelfeld} und \emph{Nachfeld} eingeführt.

\citew{Bech55a} hat noch weitere Felder für die Beschreibung der Abfolgen innerhalb von Verbalkomplexen
eingeführt, die hier aber vorerst ignoriert werden.
\end{itemize}


}

\frame{
\frametitle{Verbstellungstypen und Begriffe}

\savespace
\begin{itemize}
\item Verbendstellung
      \ea
      Peter hat erzählt, \rot<5->{dass} er das Eis \braun<5->{\gruen<4>{gegessen} \blauit<-4>{hat}}.
      \z
\pause
\item Verberststellung
        \ea
      \rot<4->{\blauit<-3>{Hat}} Peter das Eis \braun<5->{\gruen<4>{gegessen}}?
      \z
\pause
\item Verbzweitstellung
      \ea
      Peter \rot<4->{\blauit<-3>{hat}} das Eis \braun<5->{\gruen<4>{gegessen}}.
      \z
\end{itemize}

\pause
\begin{itemize}[<+->]
\item verbale Elemente nur in (\mex{-2}) kontinuierlich
\item \rot<5->{linke} und \braun<5->{rechte} Satzklammer
\item Komplementierer (\emph{weil}, \emph{dass}, \emph{ob}) in der linken Satzklammer
\item Komplementierer und finites Verb sind komplementär verteilt
\item Bereiche vor, zwischen u.\ nach Klammern: Vorfeld, Mittelfeld, Nachfeld
\end{itemize}


}

\frame{
\frametitle{Topologie des deutschen Satzes im Überblick}


\resizebox{\textwidth}{!}{
\begin{tabular}{@{}lllll@{}}
\rowcolor{structure!15}Vorfeld & linke Klammer & Mittelfeld                             & rechte Klammer & Nachfeld                   \\ 
\\
\rowcolor{structure!10}Karl    & schläft.      &                                        &                &                            \\
                       Karl    & hat           &                                        & geschlafen.    &                            \\
\rowcolor{structure!10}Karl    & erkennt       & Maria.                                 &                &                            \\
                       Karl    & färbt         & den Mantel                             & um             & den Maria kennt.           \\
\rowcolor{structure!10}Karl    & hat           & Maria                                  & erkannt.       &                             \\
                       Karl    & hat           & Maria als sie aus dem Zug stieg sofort & erkannt.       &                             \\
\rowcolor{structure!10}Karl    & hat           & Maria sofort                           & erkannt        & als sie aus dem Zug stieg. \\
                       Karl    & hat           & Maria zu erkennen                      & behauptet.     &            \\
\rowcolor{structure!10}Karl    & hat           &                                        & behauptet      & Maria zu erkennen.         \\ 
\\
\rowcolor{structure!10}        & Schläft       & Karl?                                  &                &                            \\
                               & Schlaf!       &                                        &                &                             \\
\rowcolor{structure!10}        & Iß            & jetzt dein Eis                         & auf!           &                             \\
        & Hat           & er doch das ganze Eis alleine          & gegessen.            &      \\  \\
\rowcolor{structure!10}        & weil          & er das ganze Eis alleine               & gegessen hat   & ohne sich zu schämen.\\
        & weil          & er das ganze Eis alleine               & essen können will    & ohne gestört zu werden.    \\
\rowcolor{structure!10}wer     &               & das ganze Eis alleine                  & gegessen hat.  &                             \\
\end{tabular}
}

}

% \frame{
% \frametitle{Der Prädikatskomplex}
% %


% \begin{itemize}
% \item<+-> mehrere Verben in der rechten Satzklammer: Verbalkomplex
% \item<+-> manchmal wird auch von diskontinuierlichen Verbalkomplexen
%       gesprochen (Initialstellung das Finitums)
% \item<+-> auch prädikative Adjektive (\mex{1}a) und Resultativprädikate (\mex{1}b) werden zum Prädikatskomplex gezählt:
%       \eal
%       \ex dass Karl seiner Frau treu ist
%       \ex dass Karl das Glas leer trinkt
%       \zl
% \end{itemize}

% }

\frame{
\frametitle{Die Rangprobe}
%


\begin{itemize}
\item<+-> Felder nicht immer besetzt
      \ea
      \field{Der Mann}{VF} \field{gibt}{LS} \field{der Frau das Buch,}{MF} \field{die er kennt}{NF}.
      \z
\item<+-> Test: Rangprobe \citep[S.\,72]{Bech55a}
\eal
\ex[]{
Der Mann hat der Frau das Buch gegeben, die er kennt.
}
\ex[*]{
Der Mann hat der Frau das Buch, die er kennt, gegeben.
}
\zl
Ersetzung des Finitums durch ein Hilfsverb $\to$\\
Hauptverb besetzt die rechte Satzklammer.
\end{itemize}

}


\frame{
\frametitle{Rekursives Auf"|tauchen der Felder}

\begin{itemize}
\item \citet*[S.\,82]{Reis80a}: Rekursion\\
      Vorfeld kann in Felder unterteilt sein:
\eal
\ex Die Möglichkeit, etwas zu verändern, ist damit verschüttet
      für lange lange Zeit.
\ex {}[Verschüttet für lange lange Zeit] ist damit die Möglichkeit, 
      etwas zu ver"-ändern.
\ex Wir haben schon seit langem gewu"st, da"s du kommst.
\ex {}[Gewu"st, da"s du kommst,] haben wir schon seit langem.
\zl
\pause
\item im Mittelfeld beobachtbare Permutationen auch innerhalb komplexer Vorfelder

\eal
\ex {}[\gruen{Seiner Tochter} \blau{ein Märchen} erzählen] wird er wohl müssen.
\ex {}[\blau{Ein Märchen} \gruen{seiner Tochter} erzählen] wird er wohl müssen.
\zl
\end{itemize}

}

\iftoggle{gb-intro}{
\frame{
\frametitle{Bemerkung zu Interrogativphrasen und Relativpronomina}

\begin{itemize}
\item Zuordnung zum Vorfeld dadurch motiviert, dass im Bairischen
      zusätzlich zur w/d-Phrase ein Komplementierer auftreten kann.
\item außerdem theorieinterne Gründe (Phrasenposition vs.\ Kopfposition)
\item Diese Zuordnung schafft allerdings empirische Probleme:
      \begin{itemize}
      \item Komplementierer kann mit w-Phrase koordiniert werden \citep{Reis85} %S. 301
      \item w/d-Phrasen verhalten sich in bezug auf Verum-Fokus genauso wie Komplementierer
      \item w/d-Phrasen können in Dialekten wie Komplementierer flektiert werden
      \end{itemize}
\end{itemize}


}
}

\frame{
\frametitle{Übung}

Bestimmen Sie die topologischen Felder in den Sätzen in (\mex{1}):
\eal
\ex Der Mann hat gewonnen, den alle kennen.
\ex Er gibt ihm das Buch, das Klaus empfohlen hat.
\ex Maria hat behauptet, dass das nicht stimmt.
\ex Peter hat das Buch gelesen,\\das Maria dem Schüler empfohlen hat,\\der neu in die Klasse gekommen ist.
\ex Komm!
\zl

}

\subsubsection{CP und IP im Deutschen}

\frame{
\frametitle{Das topologische Modell mit CP, IP, VP (I)}

\vfill
\centerline{
\scalebox{.7}{
\begin{forest}
    sn edges original,empty nodes
    [CP
      [{}
        [XP,terminus
          [SpecCP\\prefield, name=p1
          ]
        ]
      ]
      [C$'$
            [{}
              [C$^0$, terminus
                [C$^0$\\left SB, name=c0
                ]
              ]
            ]
            [IP
              [{}
                [XP, terminus
                  [{IP (without I$^0$, V$^0$)\\middle field}
                    [SpecIP\\subject position, set me left, name=specip
                    ]
                    [phrases inside\\the VP, name=p3
                    ]
                  ]
                ]
              ]
              [I$'$
                      [VP, name=vp
                        [V$^0$, name=v0, terminus, no path, anchor=east
                          [{V$^0$, I$^0$\\right SB}, name=p2, set me left
                          ]
                        ]
                      ]
                      [{}
                            [I$^0$, terminus, name=io
                            ]
                      ]
              ]
            ]
      ]
    ]
    \draw [thick]
      (p1.north west) rectangle (io.east |- p3.south);
    \draw
      ($(c0.north east)!1/2!(specip.west |- c0.north east)$) coordinate (p6) -- (p6 |- p3.south)
      ($(p1.north east)!1/2!(c0.north west)$) coordinate (p4) -- (p3.south -| p4)
      ($(specip.north east)!1/2!(p3.north west)$) coordinate (p5) -- (p3.south -| p5)
      ($(p2.north west)!1/2!(p2.north west -| p3.east)$) coordinate (p7) -- (p3.south -| p7)
      (p6 |- p2.south) -- (p2.south -| p7)
      (vp.south) -- (v0.center -| p3.west) -- (v0.west)
      (v0.east) -- +(4.5pt,0) -- (vp.south)
      ;
\end{forest}}}

}

% \frame{
% \frametitle{Das topologische Modell mit CP, IP, VP (II)}

% \footnotesize
% \resizebox{0.99\textwidth}{!}{
% \begin{tabular}{|l|l|l|l|l|}
% \hline
% %
% SpecCP    & \cnull      & \mc{2}{l|}{IP (ohne \inull, \vnull)} & \vnull, \inull\\
% Vorfeld   & Linke       & \mc{2}{l|}{Mittelfeld}                      & Rechte\\\cline{3-4}
%           & Satzklammer & SpecIP           & Phrasen innerhalb der VP & Satzklammer\\
%           &             & Subjektsposition &                          &\\\hline\hline
%           & dass         & Anna & [das Buch] [auf den Tisch] & legt$_k$ [ \_ ]$_k$\\
% \pause
%           & ob  & Anna & [das Buch] [auf den Tisch] & legt$_k$ [ \_ ]$_k$ \\\hline\hline
% \pause
% \ifthenelse{\boolean{gb-intro}}{
% wer$_i$      & [ \_ ] & [ t ]$_i$ & [das Buch] [auf den Tisch] & legt$_k$ [ \_ ]$_k$\\
% \pause
% was$_i$      & [ \_ ] & Anna & [ t ]$_i$ [auf den Tisch] & legt$_k$ [ \_ ]$_k$\\\hline\hline
% \pause
% }{}
%           & Legt$_k$ & Anna & [das Buch] [auf den Tisch]? & [ t ]$_k$ [ t ]$_k$\\
% \pause
%           & Legt$_k$ & Anna & [das Buch] [auf den Tisch], & [ t ]$_k$ [ t ]$_k$ \\\hline\hline
% \pause
% Anna$_i$     & legt$_k$ & [ t ]$_i$ & [das Buch] [auf den Tisch] & [ t ]$_k$ [ t ]$_k$\\
% \pause
% Wer$_i$      & legt$_k$ & [ t ]$_i$ & [das Buch] [auf den Tisch]? & [ t ]$_k$ [ t ]$_k$\\
% \pause
% {}[Das Buch]$_i$ & legt$_k$ & Anna & [ t ]$_i$ [auf den Tisch] & [ t ]$_k$ [ t ]$_k$\\
% \pause
% Was$_i$      & legt$_k$ & Anna & [ t ]$_i$ [auf den Tisch]? & [ t ]$_k$ [ t ]$_k$\\
% \pause
% {}[Auf den Tisch]$_i$ & legt$_k$ &Anna & [das Buch] [ t ]$_i$ & [ t ]$_k$ [ t ]$_k$\\\hline
% \end{tabular}
% }

% \pause
% \vfill
% Achtung: Die Bezeichner SpecCP u.\ SpecIP sind keine Kategoriensymbole. Sie kommen
% in Grammatiken mit Ersetzungsregeln nicht vor! Sie bezeichnen nur Positionen im Baum.

% }

\subsubsubsection{Deutsch als SOV-Sprache}

\frame{
\frametitle{Deutsch als SOV-Sprache}

\begin{itemize}

\item Köpfe von VP und IP (\vnull und \inull) stehen im Deutschen rechts und bilden
zusammen die rechte Satzklammer. 

\pause
\item Subjekt und alle anderen Satzglieder (Komplemente und Adjunkte) stehen links davon und bilden das
Mittelfeld. 

\pause
\item Deutsch ist damit zumindest in der D-Struktur eine sogenannte
SOV-Sprache (=~Sprache mit Grundabfolge Subjekt--Objekt-- Verb)

\begin{itemize}
\item SOV Deutsch, \ldots
\item SVO Englisch, Französisch, \ldots
\item VSO Walisisch, Arabisch, \ldots 
\end{itemize}
Etwa 40\,\% aller Sprachen sind SOV-Sprachen, etwa 25\,\% sind SVO. 


\pause
\item Nebeneffekt der SOV-Struktur: Je enger sich ein Satzglied auf das
Verb bezieht, desto näher steht es an der rechten Satzklammer und auch dann,
wenn das Verb wegbewegt wurde.
\end{itemize}

}

\frame{
\frametitlefit{Motivation der Verbletztstellung als Grundstellung: Partikeln}

\citew%[S.\,34--36]
{Bierwisch63a}: Sogenannte Verbzusätze oder Verbpartikel\\
bilden mit dem Verb eine enge Einheit.
\eal
\ex weil er morgen \blau{anfängt}
\ex Er \blau{fängt} morgen \blau{an}.
\zl
Diese Einheit ist nur in der Verbletzstellung zu sehen, was dafür spricht, diese
Stellung als Grundstellung anzusehen.
}

% \frame{
% \frametitle{Stellung der infiniten Verben}


% \eal
% \ex Dieses Buch sollte gelesen werden müssen.
% \ex This book should have been read.
% \zl

% }

\frame{
\frametitle{Stellung in Nebensätzen}

Verben in infiniten Nebensätzen und in durch eine Konjunktion eingeleiteten
finiten Nebensätzen stehen immer am Ende\\
(von Ausklammerungen ins Nachfeld abgesehen):
\eal
\ex Der Clown versucht, Kurt-Martin die Ware \blaubf{zu geben}.
\ex dass der Clown Kurt-Martin die Ware \blaubf{gibt}
\zl
}

\frame{
\frametitle{Stellung der Verben in SVO und SOV-Sprachen}

\citet{Oersnes2009b}: 
\eal
\ex dass er ihn gesehen$_3$ haben$_2$ muss$_1$
\ex 
\gll at han må$_1$ have$_2$ set$_3$ ham\\
     dass er muss haben gesehen ihn\\
\zl


}

\frame{
\frametitle{Skopus}

\citew[Abschnitt~2.3]{Netter92}:
Skopusbeziehungen der Adverbien hängt von ihrerer Reihenfolge ab (Präferenzregel?):\\
Links stehendes Adverb hat Skopus über folgendes Adverb und Verb.

\eal
\ex weil er [absichtlich [nicht lacht]]
\ex weil er [nicht [absichtlich lacht]]
\zl
\pause
Bei Verberststellung ändern sich die Skopusverhältnisse nicht.
\eal
\ex Er lacht absichtlich nicht.
\ex Er lacht nicht absichtlich.
\zl
}


\subsubsubsection{C -- die linke Satzklammer}


\frame{
\frametitle{\cnull{} -- die linke Satzklammer in Nebensätzen}

\cnull entspricht der linken Satzklammer und wird wie folgt besetzt:
\begin{itemize}
\item In Konjunktionalnebensätzen steht die unterordnende Konjunktion\\ (der Complementizer) wie im Englischen in \cnull. 

Das Verb bleibt in der rechten Satzklammer.
\ea
Er glaubt, dass sie kommt.
\z
\iftoggle{gb-intro}{
\pause
\item In Relativ- und \emph{w}-Interrogativnebensätzen bleibt die Position \cnull standardsprachlich leer. 
(In früheren und regionalen Varietäten des Deutschen konnte/kann \cnull besetzt werden.)

Das Verb bleibt ebenfalls in der rechten Satzklammer. 
\eal
\ex der Mann, den wir kennen
\ex Ich frage mich, wen du überhaupt kennst.
\zl
}
\end{itemize}

}
\frame{
\frametitle{\cnull{} -- die linke Satzklammer in Verberst- und -zweitsätzen}

\begin{itemize}
\item In Verberst- und Verbzweitsätzen wird das finite Verb über die Position \inull nach \cnull bewegt: 
\vnull $\to$  \inull $\to$ \cnull. 
\eal
\ex dass er sie kenn- -t \jambox{(Verb in \vnull)}
\ex dass er sie \_$_i$ [kenn-$_i$ -t] \jambox{(Verb in \inull)}
\ex {}[Kenn-$_i$ -t]$_j$ er sie \_$_i$ \_$_j$? \jambox{(Verb in \cnull)}
\zl

\iftoggle{gb-intro}{
\pause
\item Direkte Bewegung von \vnull nach \cnull wird durch eine im Sprachvergleich gut abgesicherte,
hier aber nicht direkt nachweisbare Beschränkung ausgeschlossen, den sogenannten \emph{Head Movement Constraint}\nocite{Travis84a-u}.
\item Für das Auftreten von Spuren gibt es allgemeine Beschränkungen,\\
      die denen der \siehe \hyperlink{bt}{Pronomenbindung} entsprechen.\\
      Das \emph{Head Movement Constraint} kann aus diesen Beschränkungen abgeleitet werden \citep{Baker88a}.
}
\end{itemize}

}

\subsubsubsection{SpecCP - das Vorfeld}

\frame{
\frametitle{SpecCP -- das Vorfeld in Deklarativsätzen (I)}

Die Position SpecCP entspricht dem Vorfeld und wird wie folgt besetzt:
\begin{itemize}
\item Deklarativsätze (Aussage-Hauptsätze): XP wird ins Vorfeld bewegt.
\ea
Gibt der Mann dem Kind jetzt den Mantel?
\z
\pause
\eal
\ex Der Mann gibt dem Kind jetzt den Mantel.
\pause
\ex Dem Kind gibt der Mann jetzt den Mantel.
\pause
\ex Den Mantel gibt der Mann dem Kind jetzt.
\pause
\ex Jetzt gibt der Mann dem Kind den Mantel.
\zl
\end{itemize}

}


\frame{

\frametitle{Verbbewegung und Bewegung nach SpecCP}

\vfill
\centerline{\scalebox{0.8}{
\begin{forest}
sm edges
[CP
[NP [diesen Mann$_i$, roof]]
[C$'$
	[C$^0$[(kenn-$_j$ -t)$_k$]]
	[IP
		[NP [jeder,roof]]
		[I$'$
			[VP
				[V$'$
					[NP[\trace$_i$]]
					[V$^0$[\trace$_j$]]]]
			[I$^0$ [\trace$_k$]]]]]]
\end{forest}}}
\vfill


}

\frame[shrink]{
\frametitle{SpecCP -- das Vorfeld in Deklarativsätzen (II)}

\begin{itemize}
\item Ausschlaggebender Faktor für die Auswahl der zu bewegenden Phrase ist die Informationsstruktur des Satzes:\\
Was an vorangehende oder sonstwie bekannte Information anknüpft, steht innerhalb des Satzes eher
links ($\to$ vorzugsweise im Vorfeld), und was für den Gesprächspartner neu ist, steht eher rechts. 
\pause
\item Bewegung ins Vorfeld von Deklarativsätzen wird auch Topikalisierung genannt.\\
      Der Fokus kann aber auch im Vorfeld stehen. Auch Expletiva.
\pause
\item Achtung:\\
      Vorfeldbesetzung hat nicht denselben Status wie die Topikalisierung im Englischen!
\pause
\item Analyse funktioniert auch für sogenannte Fernabhängigkeiten:
      \ea
      {}[Um zwei Millionen Mark]$_i$ soll er versucht haben,\\ {}[eine Versicherung \_$_i$ zu betrügen].
      \z
\end{itemize}

}


\iftoggle{gb-intro}{
\frame{
\frametitle{SpecCP -- das Vorfeld in Nebensätzen}

In Relativ- und \emph{w}-Interrogativsätzen wird eine Phrase mit einem entsprechenden Wort (je nachdem: Determinierer, Pronomen, Adverb) ins Vorfeld gestellt.
\begin{itemize}
\item Relativsätze müssen ein Relativpronomen in der vorangestellten Phrase enthalten:
\eal
\ex der Mann, [\blauit{der}] das gesagt hat
\pause
\ex der Mann, [\blauit{den}] wir kennen
\pause
\ex der Mann, [\blauit{dessen} Vorschlag] wir diskutieren
\pause
\ex der Mann, [von \blauit{dem}] wir reden
\pause
\ex der Mann, [über \blauit{dessen} Vorschlag] wir reden
\pause
\ex die Art und Weise, [\blauit{wie}] wir über ihn reden
\zl
\end{itemize}

}
\frame{
\frametitle{SpecCP -- das Vorfeld in Nebensätzen}

\begin{itemize}
\item Interrogativsätze müssen ein Interrogativpronomen in der vorangestellten Phrase enthalten:
\eal
\ex Ich frage mich, [\blauit{wer}] das gesagt hat.
\pause
\ex Ich frage mich, [\blauit{wen}] du überhaupt kennst.
\pause
\ex Ich frage mich, [\blauit{wessen} Vorschlag] wir gerade diskutieren.
\pause
\ex Ich frage mich, [von \blauit{wem}] wir gerade reden.
\pause
\ex Ich frage mich, [über \blauit{wessen} Vorschlag] wir gerade reden.
\pause
\ex Ich frage mich, [\blauit{wie}] wir über ihn reden (sollten).
\pause
\ex Ich frage mich, [\blauit{ob}] er kommt.
\zl
\pause
\item
Man spricht hier von \blaubf{\emph{w}-Bewegung} (für Engl.\ \emph{wh}-Bewegung)\\
(auch im Fall von Relativsätzen mit \emph{der}, \emph{die}, \emph{das}).
\pause
%\bigskip
\item In allen übrigen Sätzen bleibt SpecCP leer.\\
(Zu "`unsichtbarer"' Besetzung von SpecCP siehe \hyperlink{np-bewegung}{\emph{w}-Bewegung})
\end{itemize}

}


%\fi 

\subsubsubsection{V und I - die rechte Satzklammer}

\frame{
\frametitle{\vnull und \inull{} -- die rechte Satzklammer (I)}

\savespace
\begin{itemize}
\item Deutsch hat keine lexikalischen Einheiten der Kategorie I,\\
\dash, auch Hilfsverben sind wirkliche Verben (Kategorie V). 
\pause
\item VPen können verschachtelt auf"|treten (Rekursion bzw.\ Rekursivität): 

Eine Regel, hier Einfügung einer VP, kann wiederholt angewendet werden.

Eine untergeordnete VP ist jeweils Komplement der
nächsthöheren.
\ea
Bis Anna\\
{}{\blau<5>[\sub{VP}} {\blau<4>[\sub{VP}} {\blau<3>[\sub{VP}} das Buch auf den Tisch gelegt{\blau<3>]} haben{\blau<4>]} wird{\blau<5>]}, 
\z
\end{itemize}

}

\frame[shrink]{
\frametitle{\vnull und \inull{} -- die rechte Satzklammer (II)}

\setlength\leftmargini{1em}
\begin{itemize}
\item Die I-Position ist verantwortlich für:
\begin{itemize}
\item Finitheit
\item die morphosyntaktischen Merkmale Tempus und Modus
\item die Kongruenz in Person und Numerus zwischen Subjekt und Verb (\emph{Spec-Head-Agreement})
\end{itemize}
\pause
\item Wo steht Verb in Sätzen mit Endstellung des finiten Verbs?% (=~Verbletztsätze):
\begin{itemize}
\item Möglichkeit 1:\\ Das Verb bleibt (wie im Englischen) an der \vnull-Stelle stehen. 
\vnull ist dann mit der leeren \inull-Position verkettet. 
Bezeichnung: verdeckte Verkettung (oder: \hypertarget{abstrakte Bewegung}{abstrakte Bewegung}, verdeckte Bewegung).
\pause
\item Möglichkeit 2:\\ Das Verb wird (wie im Französischen) an die \inull-Stelle angehoben. 
\end{itemize}
\pause
scheint im Dt.\ kein klares Indiz für das eine oder das andere zu geben.

Allg.\ gibt es für Ansatz einer \inull-Position im
Dt.\ keinen direkten Nachweis. 

%% Dass \inull normalerweise am Satzende
%% angesetzt wird, ist eher forschungsgeschichtlich motiviert. 

Im Folgenden: Annahme einer verdeckten Verkettung
\end{itemize}

}

\subsubsubsection{Verbzusätze}

\frame{
\frametitle{Verbzusätze: Kopfadjunkte}

\begin{itemize}
\item Verbzusätze sind "`Nebenköpfe"' zu \vnull, \dash Kopfadjunkte:

\begin{center}%
\begin{tabular}[t]{ccc}
&\node{v1}{\vnull}\\[4ex]
\node{p}{\pnull}   &                  & \node{v2}{\vnull}\\[4ex]
\node{doktor}{an} & & \node{mutter}{lachen}
\end{tabular}%
\nodeconnect{v1}{p}\nodeconnect{v1}{v2}
\nodeconnect{p}{doktor}\nodeconnect{v2}{mutter}
\end{center}


\item Kopfadjunkte stellen eine Erweiterung des \xbar-Schemas dar.\\
(später mehr: \hyperlink{inkorporation}{\siehe Inkorporation})
\end{itemize}



}

\frame{
\frametitle{Verbzusätze}

\begin{itemize}
\item In V1- und V2-Sätzen "`strandet"' der Verbzusatz am Satzende, es wird also nur die eigentliche Verbform bewegt.
\eal
\ex Als Anna die Tür aufschloss, 
\ex Anna schloss die Tür auf. 
\ex Schließ die Tür auf, Anna! 
\zl

\pause
\item In Verbletztsätzen bilden Verbzusatz und Verbform in der gesprochenen Sprache eine Einheit, \dash ein prosodisches Wort.

Indiz dafür, dass das Verb in Verbletztsätzen tatsächlich an der Position \vnull steht. 

\pause
\item Anders verhalten sich Präfixbildungen (\zb \emph{verschließen}) und\\
"`feste Zusammensetzungen"' (\zb \emph{untersuchen}, \emph{umarmen}).
\end{itemize}


}

%\if 0
\subsubsubsection{SpecIP}

\frame{
\frametitlefit{Subjektposition SpecIP im Deutschen: Gibt es immer ein Subjekt?}

\begin{itemize}
\item \blaubf{Projektionsprinzip}:\\
Lexikalische Information muss syntaktisch realisiert werden. 
\item \blaubf{erweitertes Projektionsprinzip (EPP)}:\\
Lexikalische Information muss syntaktisch realisiert werden. 

Jeder Satz enthält ein Subjekt. 

\pause
\item Ein Problem für die These der universellen Geltung des EPP sind deutsche Sätze wie: 

\eal
\ex Heute wird nicht gearbeitet. 
\ex Mir ist kalt. 
\ex Ihn schwindelt.
\ex Ihm graut vor der Prüfung.
\zl
\end{itemize}

}

\frame{
\frametitlefit{Subjektposition SpecIP im Deutschen: Konstituentenstellung (I)}


Es gibt Linksversetzungen im Mittelfeld (innerhalb von IP und VP). 

Bezeichnung: \emph{Scrambling} (nach spöttischem Ausspruch des Deutschkenners Mark
Twain, vgl.\ \emph{scrambled eggs} = Rührei, \siehe auch \citew{Ross67,Ross86a-u}).
\nocite{Twain1880a}

\pause
Auslösende Faktoren von Scrambling: 
\begin{itemize}\itemsep0pt
%\item Belebtheit (präferierte Abfolge: Belebtes vor Unbelebtem)
\item Definitheit (präferierte Abfolge: Bestimmtes vor Unbestimmtem)
\item Informationsgehalt (präferierte Abfolge: Bekanntes vor Neuem)
\item Konstituentenlänge (kurz vor lang)\nocite{Behaghel09,Behaghel30}
\end{itemize}

\pause
Umgestellte Konstituenten können vor dem Subjekt stehen, bzw.\ Subjekt zwischen
anderen Konstituenten:
\eal
\ex dass [im Saal] [drei Paare] tanzen. 
\ex weil [dieses Buch] [niemand] [auf den Tisch] gelegt hat. 
\zl
$\to$ Ansatz einer Subjektposition ist nicht so offensichtlich wie im Englischen.

}

\frame{
\frametitlefit{Subjektposition SpecIP im Deutschen: Konstituentenstellung (II)}

\eal
\ex dass [im Saal] [drei Paare] tanzen. 
\ex weil [dieses Buch] [niemand] [auf den Tisch] gelegt hat. 
\zl

Vermutung:\\
Scrambling in (\mex{0}) = Bewegung einer XP an Adjunktposition vor der Subjektposition

Aber das ist nicht sicher nachweisbar: Wenn es stimmt, dass unbetonte Objektpronomen wie \emph{es} am linken
Rand der VP (an der sogenannten Wackernagelposition) stehen, weisen die
folgenden Varianten darauf hin, dass die Subjektphrase zwar an der
Subjektposition vor der Wackernagelposition stehen kann, aber nicht muss: 
\eal
\ex[]{
dass [\sub{Nom} Anna] es sofort entdeckte
}
\ex[]{
dass \_$_i$ es [\sub{Nom} Anna]$_i$ sofort entdeckte 
}
%% \ex[*]{
%% dass [Dativ Anna] es sofort auf"|fiel
%% }
%% \ex[]{
%% dass es [Dativ Anna] sofort auf"|fiel 
%% }
\zl


}

\frame{
\frametitlefit{Subjektposition SpecIP im Deutschen: Konstituentenstellung (III)}


Deutung: Im Deutschen kann das Subjekt an zwei Positionen stehen.  
\begin{itemize}
\item Option 1: Das Subjekt steht «weiter unten» im Satz (am ehesten an der
Position SpecVP) und ist mit der leeren Subjektposition SpecIP verkettet
(=~verdeckte Verkettung; andere Fachausdrücke: verdeckte Bewegung, abstrakte
Bewegung). 

\pause
{\small (leere Subjektposition kann auch für subjektlose Sätze
postuliert, wenn auch nicht nachgewiesen werden $\to$ rein theorieinterne
Rettung für das EPP.

Dann muss man aber sicherstellen, dass solche leeren Subjekte nicht an Stellen
auf"|treten, an denen nicht-leere Subjekte stehen müssen.)}

\pause
\item Option 2: Das Subjekt steht an der Subjektposition SpecIP. 
\end{itemize}
\pause
Fazit: Im Deutschen ist zwischen Subjekt (=~Argument im Nominativ) und Subjektposition (=~SpecIP)
zu unterscheiden.


}

\subsubsubsection{Expletiva}

\frame[label=expletivum]{
\frametitle{Expletiva: Pseudo-Argumente}

\begin{itemize}
\item Expletivum = "`Füllform"', Obergriff für Pseudo-Argument, Platzhalter und Korrelat 
\pause
\item Pronomina: (E: \emph{it}, D: \emph{es}, F: \emph{il}, \emph{le}) als gewöhnliches Subjekt oder Objekt (Stellvertreter) mit Bezug auf eine Nominalphrase:
\eal
\ex (I am reading a book.) [It] is interesting. I like [it]. 
\ex (Ich lese ein Buch.) [Es] ist interessant. Ich mag [es].
\ex (Je lis un livre.) [Il] est intéressant. Je [l']estime.
\zl
\pause
\item Expletiv I: Pronomen als Pseudo-Argument (unpersönliches Subjekt oder Objekt): 
\eal
\ex {}[It] is raining. %Take [it] easy! 
\ex {}[Es] regnet. Sie bringt [es] bis zur Professorin.
\ex {}[Il] pleut.
\zl
\end{itemize}
}

\frame{
\frametitle{Expletiva: Korrelate zu Nebensätzen}

%% \begin{itemize}
%% \item 
Expletiv II: Pronomen als Korrelat eines Nebensatzes\\
(Engl.\ nur von Subjektnebensätzen, Dt.\ auch von Objektnebensätzen)
%(Index i: jeweiligen Phrasen "`gehören zusammen"'):
\eal
\ex {}[It]$_i$ strikes me [that Bill did not come]$_i$
\ex Mich störte [es]$_i$, [dass Otto ständig gähnte]$_i$ 
\zl
\ea
Anna schätzte [es]$_i$ nicht, [dass Otto ständig gähnte]$_i$
\z

\pause
Pronominaladverb als Korrelat (Nebensatz entspricht Präpositionalobjekt): 
\ea
Anna fand sich nicht [damit]$_i$ ab, [dass Otto ständig gähnte]$_i$
\z
%\end{itemize}
}

\exewidth{\exnrfont(235)}
\frame{
\frametitle{Expletiva: Vorfeldplatzhalter}


\begin{itemize}
\item Expletiv III: Pronomen \emph{es} als Vorfeldplatzhalter ($\neq$ Subjektplatzhalter) (Deutsch!): 
\eal
\ex {}[Es] kamen drei Männer.\\(Aber nicht: *Es scheint, dass es drei Männer kamen.) 
\ex {}[Es] wurde gearbeitet.\\(Aber nicht: *Es scheint, dass es gearbeitet wurde.) 
\zl
\pause
\item Expletiv IV: Adverb \emph{there} als Subjektplatzhalter (Englisch!)   nur in Verbindung mit einer nachgestellten Nominalphrase, mit der das Verb gegebenenfalls kongruiert: 
\ea
{}[There]$_i$ were [three men]$_i$ in the room.\\
(Auch: It seems, that [there]$_i$ were [three men]$_i$ in the room.)
\z
\end{itemize}
}

\frame{
\frametitle{Expletiva: Subjektplatzhalter}


\begin{itemize}
\item Expletiv V: Pronomen als Subjektplatzhalter (Französisch)   das Verb kongruiert nur mit dem Expletiv:
\ea
{}[Il]$_i$ est [arrêté trois hommes]$_i$.\\(Auch: Il semble qu [il]$_i$ est arrêté [trois hommes]$_i$.) 
\z
\end{itemize}


Anmerkung: Als eine Art
Expletiv kann man auch das Reflexivpronomen bei echt reflexiven Verben ansehen
(z.\,B.: \emph{sich beeilen}, \emph{sich etwas vornehmen}).


}

\frame{
\frametitle{Das englische \emph{there}}

\resizebox{\textwidth}{!}{\begin{tabular}{|l|l|l|l|l|l|l|}\hline
SpecCP & \cnull & SpecIP & \inull & Satzadverbialien & \vnull   & Rest der VP\\\hline\hline
       &        & There  & arrive$_k$ d &                  & \_$_k$ & three Magi.\\
       &        & There  & will   &                  & arrive   & three Magi.\\
       & Will$_k$  & there  & t$_k$ &                  & arrive   & three Magi?\\
       & that   & there  & will   &                  & arrive   & three Magi.\\\hline
\end{tabular}}

\bigskip

Frage: Wo genau steht die Phrase \emph{three Magi}?\\
(\alt<beamer>{\compare{Kasus}{Kasuslehre}}{\siehe Kasuslehre}, 
Stichwort \hyperlink{unakk}{"`nichtakkusativische Verben"'})

}

\frame{
\frametitle{Das Vorfeld-\emph{es}}


\resizebox{\textwidth}{!}{
\begin{tabular}{|l|l|l|l|l|}
\hline
%
SpecCP    & \cnull      & \mc{2}{l|}{IP (ohne \inull, \vnull)} & \vnull, \inull\\
Vorfeld   & Linke       & \mc{2}{l|}{Mittelfeld}                      & Rechte\\\cline{3-4}
          & Satzklammer & SpecIP           & Phrasen innerhalb der VP & Satzklammer\\
          &             & Subjektsposition &                          &\\\hline\hline
Es        & tanzen$_k$     & drei Paare & im Saal.                       & t$_k$ t$_k$ \\
          & dass         & drei Paare & im Saal                        & tanzen$_k$ \_$_k$\\
          & dass         & *es        & drei Paare im Saal             & tanzen$_k$ \_$_k$\\
{}[Drei Paare]$_i$ & tanzen$_k$   & t$_i$     & im Saal                        & t$_k$ t$_k$\\
{}[Drei Paare]$_i$ & tanzen$_k$   & *es t$_i$ & im Saal                        & t$_k$ t$_k$\\\hline
\end{tabular}
}

\bigskip
Man beachte, dass der Stern bei \emph{es} im
dritten und fünften Beispiel den Satz als ungrammatisch kennzeichnet.



}




\subsubsubsection{Spuren und Bewegung}

\frame{
\frametitle{Gründe für den Ansatz von Spuren bei Bewegung (I)} %: Anforderung von mehreren Positionen}
%

Auslöser von Bewegung scheint ganz allgemein zu sein, dass eine Konstituente
ein Merkmal aufweist, das "`eigentlich"' nicht zu der Position paßt, an der sie
steht, sondern zu einer anderen weiter oben im Baum. 

Besonders typisch sind Phrasen mit Fragewörtern\\
(interrogative Determinierer, Pronomen und Adverbien):
\ea
Ich weiß nicht, [was] Anna erwartet hat.  
\z

\begin{itemize}\itemsep0pt\topsep0pt\partopsep0pt
\item {}[was] ist internes Argument (Objekt) von \emph{erwartet}
 und sollte Bestandteil der VP, genauer Schwester der Verbform \emph{erwartet}, sein.
%
\item {}[was] enthält Fragewort, das die Satzart (Satzmodus)
 "`Fragesatz"' deutlich macht; damit gebildete Phrase sollte in
Spezifikatorposition SpecCP des Satzkerns \cnull stehen.
\end{itemize}
}

\frame{
\frametitle{Gründe für den Ansatz von Spuren bei Bewegung (II)}

Lösung: Verkettung der beiden Positionen.\\
Phrase an der oberen Position sichtbar $\to$ Bezeichnung = \emph{Bewegung}
\ea
Ich weiß nicht, [was]$_i$ Anna [t]$_i$ erwartet hat.\\
(Gemeint: Ich nehme an, dass Anna [etwas] erwartet hat. Worum handelt es sich?) 
\z

}


\frame{
\frametitle{Beschränkungen}

%\savespace\small

Konstituenten können nicht beliebig verkettet bzw.\ bewegt werden. 

Metaphorisch ausgedrückt: Der Weg durch den Strukturbaum muss von der Endposition aus
bis zur Ausgangsposition überblickt werden.\\
Diese Beobachtung wird mit dem Konzept der Spur formalisiert: 

\ea[*]{
Ich weiß nicht, [was]$_i$ dann der Tag kam, an dem Anna [t]$_i$ erwartete.\\
(Intendiert: Ich nehme an, dass dann der Tag kam, an dem Anna [etwas] erwartet hat. Worum handelt es
 sich?)
}
\z
I.\,A.\ sind eingebettete CPs (= Nebensätze) Hindernisse für den Blick zurück zur Ausgangsposition,
\dash aus Nebensätzen kann nichts  herausbewegt (extrahiert) werden.
 
}

%% \frame{
%% \frametitle{Beschränkungen (II)}


%% Abweichungen (hier markiert mit `!') besonders interessant:

%% \eal
%% \ex ! [Was]$_i$ glaubst du, dass Anna [t]$_i$ liest?
%% \ex ? [Wer]$_i$ glaubst du, dass [t]$_i$ dieses Buch liest?\\(Beurteilung schwankt!) 
%% \ex * [Was]$_i$ glaubst du nicht, dass jemand [t]$_i$ liest? 
%% \zl

%% }

\frame{
\frametitle{Verschiedene Bewegungsarten: A- und A'-Bewegung}

\savespace

Bewegungen folgen nicht alle denselben Regeln. 

Fachliteratur unterscheidet zwischen A-Bewegung und A'-Bewegung\\
(lies: Non-A-Bewegung; andere Schreibweise: \abar-Bewegung).

\emph{A} steht für \emph{Argument}, genauer für \emph{Argumentposition}.
\begin{itemize}\itemsep0pt
\item A-Bewegung: Bewegung zur Subjektposition (SpecIP), auch als NP-Bewegung bezeichnet.
Die Subjektposition ist eine Argumentposition.   
\item A'-Bewegung:
\begin{itemize}
\item Bewegung ins Vorfeld (SpecCP), also Topikalisierung und \emph{w}-Bewegung; 
\item Linksversetzung im Mittelfeld (Scrambling); 
\item Bewegung schwach betonter Pronomen an die Wackernagel-Position; 
\item Ausklammerung ins Nachfeld (Rechtsextraposition)\\
\ldots
\end{itemize}
\end{itemize}

\hyperlink{np-bewegung}{NP-Bewegung} und \hyperlink{w-Bewegung}{\emph{w}-Bewegung} werden noch behandelt.


}


\subsubsubsection{Small Clauses: verblose Satzäquivalente}

\frame[label=sc]{
\frametitle{Small Clauses: verblose Satzäquivalente (I)}

\smallexamples\small\parskip1pt
Bestimmte verblose Prädikativkonstruktionen werden auf eine Konfiguration zurückgeführt, 
in der Prädikativ + Bezugs-NP satzähnliche Einheit bilden.

\pause
Häufig wird dafür eine besondere funktionale Kategorie Agr (=~\emph{Agreement}; deutsch: Kongruenz,
Übereinstimmung) angesetzt. 

\pause
Prädikativ ist Komplement von Agr, die Bezugs-NP dessen Spezifikator.

\pause
Bezugsphrase = externes Argument des Prädikativs\\
(oder etwas unpräzise dessen Subjekt)

\pause
\agrnull ist teils leer (\mex{2}a), teils von einem Element besetzt, das traditionell als Konjunktion ((\mex{2}b) \emph{als}) oder als Präposition
((\mex{2}c) \emph{für}) bestimmt wird.

\ea
Der Torwart sagte, [\sub{CP} der Schiedsrichter sei ein Trottel] 
\z
\eal
\ex Der Torwart nannte [\sub{AgrP} den Schiedsrichter einen Trottel].
\ex Der Torwart betrachtete [\sub{AgrP} den Schiedsrichter als einen Trottel].
\ex Der Torwart hält [\sub{AgrP} den Schiedsrichter für einen Trottel].
\zl
Zum Kasus des Small-Clause-Spezifikators später.

}

\frame{
\frametitle{Small Clauses: verblose Satzäquivalente (II)}

%\enlargethispage{-\baselineskip}
%
\vfill
\centerline{\scalebox{0.5}{%
\begin{tabular}{ccccccc}
\mc{3}{c}{\node{cp}{CP}}\\[5ex]
\mc{3}{c}{\node{cs}{C'}}\\[6ex]
\node{c}{C}&             \mc{4}{c}{\node{ip}{IP}}\\[6ex]     
       & \node{np1}{NP}          &                    & & \mc{3}{c}{\node{is}{I$'$}}\\[5ex]
       &             & & \mc{3}{c}{\node{vp}{VP}}                                          & \node{i}{I}\\[5ex]
       &             & & \mc{3}{c}{\node{vs}{V$'$}}\\[5ex]
       &             & \mc{3}{c}{\node{agrp}{AgrP}}                               & \node{v}{V}\\[5ex]
       &             & \node{np2}{NP}                 & \mc{2}{c}{\node{agrs}{Agr$'$}}\\[5ex]
       &             &                    & \node{agr}{Agr} & \node{np3}{NP}\\[6ex]
\node{weil}{weil}& de\node{dt}{r Torwa}rt & den\node{ds}{ Schiedsrich}ter & \node{fuer}{für}      & ei\node{et}{nen Trott}el & \node{haelt}{hält}   & \node{tr}{\_}\\
\end{tabular}
\nodeconnect{cp}{cs}%
\nodeconnect{cs}{c}\nodeconnect{cs}{ip}%
\nodeconnect{ip}{np1}\nodeconnect{ip}{is}%
\nodeconnect{is}{vp}\nodeconnect{is}{i}%
\nodeconnect{vp}{vs}%
\nodeconnect{vs}{agrp}\nodeconnect{vs}{v}%
\nodeconnect{agrp}{np2}\nodeconnect{agrp}{agrs}%
\nodeconnect{agrs}{agr}\nodeconnect{agrs}{np3}%
\nodeconnect{c}{weil}\nodeconnect{agr}{fuer}%
\nodeconnect{v}{haelt}\nodeconnect{i}{tr}%
\nodetriangle{np1}{dt}\nodetriangle{np2}{ds}\nodetriangle{np3}{et}%
}}
%\mbox{}
\nocite{Abraham93a}

}

\frame[label=sc-probleme]{
\frametitle{Probleme: Voranstellbarkeit des Small Clause}

Im Deutschen bilden der Small-Clause-Spezifikator und der Rest der Small Clause
auf der Ebene der S-Struktur keine Konstituente, was noch gesondert zu erklären
wäre: 

\eal
\ex[]{
{}[\sub{CP}\hspace{4mm} Der Schiedsrichter sei ein Trottel], sagte der Torwart. 
}
\ex[*]{
{}[\sub{AgrP} Den Schiedsrichter für einen Trottel] hält der Torwart. 
}
\zl

}

\frame{
\frametitle{Probleme: Kategorie des Small Clause}

\small\smallexamples\parskip0pt
Außerdem: Manche Verben kommen nur mit Prädikativen
bestimmter Kategorien vor (\citealt[S.\,70]{Fanselow91a}; \citealt[S.\,63]{Demske94a}; \citealt[S.\,232]{Hoekstra87a}).
%
\eal
\ex[]{
Herr K.\ ist kein Verbrecher.
}
\ex[]{
Herr K.\ ist unschuldig.
}
\ex[]{
Herr K.\ ist in Berlin.
}
\zl
\eal
\ex[*]{
Der Richter macht Herrn K.\ einen Verbrecher.
}
\ex[]{
Das Gericht macht Herrn K.\ müde.
}
\ex[]{
Der Richter macht Herrn K.\ zum Verbrecher.
}
\zl
\eal
\ex[]{
Herr K.\ nennt den Richter einen Idioten.
}
\ex[]{
Herr K.\ nennt den Richter voreingenommen.
}
\ex[*]{
Herr K.\ nennt den Richter als/zum Idioten.
}
\zl
Small Clauses wären aber alle Agr-Projektionen $\to$\\
direkte Selektion der Kategorie unmöglich

}

\frame{
\frametitle{Konsequenz}

Nicht in allen GB-Varianten werden Small Clauses angenommen,\\
in HPSG werden die Phänomene anders behandelt.
\nocite{Bresnan82c,Williams83a,Booij90a,Hoekstra87a,Hoeksema91c,Fanselow91a,NW93a,Neeleman94a,Neeleman95a,ps2,Stiebels96a,Winkler97a}

Siehe \citew{Mueller2002b} und Vorlesung im nächsten Semester.

}

}%\end{gb-intro}
%\fi

\iftoggle{einfsprachwiss-exclude}{

\iftoggle{gb-intro}{

\subsection{Lokale Umstellung}

\frame{
\frametitle{Lokale Umstellung}


\begin{itemize}
\item Im \mf können Argumente in nahezu beliebiger Abfolge angeordnet werden.
\eal
\ex {}[weil] \rot{der Mann} \gruen{der Frau} \blau{das Buch} gibt
\ex {}[weil] \rot{der Mann} \blau{das Buch} \gruen{der Frau} gibt
\ex {}[weil] \blau{das Buch} \rot{der Mann} \gruen{der Frau} gibt
\ex {}[weil] \blau{das Buch} \gruen{der Frau} \rot{der Mann} gibt
\ex {}[weil] \gruen{der Frau} \rot{der Mann} \blau{das Buch} gibt
\ex {}[weil] \gruen{der Frau} \blau{das Buch} \rot{der Mann} gibt
\zl
\pause
\item In (\mex{0}b--f) muss man die Konstituenten anders betonen
und die Menge der Kontexte, in denen der Satz mit der jeweiligen Abfolge
geäußert werden kann, ist gegenüber (\mex{0}a) eingeschränkt \citep{Hoehle82a}. 

Abfolge in (\mex{0}a) = \blau{Normalabfolge} bzw.\ die \blau{unmarkierte Abfolge}.
\end{itemize}
}

\frame{
\frametitle{Zwei Möglichkeiten zur Analyse der lokalen Umstellung}


Es gibt theoretisch zwei Möglichkeiten (siehe \citew{Fanselow93a}):
\begin{itemize}[<+->]
\item Man nimmt eine Grundabfolge an und leitet alle anderen Abfolgen über Move-$\alpha$ daraus ab.
\Zb \citew{Frey93a}.
\pause
\item Man läßt verschiedene Grundstrukturen zu, die für die Analysen der verschiedenen Abfolgen direkt verwendet werden.\\
Es gibt keine Spuren. \Zb \citew{Fanselow2001a}.
%% \pause
%% \item Der Ansatz mit verschiedenen Grundstrukturen ist nicht so schlimm, wie ein reiner
%% Phrasenstrukturansatz, da
\item Obwohl mittels Skopusambiguitäten immer für den bewegungsbasierten Ansatz argumentiert wurde,
sind es gerade Skopusdaten,\\
die gegen diesen Ansatz sprechen \citep{Kiss2001a,Fanselow2001a}.
\end{itemize}


}

}% not gb-intro


}%\end{einfsprachwiss-exclude}


\input{passiv-als-bewegung}


\subsection{Zusammenfassung}

\frame{
\frametitle{Zusammenfassung}

Ziele:
\begin{itemize}
\item Zusammenhang zwischen bestimmten Strukturen erfassen, z.\,B.:
      \begin{itemize}
      \item Aktiv/Passiv
      \item Verbletzt-/Verberst-/Verbzweitstellung
      \item nahezu freie Abfolge der Konstituenten im Mittelfeld bei bestimmbarer Grundabfolge
      \end{itemize}
      Abbildung von D-Struktur auf S-Struktur.
\pause
\item Spracherwerb erklären. Dazu
      \begin{itemize}
      \item möglichst allgemeines Regel-Gerüst, das für alle Sprachen gleich ist\\
            (\xbart)
      \item allgemeine Prinzipien, die für alle Sprachen gelten, aber parametrisierbar sind
      \end{itemize}

\end{itemize}

}

\frame{
\frametitle{Übung}

Zeichnen Sie Syntaxbäume für die folgenden Sätze:
\eal
\ex dass der Mann der Frau hilft
\ex dass die Frau den Mann liebt
\ex dass der Mann geliebt wird
\ex Der Mann hilft der Frau.
\ex Der Mann wird geliebt.
\zl




}
