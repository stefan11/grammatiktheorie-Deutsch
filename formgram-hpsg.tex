\section{HPSG}

\outline{

\begin{itemize}
\item Begriffe
\item Phrasenstrukturgrammatiken
\item Government \& Binding (GB)
\item Generalisierte Phrasenstrukturgrammatik (GPSG)
\item Lexikalisch-Funktionale Grammatik (LFG)
%\item Lexical Mapping Theory (LMT)
%\item PATR
\item Kategorialgrammatik (CG)
\item \blau{Kopfgesteuerte Phrasenstrukturgrammatik (HPSG)}
\item Konstruktionsgrammatik (CxG)
\item Baumadjunktionsgrammatik (TAG)
\end{itemize}
}


\frame{
\frametitle{Head-Driven Phrase Structure Grammar (HPSG)}


\begin{itemize}[<+->]
\item Von Carl Pollard und Ivan Sag Mitte der 80er Jahre entwickelt\\
      \citep{ps,ps2}
\item HPSG gehört wie LFG zur West-Coast-Linguistik.
\item Lehrmaterial und Überblicksartikel:\\
      \citew{MuellerLehrbuch3,MuellerArten,LM2006a,MuellerCurrentApproaches}\nocite{Mueller99a,Mueller2002b}
\item Handbuch mit Einführung und Überblickskapiteln zu diversen Phänomenen und Theorievergleich: \citet*{HPSGHandbook}
\item Ivan Sag war einer derjenigen, die GPSG entwickelt haben.
\end{itemize}




}


\subsection{Allgemeines zum Repräsentationsformat}

\frame{

\frametitle{Grundlegendes zur HPSG}
\small
\begin{itemize}
\item lexikalisiert (head-driven/kopfgesteuert)
\pause
\item zeichenbasiert \citep{Saussure16a-de}
\pause
%\item unifikationsbasiert
\item getypte Merkmalstrukturen (Lexikoneinträge, Phrasen, Prinzipien)
\pause
\item Mehrfachvererbung
\pause
\item monostratale Theorie\\~\\
\begin{minipage}[t]{2.5cm}
~\\[-20mm]
\begin{itemize}
\item \blau<6>{Phonologie}\rnode{1}{}
\item \blau<7>{Syntax}\rnode{2}{}
\item \blau<8>{Semantik}\rnode{3}{}
\end{itemize}
\end{minipage}%
\parbox[t]{3cm}{
\resizebox{5cm}{!}{
\(
\ms[word]{
\rnode{4}{phon}   & \blau<6>{\phonliste{ Grammatik }} \\[1mm]
synsem$|$loc & \ms[loc]{ \rnode{5}{cat}  & \blau<7>{\ms[cat]{ head & \ms[noun]{ case & \ibox{1}\\
                                                       }\\[6mm]
                                       spr & \liste{ DET[{\sc case}~\ibox{1}] } \\
                                     }} \\[6mm]
              \rnode{6}{cont} & \blau<8>{\ldots \ms[grammatik]{ inst & X \\
                                   }}\\
            }\\
}
\)
}
}
\end{itemize}
}

\frame{
\frametitle{Einflüsse}

\begin{itemize}[<+->]
\item Kategorialgramamtik\\
      (Funktor-Argument-Strukturen, Valenz, Argumentkomposition)
\item GPSG\\
      (ID/LP-Format, Slash-Mechanismus für Fernabhängigkeiten)
\item Government \& Binding\\
      (u.a.\,Analyse der Verbstellung im Deutschen)
\end{itemize}



}


\frame{
\frametitle{Valenz und Grammatikregeln: PSG}


\begin{itemize}
\item große Anzahl von Regeln:\\
      \begin{tabular}[t]{l@{~$\to$~}l@{\hspace{6em}}l}
      S &  NP, V               & {\em X schläft\/}\\
      S &  NP, NP, V           & {\em X Y liebt\/}\\
      S &  NP, PP[{\it über\/}], V           & {\em X über Y spricht\/}\\
      S &  NP, NP, NP, V       & {\em X Y Z gibt\/}\\
      S &  NP, NP, PP[{\it mit\/}], V       & {\em X Y mit Z dient\/}\\
      \end{tabular}
\pause
\item Verben müssen mit passender Regel verwendet werden.
\end{itemize}

}

\frame{

\frametitle{Valenz und Grammatikregeln: HPSG}

\begin{itemize}
\item Argumente als komplexe Kategorien in der lexikalischen Repräsentation
      eines Kopfes repräsentiert\\
      (wie Kategorialgrammatik)
\pause
\item \begin{tabular}[t]{@{}lll}
      Verb             & \comps\\
      {\em schlafen\/} & \sliste{ NP }\\
      {\em lieben\/}   & \sliste{ NP, NP }\\
      {\em sprechen\/} & \sliste{ NP, PP[{\it über\/}] }\\
      {\em geben\/}    & \sliste{ NP, NP, NP }\\
      {\em dienen\/}   & \sliste{ NP, NP, PP[{\it mit\/}] }\\  
      \end{tabular}
\end{itemize}


}

\frame{
\frametitle{Beispielstruktur mit Valenzinformation (I)}

\vfill
\hfill
\begin{forest}
sm edges
[V{[\comps \eliste]}
	[{\ibox{1} NP[\type{nom}]}
		[Peter]]
	[V{[\comps \sliste{ \ibox{1} }]}
		[schläft]]]
\end{forest}
\hfill\hfill\mbox{}
\vfill
V[\comps \sliste{ }] entspricht hierbei einer vollständigen Phrase\\
(VP oder auch S)
\vfill
}

\frame{
\frametitle{Beispielstruktur mit Valenzinformation (II)}

\vfill
\hfill
\begin{forest}
sm edges
[V{[\comps \eliste]}
	[{\ibox{1} NP[\type{nom}]}
		[Peter]]
	[V{[\comps \sliste{ \ibox{1} }]}
		[{\ibox{2} NP[\type{acc}]}
			[Maria]]
		[V{[\comps \sliste{ \iboxsp{1}, \ibox{2} }]}
			[erwartet]]]]
\end{forest}\hfill\hfill\mbox{}
\vfill

}



\frame{
\frametitle{SOV vs.\ SVO: Repräsentation von Subjekten}

\begin{itemize}
\item Wissenschaftler*innen, die zum Deutschen arbeiten, nehmen an, dass das Subjekt finiter Verben
  sich wie andere Argumente verhält. (\citealp{Pollard90a-Eng}; \citealp[\page 376]{Eisenberg94b})

HPSG: Subjekte und Komplemente werden auf derselben Valenzliste repräsentiert (\comps).

\pause
\item English: Subjekte sind anders.
\pause
\item \argst ist die zugrundeliegende Repräsentation, die alle Argumente enthält. \citep{DKW2021a}
\pause
\item Sprachabhängige Verteilung auf die Valenzmerkmale \spr und \comps.

\medskip
\oneline{%
\begin{tabular}[t]{@{}llll}
      verb          & \spr                      & \comps                                     & \argst\\
      \emph{sleep}  & \sliste{ NP[\type{nom}] } & \sliste{}                                  & \sliste{ NP[\type{nom}] }\\
      \emph{expect} & \sliste{ NP[\type{nom}] } & \sliste{ NP[\type{acc}] }                  & \sliste{ NP[\type{nom}], NP[\type{acc}] }\\
      \emph{speak}  & \sliste{ NP[\type{nom}] } & \sliste{ PP[\type{about}] }                & \sliste{ NP[\type{nom}], PP[\type{about}] }\\
      \emph{give}   & \sliste{ NP[\type{nom}] } & \sliste{ NP[\type{acc}], NP[\type{acc}] }  & \sliste{ NP[\type{nom}], NP[\type{acc}], NP[\type{acc}] }\\
      \emph{serve}  & \sliste{ NP[\type{nom}] } & \sliste{ NP[\type{acc}], PP[\type{with}] } & \sliste{ NP[\type{nom}], NP[\type{acc}], PP[\type{with}] }\\  
      \end{tabular}}

\end{itemize}


}

\frame{
\frametitle{Beispielanalyse mit \spr und \comps}

\centerline{%
\scalebox{.8}{%
\begin{forest}
sm edges
[V{\feattab{\spr \eliste,\\
            \comps \eliste}}
  [\ibox{1} NP [Kim]]
  [V{\feattab{\spr \sliste{ \ibox{1} },\\
              \comps \eliste}}
    [V{\feattab{\spr \sliste{ \ibox{1} },\\
                \comps \sliste{ \ibox{2} }}}
      [talks]]
    [\ibox{2} P{\feattab{\spr \sliste{ },\\
                \comps \sliste{ }}}
      [P{\feattab{\spr \sliste{ },\\
                \comps \sliste{ \ibox{3} }}} [about]]
      [\ibox{3} N{\feattab{\spr \sliste{ },\\
                     \comps \sliste{ }}}
        [\ibox{4} Det [the]]
        [N{\feattab{\spr \sliste{ \ibox{4} },\\
                     \comps \sliste{ }}} [summer] ]]]]]
\end{forest}}}

}


\subsubsection{Repräsentation der Konstituentenstruktur}

\frame{
\frametitle{Repräsentation der Konstituentenstruktur}

\centerline{%
\begin{forest}
sm edges
[NP
	[Det
		[dem;the]]
	[N
		[Mann;man]]]
\end{forest}
}

Der Baum kann mit Merkmalsbeschreibungen repräsentiert werden:

\ea
\ms{ 
  phon     & \phonliste{ dem Mann }\\[1mm]
  head-dtr & \onems{ phon \phonliste{ Mann }
                 }\\
  non-head-dtrs & \sliste{ \onems{ phon \phonliste{ dem }
                            }}
}
\z

}

\subsubsection{Feature geometry}

\frame{
\frametitle{Komplette Merkmalsgeometrie}

\ea
\label{LE-Grammatik}
\scalebox{.7}{%
\ms[word]{
phon   & \phonliste{ Grammatik } \\[1mm]
synsem & \ms{ loc & \ms[local]{ cat  & \ms[category]{ head & \ms[noun]{ case & \ibox{1}
                                                                      }\\[3mm]
                                                     spr & \sliste{ Det[\textsc{case}~\ibox{1}] }\\
                                                     comps & \eliste\\[1pt]
                                                    } \\[6mm]
                                cont & \ms[mrs]{
                                       ind & \ibox{2} \ms{ per & third\\
                                                           num & sg\\
                                                           gen & fem\\
                                                         }\\
                                       rels & \sliste{ \ms[grammatik]{ inst & \ibox{2} 
                                                                    } }
                                        }
                              }\\
               nonloc & \ms{ inher$|$slash   & \eliste{}\\
                             to-bind$|$slash & \eliste{}\\
                           }
            }
}}
\z

Information, die für Strukturteilung gebraucht wird, wird zusammen gruppiert.


}

\subsubsection{ID-Schemata}

\frame{
\frametitle{Das Kopf-Komplement-Schema (vorläufig)}


\type{head-complement-phrase}\istype{head"=complement"=phrase} \impl\\
\onems{
      synsem$|$loc$|$cat$|$comps \ibox{1} \\
      head-dtr$|$synsem$|$loc$|$cat$|$comps \ibox{1} $\oplus$ \sliste{ \ibox{2} } \\
      non-head-dtrs \sliste{ [ \synsem \ibox{2} ] }
      }

\pause

\ea
\onems[head-complement-phrase]{
phon \phonliste{ Peter schläft }\\
synsem$|$loc$|$cat$|$comps \eliste\\
head-dtr \onems{ phon \phonliste{ schläft }\\
                 synsem$|$loc$|$cat$|$comps \sliste{ \ibox{1} NP[\type{nom}] }
               }\\
non-head-dtrs \sliste{ \onems{ phon \phonliste{ Peter }\\
                               \synsem \ibox{1}
                             } }
}
\z


}

\subsubsection{LP-Regeln}

\frame{
\frametitle{Linearisierungsregeln}

\eal
\ex\label{lp-ini-arg} 
Head[\initial$+$] $<$ Complement
\ex 
Complement $<$ Head[\initial --]
\zl

\pause
Präpositionen haben \initialw `$+$' und müssen deshalb ihren Argumenten vorangehen.
\eal
\ex[]{
{}[in [den Schrank]]
}
\ex[*]{
{}[[den Schrank] in]
}
\zl
\pause
Verben in Letztstellung haben den Wert `$-$' und müssen ihren Argumenten folgen.
\eal
\ex[]{
{}dass [er [ihn umfüllt]]
}
\ex[*]{
{}dass [er [umfüllt ihn]]
}
\zl


}


\subsubsection{Kopfmerkmale}

\frame{
\frametitle{Kopfmerkmale}

\begin{itemize}
\item Information über die Verbform muss am obersten Knoten der Projektion verfügbar sein:
\eal
\label{bsp-projektion-v-merkmale}
\ex[]{
{}[Dem Mann helfen] will er nicht.
}
\ex[]{
{}[Dem Mann geholfen] hat er nicht.
}
\ex[*]{
{}[Dem Mann geholfen] will er nicht.
}
\ex[*]{
{}[Dem Mann helfen] hat er nicht.
}
\zl
\end{itemize}
}


\frame{
\frametitle{Projektion von Merkmalen entlang des Kopfpfades}

\settowidth{\offset}{V[\type{fi}}
\settowidth{\offsetup}{V[\type{fin}}
\centerline{
\begin{forest}
sm edges, for tree={l+=\baselineskip}
[\gruen{V}{[\gruen{\type{fin}}, \comps \eliste]}, name=fin1
	[\ibox{1} NP{[\type{nom}]}
		[jemand]]
	[\gruen{V}{[\gruen{\type{fin}}, \comps \sliste{ \ibox{1} }]}, name=fin2
		[\ibox{2} NP{[\textit{dat}]}
			[dem Kind,roof]]
		[\gruen{V}{[\gruen{\type{fin}}, \comps \sliste{ \ibox{1}, \ibox{2} }]}, name=fin3
			[\ibox{3} NP{[\textit{acc}]}
				[das Buch,roof]]
			[\gruen{V}{[\gruen{\type{fin}}, \comps \sliste{ \ibox{1}, \ibox{2}, \ibox{3} }]}, name=fin4
				[gibt]]]]]	
tikz={\draw[<->] ($(fin1.south west)+(\offsetup,0)$) to ($(fin2.north west)+(\offset,0)$);
      \draw[<->] ($(fin2.south west)+(\offsetup,0)$) to ($(fin3.north west)+(\offset,0)$);
      \draw[<->] ($(fin3.south west)+(\offsetup,0)$) to ($(fin4.north west)+(\offset,0)$);}
\end{forest}
}

}

\frame{
\frametitle{Strukturteilung der \head-Werte}

\centerline{
\scalebox{.8}{%
\begin{forest}
sm edges
[\ms{head & \gruen{\ibox{1}}\\
     comps & \sliste{ }
     }
	[{\ibox{2} NP{[\type{nom}]}}
		[jemand]]
	[\ms{
             head & \gruen{\ibox{1}}\\
             comps & \sliste{ \ibox{2} }
             }
		[\ibox{3} NP{[\textit{dat}]}
			[dem Kind, roof]]
		[\ms{
                                                                                   head & \gruen{\ibox{1}}\\
                                                                                   comps & \sliste{ \ibox{2}, \ibox{3} }
                                                                                    }
			[\ibox{4} NP{[\textit{acc}]}
				[das Buch, roof]]
			[\ms{
                                                                                   head & \gruen{\ibox{1} \ms[verb]{
                                                                                                  vform & fin
                                                                                                  }}\\
                                                                                   comps & \sliste{ \ibox{2}, \ibox{3}, \ibox{4} }
                                                                                    }
				[gibt]]]]]	
\end{forest}}}
}


\subsubsection{Typhierarchien und Vererbung}


\frame{
\frametitle{Typhierarchien und Vererbung}


\centerline{%
\begin{forest}
type hierarchy
[sign
  [word]
  [phrase 
    [non-headed-phrase]
    [headed-phrase [head-complement-phrase]]]]
\end{forest}}

\begin{itemize}
\item Alle Merkmalstrukutren sind in der HPSG getypt.
\pause
\item Typen werden in Hierarchien geordnet.
\pause
\item Subtypen erben Beschränkungen von Obertypen.

\pause

\item Beispiel: \type{headed-phrase}
\ea
\type{headed"=phrase}\istype{headed"=phrase} \impl
\ms{ 
synsem$|$loc$|$cat$|$head \ibox{1}\\
head-dtr$|$synsem$|$loc$|$cat$|$head \ibox{1}\\
} 
\z


\end{itemize}


}


\frame{
\frametitle{Vererbung von Beschränkungen}

\begin{itemize}
\item
\ea
\label{head-arg-schema-hfp}
Head-Complement Schema + Head Feature Principle:\\
\onems[head-complement-phrase~]{
synsem$|$loc$|$cat  \ms{ \visible<2->{\gruen{head}   & \gruen{\ibox{1}}} \\
                          comps & \ibox{2}
                        }\\
head-dtr$|$synsem$|$loc$|$cat \ms{ \visible<2->{\gruen{head}   & \gruen{\ibox{1}}} \\
                                   comps & \ibox{2} $\oplus$ \sliste{ \ibox{3} }
                                 } \\
non-head-dtrs   \sliste{ [ synsem \ibox{3} ] }
}
\z
\medskip

Beschränkungen für \type{head-complement-phrase} \pause
und von \type{headed-phrase} geerbte Beschränkungen
\pause
\item Vererbungshierarchien sind wichtig, um Generalisierungen zu erfassen.\\
Sie werden im Bereich des Lexikons seit \citew*{FPW85a} benutzt.
\end{itemize}


}






\subsection{Passiv}

\frame{
\frametitle{Passiv}

\begin{itemize}[<+->]
\item HPSG folgt Bresnans Argumentation, dass das Passiv im Lexikon behandelt werden sollte.

\item Eine Lexikonregel nimmt den Verbstamm als Eingabe und lizenziert die Partizipform, wobei das
prominenteste Argument (das sogenannte designierte Argument) unterdrückt wird.

\item Da grammatische Funktionen in der HPSG keine Bestandteile der Theorie sind,
      braucht man auch keine Mapping-Prinzipien,\\
      die Objekte auf Subjekte mappen.

\item Allerdings muss die Kasusänderung bei Passivierung erklärt werden.
\end{itemize}



}

\subsubsection{Struktureller Kasus}

\frame{
\frametitle{Struktureller und lexikalischer Kasus}

\begin{itemize}
\item Wenn Kasus von Argumenten von der syntaktischen Umgebung abhängt,
      spricht man von \blau{strukturellem Kasus}. \\
      Ansonsten haben die Argumente \blau{lexikalischen Kasus}.
\pause
\item Beispiele für strukturellen Kasus sind:
\eal
\ex \blau{Der Installateur} kommt.
\pause
\ex Der Mann läßt \blau{den Installateur} kommen.
\pause
\ex das Kommen \blau{des Installateurs}
\zl

\pause
\item In (\mex{0}) handelt es sich um Subjektskasus, in (\mex{1}) um Objektskasus:
\eal
\ex Karl schlägt \blau{den Hund}.
\ex \blau{Der Hund} wird geschlagen.
\zl
\end{itemize}
}

\subsubsection{Lexikalischer Kasus}


\frame{
\frametitle{Lexikalische Kasus}

\begin{itemize}
\item Vom Verb abhängiger Genitiv ist lexikalischer Kasus:\\
      Bei Passivierung ändert sich der Kasus eines Genitivobjekts nicht.
\eal
\ex[]{
Wir gedenken \blau{der Opfer}.
}
\ex[]{
\blau{Der Opfer} wird gedacht.
}
\ex[*]{
\blau{Die Opfer} wird/werden gedacht.
}
\zl
\pause
(\mex{0}b) = unpersönliches Passiv, es gibt kein Subjekt.

\pause
\item Den Dativ zähle ich zu den lexikalischen Kasus (umstritten).
\pause
\item Zur Diskussion und weiteren Fällen von lexikalischem Kasus siehe \citew{MuellerLehrbuch3}.
\end{itemize}

}


\subsubsection{Valenzinformation und das Kasusprinzip}


\frame{
\frametitle{Valenzinformation und das Kasusprinzip}





\begin{prinzip-break}[\hypertarget{case-p}{Kasusprinzip (vereinfacht)}]
\label{case-p}
\begin{itemize}
\item In einer Liste, die sowohl das Subjekt als auch die Komplemente eines verbalen Kopfes
      enthält, bekommt das erste Element mit strukturellem Kasus 
      Nominativ.
\item Alle anderen Elemente der Liste, die strukturellen Kasus tragen, bekommen Akkusativ.
\end{itemize}
\end{prinzip-break}


}

\subsubsubsection{Aktiv}

\frame{
\frametitle{Aktiv}

prototypische Valenzlisten:
\ea
\begin{tabular}[t]{@{}l@{~}l@{~}l}
a. & \emph{schläft}:     & \comps \sliste{ NP[\type{str}]$_j$ }\\
b. & \emph{unterstützt}: & \comps \sliste{ NP[\type{str}]$_j$, NP[\type{str}]$_k$ }\\
c. & \emph{hilft}:       & \comps \sliste{ NP[\type{str}]$_j$, NP[\type{ldat}]$_k$ }\\
d. & \emph{schenkt}:     & \comps \sliste{ NP[\type{str}]$_j$, NP[\type{str}]$_k$, NP[\type{ldat}]$_l$ }\\
\end{tabular}
\z
\emph{str} steht für \emph{strukturell}, \emph{ldat} für lexikalischen Dativ.

\pause
Das erste Element in der \compsl bekommt Nominativ.\\
Alle anderen mit strukturellem Kasus bekommen Akkusativ.

}

\subsubsubsection{Kasusvergabe im Passiv}

\frame[shrink]{
\frametitle{Passiv}

\ea
\begin{tabular}[t]{@{}l@{~}l@{~}l}
a. & \emph{schläft}:     & \comps \sliste{ NP[\type{str}]$_j$ }\\
b. & \emph{unterstützt}: & \comps \sliste{ NP[\type{str}]$_j$, NP[\type{str}]$_k$ }\\
c. & \emph{hilft}:       & \comps \sliste{ NP[\type{str}]$_j$, NP[\type{ldat}]$_k$ }\\
d. & \emph{schenkt}:     & \comps \sliste{ NP[\type{str}]$_j$, NP[\type{str}]$_k$, NP[\type{ldat}]$_l$ }\\
\end{tabular}
\z

Bei Passivierung der Verben ergeben sich die folgenden \comps"=Listen:
\ea
\begin{tabular}[t]{@{}l@{~}l@{~}l}
a. & \emph{geschlafen wird}:  & \comps \sliste{ }\\
b. & \emph{unterstützt wird}: & \comps \sliste{ NP[\type{str}]$_k$ }\\
c. & \emph{geholfen wird}:    & \comps \sliste{ NP[\type{ldat}]$_k$ }\\
d. & \emph{geschenkt wird}:   & \comps \sliste{ NP[\type{str}]$_k$, NP[\type{ldat}]$_l$ }\\
\end{tabular}
\z
In (\mex{0}) steht jetzt eine andere NP an erster Stelle.\\
Wenn diese NP strukturellen Kasus hat, bekommt sie Nominativ,\\
wenn das wie in (\mex{0}c) nicht der Fall ist, bleibt der Kasus, wie er ist,\\
nämlich lexikalisch spezifiziert.
}

\subsection{Verbstellung}

\frame[shrink=5]{
\frametitle{\large Repräsentationen und Lexikonregeln: Verbbewegung}

~
\vfill
\hfill%
%\scalebox{0.85}{%
\begin{forest}
sm edges
[VP
	[V \sliste{ VP//V }, name=vini
	   [V,name=vlast [kennt$_j$]]]
	[VP//V, name=vp
	   [NP [jeder]]
	   [V$'$//V, name=vbar
	     [NP [diesen Roman, roof]]
		[V//V,name=vtrace [ \trace$_j$]]]]]
%\draw[<->] (vone) to (vtwo);
%%\draw (-2,-5) to[grid with coordinates] (4,0.5);
%% \draw[<-] (3,-3.4) .. controls (3.2,-3.6) .. (3.5,-3.4)
%%                    .. controls ()         .. (;
\draw[<->] ($(vtrace.south)+(-.25,.1)$)    to [bend right=45]  ($(vtrace.south)+(.25,.1)$);
\draw[<->] (vtrace)                        to [out=45, in=0]  (vbar);
\draw[<->] ($(vbar.north east)+(-0.2,0)$)  to [out=80, in=0]  (vp);
\draw[<->] ($(vp.north east)+(-0.25,-.1)$)  to [out=145,in=35] ($(vini.north east)+(-.5,-.1)$);
\draw[<->] ($(vini.south east)+(-.45,.1)$) to [bend left=30] ($(vlast.north east)+(-.1,-.1)$);
\end{forest}
%}
\hfill\hfill\mbox{}
\vfill

\begin{itemize}[<+->]
\item In Verberstsätzen steht in der Verbletztposition eine Spur.
\item In Verberststellung steht eine besondere Form des Verbs,\\
      die eine Projektion der Verbspur selegiert.
\item Dieser spezielle Lexikoneintrag ist durch eine Lexikonregel lizenziert.
\item Verbindung Verb/Spur durch Informationsweitergabe im Baum
\end{itemize}

}


\subsection{Lokale Umstellung}

\frame{
\frametitle{Lokale Umstellung}

Mehrere Möglichkeiten (zu Überblick siehe \citealt{MuellerOrder}):
\begin{itemize}
\item ganz flache Strukturen. Umordnung wie bei GPSG
\pause
\item binär verzweigende Strukturen,\\
      Abbindung der Argumente in beliebiger Reihenfolge
\pause
\item Lexikonregeln mit Umordnung der Elemente in Valenzlisten

\end{itemize}

}

\subsubsection{Binär verzweigende Strukturen}

\frame{
\frametitle{Beispiel: Normalabfolge}
\eal
\ex weil jeder diesen Roman kennt
\ex weil diesen Roman jeder kennt
\zl

\centerline{%
\begin{forest}
sm edges
[V{[\comps \sliste{}]}
	[\ibox{1} NP{[\type{nom}]}
		[jeder]]
	[V{[\comps \sliste{ \ibox{1} }]}
		[\ibox{2} NP{[\type{acc}]}
			[diesen Roman, roof]]
		[V{[\comps \sliste{ \iboxsp{1}, \ibox{2} }]}
			[kennt]]]]
\end{forest}
}%

}

\frame{
\frametitle{Beispiel: Umstellung}


\centerline{%
\begin{forest}
sm edges
[V{[\comps \sliste{}]}
	[\ibox{2} NP{[\type{acc}]}
		[diesen Roman, roof]]
	[V{[\comps \sliste{ \ibox{2} }]}
        	[\ibox{1} NP{[\type{nom}]}
	        	[jeder]]
		[V{[\comps \sliste{ \iboxsp{1}, \ibox{2} }]}
			[kennt]]]]
\end{forest}
}%

\medskip
Unterschied nur in Abbindungsreihenfolge der Elemente in \comps
}



\subsection{Fernabhängigkeiten}

\frame{

%\frametitle{Repräsentationen und Lexikonregeln: 
\frametitle{Konstituentenbewegung}

\vfill
\hfill%
\scalebox{0.7}{
\begin{forest}
sm edges
[VP
	[NP$_i$,name=np
		[diesen Roman,roof]]
	[VP/NP,name=vpnp2
		[V
			[V
				[kennt$_j$]]]
		[VP/NP,name=vpnp1
			[NP/NP, name=npnp
				[\trace$_i$]]
			[V$'$
				[NP
					[jeder]]
				[V
				  [\trace$_j$]]]]]]
\draw[<->] ($(npnp.east)$)  to [bend right=45] ($(vpnp1.south east)+(-.25,.1)$);
\draw[<->] ($(vpnp1.north east)+(-.26,-.1)$)  to [bend right=45] ($(vpnp2.east)+(-0,0)$);
\draw[<->] ($(vpnp2.north)+(.26,-0)$) parabola[parabola height=5mm] ($(np.north)+(-.15,0)$);
\end{forest}
}
\hfill\hfill\mbox{}
\vfill
\begin{itemize}[<+->]
\item Wie bei Verbbewegung: Spur an ursprünglicher "`normaler"' Position.
\item Weiterreichen der Information im Baum
\item Konstituentenbewegung ist nicht lokal, Verbbewegung ist lokal\\
      mit zwei verschiedenen Merkmalen modelliert ({\sc slash} vs.\ {\sc dsl})
\end{itemize}

}
