\iftoggle{gb-intro}{
\section{Government \& Binding}

%\if 0


\frame{
\frametitle{Gliederung}

\begin{itemize}
%\item Ziele
%\item Wiederholung von Grundbegriffen
%\item Grundfragen der Sprachwissenschaft
\item Grammatikmodelle
\begin{itemize}
\item Phrasenstrukturgrammatik
\item Transformationsgrammatik und deren Nachfolger
\begin{itemize}
\item \blaubf{Geschichtliches und Motivation}
\item Das T-Modell im Überblick
\item Grundbegriffe: Theta-Rollen, Externes Argument, \ldots
\item Lexikoneinträge
\item Syntaktische Kategorien
\item \xbar-Schemata
\item Die Struktur des deutschen Satzes
\item Kasus
\item NP-Bewegung
\item Bindungstheorie
\item \emph{w}-Bewegung
\item Inkorporation
\end{itemize}
\end{itemize}
\end{itemize}

}

}%\end{gb-intro}

\subsection{Geschichtliches und Motivation}

\frame{
\frametitle{Transformationsgrammatik}

\begin{itemize}
\item Noam Chomsky, MIT, argumentiert für Transformationen \citep{Chomsky57a}

\pause
\item Manfred Bierwisch (Dissertation Leipzig, 1960) ist der erste Deutsche, der
  transformationsgrammatische Ansätze verfolgt \citep{Bierwisch63a}

\pause
\item später an der Akademie der Wissenschaften der DDR
\pause
\item nicht immer einfach \ldots

\pause
\item Interessantes Gespräch mit Carla Umbach und Annette Leßmöllmann

\url{http://www.gespraech-manfred-bierwisch.de/}

\pause
\item West-Deutschland Lehrbuch 1971 \citep{BechertClementThummel1971a-u} 
%  und Gründung der Linguistischen
%  Berichte durch Peter Hartmann und Arnim von Stechow.

\pause
\item Transformationsgrammatik (heute Government \& Binding bzw. Minimalismus) ist weit verbreitet
      mit eigenen Tagungen, Zeitschriften, Buchreihen

\end{itemize}


}

\frame{
\frametitle{Phrasenstrukturgrammatiken und natürliche Sprachen}

Chomsky: Zusammenhänge zwischen bestimmten Sätzen (\zb Aktiv und Passiv) können nicht erfaßt
      werden. $\to$
      Transformationen:

\begin{table}[H]
\begin{tabular}{@{}l@{~}l@{~}l@{~}l}
NP& V &NP &$\to$ 3 [\sub{AUX} be] 2en [\sub{PP} [\sub{P} by] 1]\\
1 & 2 &3\\
\end{tabular}
\end{table}

\eal
\ex John loves Mary.
\ex Mary is loved by John.
\zl

Ein Baum mit der Symbolfolge auf der linken Regelseite wird auf einen Baum mit der Symbolfolge
auf der rechten Regelseite abgebildet.

}


\frame{
\frametitle{Transformation eines Aktivbaumes in einen Passivbaum}

\vfill

%\oneline{%
\hfill
\begin{forest}
%sm edges
[S, for tree={parent anchor=south, child anchor=north}
  [NP [John] ]
  [VP
    [V [loves] ]
    [NP [Mary] ] 
  ]]
\end{forest}
\hspace{1em}
\raisebox{6\baselineskip}{$\leadsto$}
\hspace{1em}
  \begin{forest}
  %sm edges
  [S, for tree={parent anchor=south, child anchor=north}
  	[NP[Mary]]
	[VP
	[Aux[is, tier=word]]
	[V[loved, tier=word]]
	[PP
	[P[by, tier=word]]
	[NP[John, tier=word]]]]]
\end{forest}
\hfill\mbox{}
%}

\vfill

\begin{tabular}{@{}l@{~}l@{~}l@{~}l}
NP& V &NP &$\to$ 3 [\sub{AUX} be] 2en [\sub{PP} [\sub{P} by] 1]\\
1 & 2 &3\\
\end{tabular}

\vfill

}


\iftoggle{einfsprachwiss-exclude}{
\frame{
\frametitle{Komplexität, Transformationen und natürliche Sprachen}

\small
\begin{itemize}
\item Es gibt Ersetzungsgrammatiken verschiedener Komplexität.\\(Chomsky-Hierarchie, Typ 3--0)
\pause
\item bisher behandelte so genannte kontextfreie Grammatiken sind vom Typ 2.
\pause
\item höchste Komplexitätsstufe (Typ 0) ist zu mächtig für natürliche Sprachen.\\
$\to$ Einschränkung nötig.
\pause
\item Transformationsgrammatiken entsprechen Typ-0-Phrasenstrukturgrammatiken hinsichtlich ihrer
      Komplexität \citep{PR73a-u}.
\pause
\item Transformationen sind nicht genügend restringiert,\\Interaktionen nicht überschaubar,\\
      Probleme mit Transformationen, die Material löschen (siehe \citew{Klenk2003a})
\pause
\item $\to$ neue theoretische Ansätze, Government \& Binding: Einschränkungen für Form der Grammatikregeln,
      Wiederauffindbarkeit von Elementen in Bäumen, allgemeine Prinzipien zur Beschränkung von Transformationen

\end{itemize}

}

\frame{
\frametitle{Hypothese zum Spracherwerb: Prinzipien \& Paramater}

\begin{itemize}
\item<+-> Ein Teil der sprachlichen Fähigkeiten ist angeboren.\\
          (Wird nicht von allen Linguisten geteilt! Diskussion: \citew{MuellerGTBuch2,MuellerGT-Eng1})
\item<+-> Prinzipien, die von sprachlichen Strukturen nicht verletzt werden dürfen.
\item<+-> Die Prinzipien sind parametrisiert, \dash es gibt Wahlmöglichkeiten.\\
      Ein Parameter kann für verschiedene Sprachen verschieden gesetzt sein.

\medskip
\pause
      Beispiel: \\

Prinzip: Ein Kopf steht in Abhängigkeit vom Parameter {\sc stellung}\\
vor oder nach seinen Komplementen.

\medskip
%% \begin{tabular}[t]{@{}l@{ }l@{}}
%%                 Englisch  &$\to$ Verb steht vor Komplementen\\
%%                 Japanisch &$\to$ Verb steht nach Komplementen\\
%%                 \end{tabular}


\eal
\ex 
\gll be showing pictures of himself\\
     ist zeigend Bilder von sich\\\jambox{(English)}
\ex
\gll zibun -no syasin-o mise-te iru\\
     sich  von Bild     zeigend sein\\\jambox{(Japanisch)}
\zl

\end{itemize}

}
}%\end{einfsprachwiss-exclude}

\lecture{T-Modell}{t-modell-lec}

\subsection{Das T-Modell}


\iftoggle{gb-intro}{%
\frame{

\frametitle{Gliederung}

\begin{itemize}
%% \item Ziele
%% \item Wiederholung von Grundbegriffen
%% \item Grundfragen der Sprachwissenschaft
\item Grammatikmodelle
\begin{itemize}
\item Phrasenstrukturgrammatik
\item Transformationsgrammatik und deren Nachfolger
\begin{itemize}
\item Geschichtliches und Motivation
\item \blaubf{Das T-Modell im Überblick}
\item Grundbegriffe: Theta-Rollen, Externes Argument, \ldots
\item Lexikoneinträge
\item Syntaktische Kategorien
\item \xbar-Schemata
\item Die Struktur des deutschen Satzes
\item Kasus
\item NP-Bewegung
\item Bindungstheorie
\item \emph{w}-Bewegung
\item Inkorporation
\end{itemize}
\end{itemize}
\end{itemize}

}

}%\end{gb-intro}

\frame{
\frametitle{Tiefen- und Oberflächenstruktur}

\begin{itemize}[<+->]
\item Chomsky hat behauptet, dass man mit einfachen PSGen gewisse Zusammenhänge nicht adäquat erfassen kann.\\
      \ZB den Zusammenhang zwischen Aktivsätzen und Passivsätzen.

\item Er nimmt deshalb eine zugrundeliegende Struktur an,\\
      die sogenannte \blaubf{Tiefenstruktur}.

\item Eine Struktur kann auf eine andere Struktur abgebildet werden.

Dabei können \zb Teile gelöscht oder umgestellt werden. 

Als Folge solcher Transformationen gelangt man
von der Tiefenstruktur zu einer neuen Struktur, der \blaubf{Oberflächenstruktur}.

\medskip
\begin{tabular}{@{}l@{ = }l@{}}
\emph{Surface Structure} & S-Struktur\\
\emph{Deep Structure} & D-Struktur\\
\end{tabular}
\end{itemize}

}

%\beamertemplatebackfindforwardnavigationsymbolshorizontal


\frame[label=t-modell]{
\frametitle{Das T-Modell}


%% \centerline{%
%% \resizebox{0.8\linewidth}{!}{
%% \begin{tabular}{@{}ccc@{}}
%% \xbar-Theorie der \\
%% \node{psr}{Phrasenstrukturregeln} & & \hyperlink{lexikon}{\node{lex}{Lexikon}}\\[6ex]
%% &\hyperlink{ds}{\node{ds}{D-Struktur}}\\[2ex]
%% \mc{3}{@{}c@{}}{\hyperlink{move-alpha}{move-$\alpha$}}\\[4ex]
%% &\hyperlink{ss}{\node{ss}{S-Struktur}}\\[6ex]
%% \node{tilg}{Tilgungsregeln},             && \node{ana}{Regeln des anaphorischen Bezugs,}\\
%% \node{filter}{Filter, phonol.\ Regeln}  && \node{quant}{der Quantifizierung und der Kontrolle}\\[6ex]
%% \hyperlink{pf}{\node{pf}{Phonetische}}             && \hyperlink{lf}{\node{lf}{Logische}}\\
%% \hyperlink{pf}{Form (PF)}               && \hyperlink{lf}{Form (LF)}\\
%% \end{tabular}
%% \anodeconnect{psr}{ds}\anodeconnect{lex}{ds}
%% \anodeconnect{ds}{ss}
%% \anodeconnect{ss}{tilg}\anodeconnect{ss}{ana}
%% \anodeconnect{filter}{pf}\anodeconnect{quant}{lf}
%% }}

\vfill
\centerline{%
\begin{forest}
for tree = {edge={->},l=4\baselineskip}
[D-structure
     [S-structure,edge label={node[midway,right]{move $\alpha$}} 
            [Deletion rules{,}\\Filter{,} phonol.\ rules
                    [Phonetic\\Form (PF)]]
            [Anaphoric rules{,}\\rules of quantification and control
                    [Logical\\Form (LF)]]]]
\end{forest}}

\vfill

}



\gotobuttonleft{t-modell}{T-Modell}

\frame[label=lexikon]{
\frametitle{Das T-Modell: Das Lexikon}
\showsingleitemframe
\savespace
\begin{itemize}
\item<+> Enthält zu jedem Wort einen Lexikoneintrag mit Information zu:
\begin{itemize}\itemsep0pt
\item morphophonologischer Struktur
\item syntaktischen Merkmalen
\item Selektionsraster (=~Valenzrahmen)
\item \ldots
\end{itemize}
Beinhaltet Wortformen- und Morphemliste sowie eine Wortbildungskomponente (=~Morphologie)

\item<+> Lexikon bildet Schnittstelle zur semantischen Interpretation\\
der einzelnen Wortformen.

\item<+> Wortschatz ist nicht von Universalgrammatik bestimmt,\\
nur bestimmte strukturelle Anforderungen sind prädeterminiert

\item<+> Morphosyntaktische Merkmale (\zb Genus) ebenfalls nicht vorbestimmt:
Universalgrammatik gibt nur Bandbreite von Möglichkeiten vor.
% und setzt einige Minimalanforderungen.
\end{itemize}

}

\frame{
\frametitle{Das T-Modell: D-Struktur, Move-$\alpha$ und S-Struktur (I)}

\begin{itemize}
\item<+-> Phrasenstruktur $\to$\\
Beschreibung der Beziehungen zwischen einzelnen Elementen mögl.

\item<+-> Gewisses Format für Regeln ist vorgegeben (\xbar-Schema).

Lexikon + Strukturen der \xbar-Syntax = Basis für die D-Struktur

\hypertarget{ds}{D-Struktur} = syntaktische Repräsentation der im Lexikon festgelegten
Selektionsraster (Valenzrahmen) der einzelnen Wortformen.
\end{itemize}


}

\frame{
\frametitle{Das T-Modell: D-Struktur, Move-$\alpha$ und S-Struktur (II)}

\begin{itemize}
\item<+-> Konstituenten stehen an der Oberfläche nicht unbedingt an der Stelle,
die der Valenzrahmen vorgibt:
\eal
\ex {}[dass] der Mann der Frau das Buch gibt
\ex Gibt der Mann der Frau das Buch?
\ex Der Mann gibt der Frau das Buch.
\zl
\item<+-> deshalb Transformationsregel für Umstellungen:\\
\hypertarget{move-alpha}{Move-$\alpha$} = "`Bewege irgendetwas irgendwohin!"'

Was genau wie und warum bewegt werden kann,\\
wird von Prinzipien geregelt.

\end{itemize}


}

\frame[label=ss]{
\frametitle{Das T-Modell: D-Struktur, Move-$\alpha$ und S-Struktur (III)}

\begin{itemize}
\item Von Lexikoneinträgen bestimmte Relationen zwischen einem Prädikat
und seinen Argumenten müssen für semantische Interpretation auf allen Repräsentationsebenen auffindbar sein.

\item
$\to$ Ausgangspositionen bewegter Elemente durch Spuren markiert.

\eal
\ex {}[dass] der Mann der Frau das Buch gibt
\ex Gibt$_i$ der Mann der Frau das Buch \_$_i$?
\ex {}[Der Mann]$_j$ gibt$_i$ \_$_j$ der Frau das Buch \_$_i$.
\zl

Verschiedene Spuren werden mit Indizes markiert.\\
Andere Darstellung: \emph{e} oder \emph{t}.

\item
\hypertarget{ss}{S-Struktur} ist eine oberflächennahe Struktur,\\
darf aber nicht mit real geäußerten Sätzen gleichgesetzt werden.
\end{itemize}
}


\frame[label=pf]{
\frametitle{Das T-Modell: Die Phonetische Form}

Auf PF werden phonologische Gesetzmäßigkeiten eines Satzes repräsentiert. 

Sie stellt den Output zum Sprechmodul dar.

\pause
Beispiel: \emph{wanna}"=Kontraktion

\eal
\ex The students want to visit Paris.
\ex The students wanna visit Paris.
\zl
Die Kontraktion in (\mex{0}) wird durch die optionale Regel in (\mex{1}) lizenziert:
\ea
want + to $\to$ wanna
\z


}

\frame[label=lf]{
\frametitle{Das T-Modell: Die Logische Form (I)}

\begin{itemize}
\item Logische Form ist eine syntaktische Ebene, die zwischen der S-Struktur und der
semantischen Interpretation eines Satzes vermittelt.

anaphorischer Bezug (Bindung): Worauf bezieht sich ein Pronomen?
\eal
\ex Peter kauft einen Tisch. Er gefällt ihm.
\ex Peter kauft eine Tasche. Er gefällt ihm.
\ex Peter kauft eine Tasche. Er gefällt sich.
\zl

\pause
\item Quantifizierung:
\ea
Every man loves a woman.
\z
$\forall x \exists y (man(x) \to (woman(y) \wedge love(x,y))$\\
$\exists y \forall x (man(x) \to (woman(y) \wedge love(x,y))$
\end{itemize}

}

\frame{
\frametitle{Das T-Modell: Die Logische Form (II)}

Kontrolltheorie:\\
Wodurch wird die semantische Rolle des Infinitiv-Subjekts gefüllt?

\eal
\ex Der Professor schlägt dem Studenten vor,\\die Klausur noch mal zu schreiben.
\pause
\ex Der Professor schlägt dem Studenten vor,\\die Klausur nicht zu bewerten.
\pause
\ex Der Professor schlägt dem Studenten vor,\\gemeinsam ins Kino zu gehen.
\zl

}

%\beamertemplatenavigationsymbolsempty 
\mode<beamer>{\beamertemplatebackfindforwardnavigationsymbolshorizontal}

\subsection{Das Lexikon}

\frame{
\frametitle{Lexikon: Grundbegriffe (I)}

\small
\begin{itemize}
\item Bedeutung von Wörtern $\to$ Phrasenbildung mit bestimmten Rollen ("`handelnde Person"'
      oder "`betroffene Sache"')

      Beispiel: semantischer Beitrag von \emph{kennen} in (\mex{1}a) ist (\mex{1}b):
      \eal
\ex Maria kennt den Mann.
\ex kennen'(x,y)
\zl
\pause
\item Solche Beziehungen werden mit dem Begriff \blaubf{Selektion} bzw.\ \blaubf{Valenz} erfaßt.

Achtung:\\
Logische Valenz kann sich von syntaktischer Valenz unterscheiden!
\pause
\item Man spricht auch von \blaubf{Subkategorisierung}:

\emph{kennen} ist für ein Subjekt und ein Objekt subkategorisiert.

Das Wort \emph{subkategorisieren} hat sich verselbständigt, auch wie folgt gebraucht:

\emph{kennen} subkategorisiert für ein Subjekt und ein Objekt.
\end{itemize}


}

\lecture{Lexikon}{lexikon}

\frame{
\frametitle{Lexikon: Grundbegriffe (II)}
\savespace
\begin{itemize}
\item \emph{kennen} wird auch \blaubf{Prädikat} genannt\\
(weil \emph{kennen'} das logische Prädikat ist).

Vorsicht:\\
entspricht nicht der Verwendung des Begriffs in der Schulgrammatik.
\pause
\item Subjekt und Objekt sind die \blaubf{Argumente} des Prädikats.
\pause
\item Spricht man von der Gesamtheit der Selektionsanforderungen, verwendet man
Begriffe wie \blaubf{Argumentstruktur}, \blaubf{Valenzrahmen}, \blaubf{Selektionsraster},
\blaubf{Subkategorisierungsrahmen}, \blaubf{thematisches Raster} oder
\blaubf{Theta-Raster} = \blaubf{$\theta$-Raster}
(\emph{thematic grid}, \emph{Theta-grid})
\pause
\item \blaubf{Adjunkte} (oder Angaben) modifizieren semantische Prädikate,\\
man spricht auch von Modifikatoren.\\
Adjunkte sind in Argumentstrukturen von Prädikaten nicht vorangelegt.
\end{itemize}



}



\frame{
\frametitle{Das Theta-Kriterium}

\label{theta-kriterium}%
Argumente nehmen im Satz typische Positionen (Argumentpositionen) ein.

Theta-Kriterium:
\begin{itemize}
\item Jede Theta-Rolle wird an genau eine Argumentposition vergeben.
\item Jede Phrase an einer Argumentposition bekommt genau eine Theta-Rolle.
\end{itemize}

}


\frame{
\frametitle{Externes Argument und interne Argumente}

\savespace
\begin{itemize}[<+->]
\item Argumente stehen in Rangordnung, \dash, man kann zwischen ranghohen und rangniedrigen Argumenten unterscheiden.

\item Ranghöchstes Argument von V und A hat besonderen Status. 

Weil es oft (im manchen Sprachen: immer) an einer Position außerhalb der Verbal- bzw.\ Adjektivphrase steht,\\
wird es auch als \blaubf{\hypertarget{ext-arg}{externes Argument}} bezeichnet. 

\item Die übrigen Argumente stehen an Positionen\\
      innerhalb der Verbal- bzw. Adjektivphrasen.

Bezeichnung: \blaubf{internes Argument} oder \blaubf{Komplement}

\item Faustregel für einfache Sätze: Externes Argument = Subjekt.
\end{itemize}

}

\frame{
\frametitle{Einzelne Theta-Rollen}
%\savespace
\begin{itemize}
\item<+-> Wenn es sich bei den Argumenten um Aktanten handelt,\\
kann man drei Klassen von Theta-Rollen unterscheiden. 

\item<+-> Wenn ein Verb mehrere Theta-Rollen dieser Art vergibt,\\
hat Klasse 1 gewöhnlich den höchsten Rang, Klasse 3 den niedrigsten:   

\begin{itemize}
\item Klasse 1: \blaubf{Agens} (handelnde Person), Auslöser eines Vorgangs oder Auslöser einer Empfindung (Stimulus), \blaubf{Träger einer Eigenschaft}
\item<+-> Klasse 2: \blaubf{Experiencer} (wahrnehmende Person), \blaubf{nutznießende Person} (Benefaktiv) (oder auch das Gegenteil: vom einem Schaden betroffene Person), \blaubf{Possessor} (Besitzer, Besitzergreifer   oder auch das Gegenteil: Person, der etwas abhanden kommt oder fehlt)
\item<+-> Klasse 3: \blaubf{Patiens} (betroffene Person oder Sache), \blaubf{Thema}
\end{itemize}

\item<+-> Vorsicht!\\
großes Wirrwarr bei Zuordnungen von semantischen Rollen zu Verben
\nocite{Gruber65a-u,Fillmore68,Fillmore71a-u,Jackendoff72a-u,Dowty91a}
\end{itemize}
}

\iftoggle{gb-intro}{
\frame{
\frametitle{Theta-Rolle und syntaktische Kategorie (I)}


\begin{itemize}
\item Argumente meist als \blaubf{Nominalphrasen} realisiert:
\ea
{}[\sub{NP} Der Beamte] verlangte [\sub{NP} einen schriftlichen Antrag].
\z
\pause
\item bei passender Theta-Rolle auch als \blaubf{Nebensatz}:
\eal
\ex Der Beamte verlangte,\\
    {}[dass der Antrag schriftlich eingereicht wird].
\ex Der Beamte verlangte,\\
    {}[den Antrag schriftlich einzureichen].
\zl
\pause
\item oder als \hyperlink{sc}{\blaubf{Small Clause}} = satzwertige Fügung ohne Verb,\\
      bestehend aus Nominalphrase + Prädikativ
\eal
\ex Anna findet, [dass das Häschen niedlich ist]. (Nebensatz)
\ex Anna findet [das Häschen niedlich]. (Small Clause)
\zl
\eal
\ex Otto macht, [dass der Tisch sauber wird]. (Nebensatz)
\ex Otto macht [den Tisch sauber]. (Small Clause)
\zl
\end{itemize}

}

\frame{
\frametitle{Theta-Rolle und syntaktische Kategorie (II)}

Angabe der syntaktischen Kategorie eines Arguments (NP, Nebensatz%
%\iftoggle{gb-intro}{, Small Clause}
) 
im Theta-Raster meist unnötig ($\to$ freie Wahl).

\medskip
Allerdings:
\eal
\ex[]{
Er weiß, dass Peter kommt.
}
\ex[*]{
Er kennt, dass Peter kommt.
}
\ex[]{
Er kennt die Vermutung, dass Peter kommt.
}
\zl

}
} % gb-intro


\frame{
\frametitle{Ein Lexikoneintrag (I)}

\small
Über welche Information muss man verfügen, um ein Wort sinnvoll anzuwenden? 

Antwort: Das mentale Lexikon enthält Lexikoneinträge (englisch: \emph{lexical entries}),\\ 
in denen die spezifischen Eigenschaften der syntaktischen Wörter aufgeführt sind:  

\begin{itemize}
\item Form
\item Bedeutung (Semantik)
\item Grammatische Merkmale:\\
      syntaktische Wortart + morphosyntaktische Merkmale   
\item Theta-Raster
\end{itemize}



}

\frame{
\frametitle{Ein Lexikoneintrag (II)}


{\footnotesize
\begin{tabular}{|l|ll|}
\hline
Form     & \emph{hilft}&\\\hline
Semantik & helfen'     &\\\hline
Grammatische Merkmale & \multicolumn{2}{l|}{Verb, 3.\ Person Singular Indikativ Präsens Aktiv}\\\hline\hline
%\setlength{\arrayrulewidth}{9pt}
Theta-Raster                &&\\\hline
Theta-Rollen                & \underline{Agens} & Benefaktiv\\[2mm]\hline
Grammatische Besonderheiten &                   & Dativ\\\hline
\end{tabular}
}

\bigskip
Argumente sind nach dem Rang geordnet:\\
ranghöchstes Argument steht ganz links. 

In diesem Fall ist das ranghöchste Argument das externe Argument.

Das externe Argument wird unterstrichen.

}

\iftoggle{gb-intro}{
\frame{
\frametitle{Ein Lexikoneintrag (III)}


\begin{itemize}[<+->]
\item Bei den grammatischen Besonderheiten wird nur angegeben,\\
was nicht von allgemeinen Regeln abgeleitet werden kann,\\
also besonders zu lernen ist. 

\item \emph{helfen} $\to$ Kasus des internen Arguments = Dativ\\
(Zweiwertige Verben haben sonst internes Argument im Akkusativ)

\item Kasus des externen Arguments folgt aus allgemeinen Regeln:\\
In einfachen Sätzen erscheint es als Subjekt und erhält Nominativ. 

\item Argumente von \emph{helfen} werden gewöhnlich als NPen realisiert.
\end{itemize}
}
}% gb-intro


\lecture{\xbar-Theorie}{x-bar}
\subsection{\xbar-Theorie}


\iftoggle{gb-intro}{
\frame{
\frametitle{Gliederung}

\begin{itemize}
%% \item Ziele
%% \item Wiederholung von Grundbegriffen
%% \item Grundfragen der Sprachwissenschaft
\item Grammatikmodelle
\begin{itemize}
\item Phrasenstrukturgrammatik
\item Transformationsgrammatik und deren Nachfolger
\begin{itemize}
\item Geschichtliches und Motivation
\item Das T-Modell im Überblick
\item Grundbegriffe: Theta-Rollen, Externes Argument, \ldots
\item Lexikoneinträge
\item Syntaktische Kategorien
\item \blaubf{\xbar-Schemata}
\item Die Struktur des deutschen Satzes
\item Kasus
\item NP-Bewegung
\item Bindungstheorie
\item \emph{w}-Bewegung
\item Inkorporation
\end{itemize}
\end{itemize}
\end{itemize}

}
}%\end{gb-intro}


\frame{

\frametitle{Anmerkung zur Verbreitung von \xbar-Regeln}

\xbar-Theorie wird auch in vielen anderen Frameworks angenommen:\\
\begin{itemize}
%\item Government \& Binding (GB):\\\citew*{Chomsky93a}
\item Lexical Functional Grammar (LFG):\\\citew{Bresnan82a-ed,Bresnan2001a,BF96a-ed,Berman2003a}
\item Generalized Phrase Structure Grammar (GPSG):\\ \citew*{GKPS85a}
\end{itemize}

Es wird nicht unbedingt dasselbe Kategorieninventar benutzt.\\
Insbesondere bei sogenannten funktionalen Kategorien (\zb INFL).

}

\subsubsection{Köpfe}


\frame{
\frametitle{\xbar-Theorie: Köpfe}
\pause
Kopf bestimmt die wichtigsten Eigenschaften\\
einer Wortgruppe/Phrase/Projektion
\eal
\ex Karl \blaubf{schläft}.
\ex Karl \blaubf{liebt} Maria.
\ex \blaubf{in} diesem Haus
\ex ein \blaubf{Mann}
\zl


}

\subsubsection{Lexikalische Kategorien}

\frame{
\frametitle{\xbar-Theorie: Lexikalische Kategorien}
\label{slide-lex-kat-gb}%

Untereinteilung in lexikalische und funktionale Kategorien\\ 
($\approx$ Unterscheidung zwischen offenen und geschlossenen Wortklassen)

Lexikalische Kategorien: 
\begin{itemize}
\item V = Verb
\item N = Nomen
\item A = Adjektiv
\item P = Präposition
\item Adv = Adverb
\end{itemize}
% Abney87a:64--65 Meinunger2000a:38--39

}

\iftoggle{einfsprachwiss-exclude}{
\frame[shrink=10]{
\frametitle{\xbar-Theorie: Lexikalische Kategorien (Kreuzklassifikation)}
%\label{slide-lex-kat-gb}%


Versuch, die lexikalischen Kategorien mittels Kreuzklassifikation auf elementarere Merkmale zurückzuführen:

\bigskip

\centerline{\renewcommand{\arraystretch}{1.5}
\begin{tabular}{|c|c|c|}\hline
 & $-$ V & + V \\\hline
$-$ N & P = [ $-$ N, $-$ V] &  V = [ $-$ N, + V] \\\hline
  + N & N = [+ N, $-$ V]    &  A = [+ N, + V]\\\hline
\end{tabular}
}
\pause

\bigskip

Kreuzklassifikation $\to$ einfach auf Adjektive und Verben referieren:\\
Alle lexikalischen Kategorien, die [ + V] sind,\\
sind entweder Adjektive oder Verben.

Generalisierungen mgl.\ \zb: [ + N]-Kategorien können Kasus tragen 
% [ $-$ N]-Kategorien können Kasus regieren (allerdings auch A)
\medskip

Anmerkung: Adverbien können als einstellige Präpositionen behandelt werden.\nocite{Chomsky70a}

}



\frame{
\frametitle{Kopfposition in Abhängigkeit von Kategorie}

Bei Präpositionen und Nomina steht der Kopf vorn:
\eal
\ex \gruen{für} Marie
\ex \gruen{Bild} von Maria
\zl

Bei Adjektiven und Verben steht er hinten:
\eal
\ex dem König \gruen{treu}
\ex dem Mann \gruen{helfen}
\zl

\pause

$\to$ \begin{tabular}[t]{@{}l@{}}
      {}[+ V] $\equiv$ Kopf hinten\\
      {}[$-$ V] $\equiv$ Kopf vorn\\
      \end{tabular}

\pause
Problem: Postpositionen (P = [$-$ V])
\ea
des Geldes wegen
\z

}
}%\end{einfsprachwiss-exclude}

\iftoggle{gb-intro}{
\frame{

\frametitle{Kasuszuweisung in Abhängigkeit von Kategorie}

\citet[S.\,48]{Chomsky93a}:
Im Englischen haben nur Präpositionen und Verben kasustragende Argumente.

Nur [$-$ N]-Kategorien weisen im Englischen Kasus zu.

\pause
Wie ist das im Deutschen?
\pause
\ea
dem König treu
\z
\pause
Trotzdem will man mitunter Teilklassen herausgreifen,\\
um etwas über sie zu sagen.
\pause
Oder eventuell: [$-$ N]-Kategorien weisen in allen Sprachen Kasus zu,
in manchen Sprachen kommen noch weitere Kategorien hinzu.

}

\frame{
\frametitle{Anmerkung zu Kreuzklassifikationen}

\begin{itemize}
\item Beschreibungen mittels Merkmalen, die +/$-$ als Wert haben,\\
machen Vorhersagen über mögliche Wertbelegungen.

Zwei Merkmale $\to$ vier Kombinationen,\\
Drei Merkmale $\to$ acht Kombinationen

\item Wenn Wertkombinationen nicht belegt sind,\\
sind binäre Merkmale nicht geeignet.

\end{itemize}

}
}%\end{gb-intro}

\subsubsection{Funktionale Kategorien}

\frame{
\frametitle{\xbar-Theorie: Funktionale Kategorien}

%\begin{gb-intro}
Keine Kreuzklassifikation:\bigskip
%\end{gb-intro}


\begin{tabular}{lp{\linewidth}@{}}
C   & COMP = complementizer\\[\baselineskip]
I   & Finitheit (sowie Tempus und Modus);\\
    & in älteren Arbeiten auch INFL (engl. inflection = Flexion),\\
    & in neueren Arbeiten auch T (Tempus) \\[\baselineskip]
\iftoggle{gb-intro}{
Agr & Agreement (Übereinstimmung, Kongruenz)\\[\baselineskip]
}
D   & Determinierer (Artikelwort)\\[\baselineskip]
\end{tabular}


}



\subsubsection{Annahmen und Regeln}


\frame{
\frametitle{\xbar-Theorie: Annahmen (I)}

\begin{itemize}
\item \blaubf{Endozentrizität}:\\
Jede Phrase hat einen Kopf,\\
und jeder Kopf ist in eine Phrase eingebettet.\\
(fachsprachlich: Jeder Kopf wird zu einer Phrase projiziert.)\\
Phrase und Kopf haben die gleiche syntaktische Kategorie. 

\iftoggle{gb-intro}{
\pause
\item \blaubf{Binarität} als heute vorherrschende Annahme:\nocite{Kayne84a-u}\\
Phrasenstrukturen verzweigen binär,\\
\dash, es gibt keine Drei- oder Mehrfachverzweigungen. 
}
\pause
\item Die Äste von Baumstrukturen können sich nicht überkreuzen.\\
(\emph{Non-Tangling Condition})
\end{itemize}

}

% \frame{
% \frametitle{\xbar-Theorie: Annahmen (II)}

% Phrasen sind mindestens dreistöckig:
% \begin{itemize}
% \item X$^0$ = Kopf
% \item X' = Zwischenebene (X-Bar, X-Strich; $\to$ Name des Schemas) 
% \item XP = oberster Knoten (=~X'' = $\overline{\overline{\mbox{X}}}$), auch Maximalprojektion genannt
% \end{itemize}
% Neuere Analysen $\to$ teilweise Verzicht auf nichtverzweigende X'-Knoten
% \nocite{Muysken82a}


% }

% \frame[shrink]{
% \frametitle{Minimaler und maximaler Ausbau von Phrasen}

% \bigskip

% \small\centerline{\begin{tabular}[t]{c}
% \node{xp}{XP}\\[5ex]
% \node{xs}{X'}\\[5ex]
% \node{x}{X}\\
% \end{tabular}%
% \nodeconnect{xp}{xs}\nodeconnect{xs}{x}\hspace{10ex}%
% \begin{tabular}[t]{cccc}
% \multicolumn{2}{c}{\hspace{18mm}\node{xp2}{XP}}\\[5ex]
% \node{spec}{Spezifikator} & \multicolumn{2}{c}{\node{xs2}{X'}}\\[5ex]
%                           & \node{adj}{Adjunkt} & \multicolumn{2}{c}{~~~~~\node{xs22}{X'}}\\[5ex]
%                           &                     & \node{comp}{Komplement} & \node{x2}{X}\\
% \end{tabular}
% \nodeconnect{xp2}{spec}\nodeconnect{xp2}{xs2}%
% \nodeconnect{xs2}{adj}\nodeconnect{xs2}{xs22}%
% \nodeconnect{xs22}{comp}\nodeconnect{xs22}{x2}}


% \begin{itemize}
% \item Adjunkte sind optional\\
% $\to$ muss nicht unbedingt ein X' mit Adjunkttochter geben.
% \pause
% \item Für manche Kategorien gibt es keinen Spezifikator, oder er ist optional.\\
% %(Zusätzliche Regel nötig $\overline{\overline{\mbox{X}}} \rightarrow \xbar$)
% \pause
% \item zusätzlich mitunter: Adjunkte an XP und Kopfadjunkte an X. 
% \ifthenelse{\boolean{gb-intro}}{
% (dazu \hyperlink{inkorporation}{später})
% }{}
% \end{itemize}

% }


% \frame{
% \frametitle{\xbar-Theorie: Regeln nach \citep{Jackendoff77a}}\nocite{KP90a}\nocite{Pullum85a}



% \oneline{\(
% \begin{array}{@{}l@{\hspace{1cm}}l@{\hspace{1cm}}l}
% \xbar\mbox{-Regel} & \mbox{mit Kategorien} & \mbox{Beispiel}\\[2mm]
% \overline{\overline{\mbox{X}}} \rightarrow \overline{\overline{\mbox{Spezifikator}}}~~\xbar & \overline{\overline{\mbox{N}}} \rightarrow \overline{\overline{\mbox{DET}}}~~\overline{\mbox{N}} & \mbox{das [Bild von Maria]} \\
% \xbar \rightarrow \xbar~~\overline{\overline{\mbox{Adjunkt}}}             & \overline{\mbox{N}} \rightarrow \overline{\mbox{N}}~~\overline{\overline{\mbox{REL\_SATZ}}} & \mbox{[Bild von Maria] [das alle kennen]}\\
% \xbar \rightarrow \overline{\overline{\mbox{Adjunkt}}}~~\xbar             & \overline{\mbox{N}} \rightarrow \overline{\overline{\mbox{ADJ}}}~~\overline{\mbox{N}} & \mbox{schöne [Bild von Maria]}\\
% \xbar \rightarrow \mbox{X}~~\overline{\overline{\mbox{Komplement}}}*               & \overline{\mbox{N}} \rightarrow \mbox{N}~~\overline{\overline{\mbox{P}}} & \mbox{Bild [von Maria]}\\\\
% \end{array}
% \)}

% X steht für beliebige Kategorie, X ist Kopf,\\
% `*' steht für beliebig viele Wiederholungen

% \medskip
% X kann links oder rechts in Regeln stehen

% }

%\fi

\iftoggle{gb-intro}{
\subsubsection{Verbalphrasen}


\frame{
\frametitle{Verbalphrasen: Externes Argument und Spezifikator}
\savespace
\begin{itemize}
\item ranghöchstes Verbargument hat besonderen Status.\\
      (in einfachen Sätzen = Subjekt)
\begin{itemize}
\item Standardannahme: ranghöchstes Argument steht immer außerhalb der VP ($\to$~\hyperlink{ext-arg}{externes Argument}).
VP hat keinen Spezifikator.
\pause
\item neuere Arbeiten: Subjekt wird zunächst als VP-Spezifikator generiert.\nocite{FS86a-u,KS91a-u} 
In einigen Sprachen wird es von dort aus immer an eine Position außerhalb der VP angehoben, in anderen Sprachen, so auch im Deutschen, zumindest unter bestimmten Bedingungen (zum Beispiel bei Definitheit).\nocite{Diesing92a}
\end{itemize}
\pause
Wir folgen der Standardannahme.
%\iftoggle{{gb-intro}{\\
%Einzelheiten und Besonderheiten (Passiv, nichtakkusativische V)  später.}
\pause
\item Übrige Argumente sind Komplemente der VP (=~interne Argumente).

Wenn Verb ein einziges Komplement verlangt,\\
ist dieses nach \xbar-Schema Schwester des Kopfes V$^0$ und Tochter von V'.\\
Prototypisches Komplement: Akkusativobjekt.
\end{itemize}


}



\frame[shrink=13]{
\frametitle{Verbalphrasen: Adjunkte}

Adjunkte (freie Angaben) zweigen entsprechend dem \xbar-Schema oberhalb der Komplemente von V' ab.

\ea
weil der Mann morgen den Jungen trifft
\z

{\small\begin{tabular}{ccc}
\multicolumn{2}{c}{\node{vp}{VP}}\\[4ex]
\multicolumn{2}{c}{\node{vs}{V'}}\\[4ex]
\node{adv}{Adv} & \multicolumn{2}{c}{\node{vs2}{V'}}\\[4ex]
                & \node{np}{NP} & \node{v}{V}\\[5ex]
\node{nat}{morgen} & de\node{dj}{n Jung}en    & \node{kennt}{trifft}\\
\end{tabular}
\nodeconnect{vp}{vs}%
\nodeconnect{vs}{adv}\nodeconnect{vs}{vs2}%
\nodeconnect{adv}{nat}%
\nodeconnect{vs2}{np}\nodeconnect{vs2}{v}%
\nodetriangle{np}{dj}%
\nodeconnect{v}{kennt}
}



}



\frame{
\frametitle{Verbalphrasen: Dreistellige Verben (I)}


Was passiert mit dreistelligen Verben (Verben mit zwei Komplementen)?
\ea
als Anna ihrer Freundin den Brief zeigte
\z
\pause
{\small\begin{tabular}{ccc}
\multicolumn{3}{c}{\node{vpd}{VP}}\\[4ex]
\multicolumn{3}{c}{\node{vsd}{V'}}\\[6ex]
\node{npd}{NP} & \node{npd2}{NP} & \node{vd}{V}\\[5ex]
ihr\node{if}{er Freun}din & de\node{db}{n Bri}ef    & \node{zeigte}{zeigte}\\
\end{tabular}
\nodeconnect{vpd}{vsd}%
\nodeconnect{vsd}{npd}\nodeconnect{vsd}{npd2}\nodeconnect{vsd}{vd}%
\nodetriangle{npd}{if}%
\nodetriangle{npd2}{db}%
\nodeconnect{vd}{zeigte}}

\vfill

Struktur ist nicht binär verzweigend.


}

\frame{
\frametitle{Verbalphrasen: Dreistellige Verben (II)}


Alternative: binär verzweigende Strukturen
%% \ea
%% als Anna ihrer Freundin den Brief zeigte.
%% \z

\begin{tabular}{ccc}
\multicolumn{2}{c}{\node{vpd3}{VP}}\\[4ex]
\multicolumn{2}{c}{\node{vsd3}{V'}}\\[4ex]
\node{npd3}{NP} & \multicolumn{2}{c}{\node{vsd32}{V'}}\\[4ex]
                & \node{npd32}{NP} & \node{vd3}{V}\\[5ex]
ih\node{if3}{rer Freund}in & de\node{db3}{n Bri}ef    & \node{zeigte3}{zeigte}\\
\end{tabular}
\nodeconnect{vpd3}{vsd3}%
\nodeconnect{vsd3}{npd3}\nodeconnect{vsd3}{vsd32}%
\nodetriangle{npd3}{if3}%
\nodeconnect{vsd32}{npd32}\nodeconnect{vsd32}{vd3}%
\nodetriangle{npd32}{db3}%
\nodeconnect{vd3}{zeigte3}

}

\frame{
\frametitle{Verbalphrasen: Binarität}

Binär verzweigende Strukturen:
\begin{itemize}
\item brauchen zusätzliche Regeln für Rekursion mit X' und Argumenten\\
brauchen Mechanismus, der sicherstellt,\\
dass Adjunkte nach Komplementen mit Köpfen verbunden werden.
\pause
\item alternativ:\\
      zusätzliche Kategorien, die helfen, das \xbar-Schema einzuhalten\\
      Stichwort VP-Shell-Analyse \citep{Larson88a-u}
\end{itemize}

}

\frame{
\frametitle{Binär verzweigende vs.\ flache Strukturen und Lernbarkeit}

Die Argumentation in \citew[Kapitel~2.5]{Haegeman94a-u} ist dubios.

Manche der betrachteten Strukturen scheiden schon aus semantischen Gründen aus.

Mit Lernbarkeitsargumenten wird viel Schindluder getrieben.

%% \pause
%% Beispiel für Lernbarkeitsargumentation:\\

%% Daten X, Y, Z sind Evidenz dafür, dass Sprache A eine VP hat.\\
%% Sprecher der Sprache A haben keine für sie zugängliche Evidenz im Input,
%% die es ermöglicht, diese Tatsache zu lernen. $\to$ Existenz der VP
%% ist Bestandteil der Universalgrammatik und damit angeboren. $\to$
%% alle Sprachen besitzen eine VP.
%%
%% Fanselow87a:Chapter 1




}



\subsubsection{Nominalphrasen}

\frame{
\frametitle{Nominalphrasen: Spezifikatoren und Adjunkte}

\begin{columns}

\column{80mm}
Artikelwörter (Determinierer) sind Spezifikatoren der NP. 

\bigskip
Etikettierung der Kategorie in der Fachliteratur:\\
D (Determinierer) oder Art (Artikel). 

\bigskip
Attributive Adjektivphrasen (AP) sind Adjunkte; sie werden normalerweise links angefügt:

\bigskip
\uncover<2->{
Relativsätze sind ebenfalls Adjunkte, stehen aber rechts vom Kopf N$^0$.
}
\column{40mm}
\only<1->{
{\small%
\begin{tabular}{cccc}
\multicolumn{3}{c}{\node{np}{NP}}\\[4ex]
\node{dp}{DP}  & \multicolumn{3}{c}{\node{ns}{N'}~~~~}\\[4ex]
\node{ds}{D'}  & \node{ap}{AP}    & \multicolumn{2}{c}{\node{ns2}{N'}}\\[4ex]
\node{d}{\dnull}    & \node{as}{A'}    & \node{ap2}{AP}   & \node{ns3}{N'}\\[4ex]
               & \node{a}{\anull}      & \node{as2}{A'}   & \node{n}{\nnull}   \\[4ex]
               &                  & \node{a2}{\anull}     &               \\[4ex]
\node{das}{das} & \node{dicke}{dicke} & \node{alte}{alte} & \node{buch}{Buch}\\
\end{tabular}%
\nodeconnect{np}{dp}\nodeconnect{np}{ns}%
\nodeconnect{dp}{ds}\nodeconnect{ds}{d}%
\nodeconnect{ns}{ap}\nodeconnect{ns}{ns2}%
\nodeconnect{ap}{as}\nodeconnect{as}{a}%
\nodeconnect{ns2}{ap2}\nodeconnect{ns2}{ns3}\nodeconnect{ns3}{n}%
\nodeconnect{ap2}{as2}\nodeconnect{as2}{a2}%
\nodeconnect{d}{das}\nodeconnect{a}{dicke}\nodeconnect{a2}{alte}\nodeconnect{n}{buch}%
}%
}
\end{columns}
\pause

}

\frame{
\frametitle{Nominalphrasen: Genitivattribute (I)}

\begin{itemize}
\item<+-> Nominalphrasen im Genitiv (Genitivattribute): 
verschiedene Typen: 
\begin{itemize}
\item Genitivus possessivus
\item Genitivus subjectivus
\item Genitivus  objectivus 
\end{itemize}
In (\mex{1}) Subjekt- und Objekt-Lesart möglich:
\ea
Die aus Angst um ihre Sicherheit und die ihrer Familie zurückgetretenen Journalisten berichten von \blauit{Einschüchterungen Pekinger Politiker und prochinesischer Kreise} in Hongkong.\footnote{taz, 29.05.2004, S.\,11}
\z
Durch Kontext klar: Subjektlesart
\item<+-> erscheinen im Deutschen an zwei Positionen:\\
          vor dem Nomen (=~pränominal) und danach (=~postnominal). 

\eal
\ex {}[des Kaisers] neue Kleider
\ex {}die neuen Kleider [des Kaisers] 
\zl
\end{itemize}

}

\frame{
\frametitle{Nominalphrasen: Genitivattribute (II)}

\begin{itemize}
\item<+-> pränominaler Genitiv tritt nie zusammen mit Artikelwort auf

\item<+-> pränominaler Genitiv legt übergeordnete NP in Definitheit fest,\\
verhält sich also in dieser Hinsicht wie ein definiter Artikel. 

$\to$ Annahme: solche NPs nehmen ebenfalls die Spezifikatorposition ein.

\item<+-> Umstritten ist, ob dies die "`Originalposition"' der Genitivphrase ist (Genitiv-NP dort basisgeneriert) oder ob sie dorthin bewegt worden ist. 

\bigskip
\item<+-> Postnominale Genitivphrasen sind teils Komplemente (=~valenzbedingt), teils Adjunkte.
\eal
\ex der Vater des Jungen (Komplement)
\ex die Konstruktion einer Vertikalsonnenuhr (Komplement)
\ex der Mantel des Jungen (Adjunkt)
\zl
\end{itemize}

}

\frame{
\frametitle{Nominalphrasen: NP (III)}

Präpositionalphrasen als Attribute von Nomen sind Komplement (\mex{1}) oder Adjunkt (\mex{2}): 
\eal
\ex die Freude [über den Erfolg]
\ex der Anteil [am Erfolg] 
\zl
\eal
\ex die Brücke [über die Weser] 
\ex die Sitzung [am Freitag]
\zl

}

\frame{
\frametitle{Nominalphrasen: DP (I)}

\begin{columns}

\column{95mm}
Alternativer Ansatz (sehr oft in neueren Arbeiten):\\
Nominalphrasen sind in eine funktionale Kategorie DP eingebettet;
sie sind Komplement des Kopfes D.\nocite{Abney87a-u}

\pause
\bigskip
Der Artikel wird entweder mit D identifiziert:

\column{25mm}
{\small
\begin{tabular}{cc}
\multicolumn{2}{c}{\node{dp}{DP}}\\[4ex]
\multicolumn{2}{c}{\node{ds}{D'}}\\[4ex]
\node{d}{\dnull} & \node{np}{NP}\\[4ex]
  & \node{ns}{N'}\\[4ex]
  & \node{n}{\nnull}\\[4ex]
\node{das}{das} & \node{buch}{Buch}\\
\end{tabular}
\nodeconnect{dp}{ds}%
\nodeconnect{ds}{d}\nodeconnect{ds}{np}%
\nodeconnect{np}{ns}\nodeconnect{ns}{n}%
\nodeconnect{d}{das}%
\nodeconnect{n}{buch}%
}
\end{columns}

}

\frame{
\frametitle{Nominalphrasen: DP (II)}

\begin{columns}

\column{85mm}
Oder Artikel wird als Spezifikator der DP analysiert (syntaktische Kategorie: Art):

\column{35mm}
{\small
\begin{tabular}{ccc}
\multicolumn{3}{c}{\node{dp2}{DP}}\\[4ex]
\node{artp}{ArtP} & \multicolumn{2}{c}{\node{ds2}{D'}}\\[4ex]
\node{arts}{Art'} & \node{d2}{\dnull} & \node{np2}{NP}\\[4ex]
\node{art}{Art$^0$}  &  & \node{ns2}{N'}\\[4ex]
     &  & \node{n2}{\nnull}\\[4ex]
\node{das2}{das} & \node{e}{[~~]} & \node{buch2}{Buch}\\
\end{tabular}
\nodeconnect{dp2}{ds2}%
\nodeconnect{dp2}{artp}\nodeconnect{artp}{arts}\nodeconnect{arts}{art}%
\nodeconnect{ds2}{d2}\nodeconnect{ds2}{np2}%
\nodeconnect{np2}{ns2}\nodeconnect{ns2}{n2}%
\nodeconnect{d2}{e}
\nodeconnect{art}{das2}%
\nodeconnect{n2}{buch2}%
}

\end{columns}


}

\subsubsection{Adjektivphrasen}

\frame{
\frametitle{Adjektivphrasen: Komplemente (I)}


Adjektive können interne Argumente (Komplemente) bei sich haben:\\
\begin{itemize}
\item Genitivobjekt\pause
\ea
Er ist [\sub{AP} [\sub{A'}  [\sub{NP} des kalten Wetters] überdrüssig]].
\z
\pause
\item Dativobjekt\pause
\ea
Er ist [\sub{AP} [\sub{A'}  [\sub{NP} seinem Vater] ähnlich]].
\z
\pause
\item Akkusativobjekt\pause
\ea
Er ist [\sub{AP} [\sub{A'}  [\sub{NP} den Lärm] gewohnt]].
\z
\pause
\item Präpositionalobjekt\pause
\ea
Er ist [\sub{AP} [\sub{A'}  [\sub{PP} auf ihre Tochter] stolz]].
\z
\end{itemize}

}

\frame{
\frametitle{Adjektivphrasen: Komplemente (II)}


\begin{itemize}
\item Prädikative
\ea
Sie ist [\sub{AP} [\sub{A'}  [\sub{KonjP} als Geschäftsführerin] tätig]].
\z
\pause
\item Adverbialien
\pause
\ea
Er ist [\sub{AP} [\sub{A'}  [\sub{PP} in Bremen] ansässig]].
\z
\end{itemize}

}

\frame{
\frametitle{Adjektivphrasen: Spezifikatoren}


\begin{itemize}
\item Adverbiale Akkusative und andere Gradausdrücke sind Spezifikatoren:
\ea
Der Sack ist [\sub{AP} [\sub{NP} einen Zentner] [\sub{A'}  [\sub{A} schwer]]]. 
\z

Alternative: eine spezielle funktionale Hülle DegP für Gradausdrücke\\
(DegP =~Degree Phrase, Gradphrase),\\
analog zur DP-Hypothese: Deg$^0$ ist leer und \emph{einen Zentner} steht in SpecDegP.
\end{itemize}

}

\frame{
\frametitle{Adjektivphrasen: Externes Argument}

\begin{itemize}
\item<+-> Adjektive haben wie Verben ein externes Argument.
\item<+-> Wenn die AP als Adjunkt ein Nomen modifiziert (=~attributives Adjektiv), ist der Schwesterknoten N' der AP mit dem externen Argument koreferent. 

\item<+-> Bei prädikativen Adjektiven fungiert meist das Subjekt oder das Akkusativobjekt als externes Argument der AP: 
\eal
\ex {}[\sub{NP} Der Tisch] ist [\sub{AP} sauber]. 
\ex Otto macht [\sub{NP} den Tisch] [\sub{AP} sauber].
\zl

\end{itemize}

}

\frame[label=sc]{
\frametitle{Adjektivphrasen -- Exkurs: Small Clauses (I)}


\begin{description}
\item[Theta-Kriterium]
Jedes Argument bekommt genau eine Theta-Rolle.
\end{description}

Theta-Kriterium $\to$\\
Objekt in (\mex{1}) bekommt nur von der AP eine Theta-Rolle.
\ea
Otto macht [\sub{NP} den Tisch] [\sub{AP} sauber].
\z
\pause
Zwei Möglichkeiten für den Umgang mit (\mex{0})
\begin{enumerate}
\item Wir verwerfen das Theta-Kriterium\\
      und lassen zu, dass es Argumente gibt, die keine Theta-Rolle bekommen. (\zb in LFG, HPSG)
\item 
Generativen Grammatik Chomsky'scher Prägung:\\
Beziehung zwischen dem Verb einerseits und dem Objekt und der prädikativen AP 
andererseits kommt über eine satzartige Zwischenschicht zustande, 
ein sogenannter Small Clause.%(hier relativ theorieneutral als SC (=~Small Clause) klassifiziert) 
\end{enumerate}
}

\frame{
\frametitle{Adjektivphrasen -- Exkurs: Small Clauses (II)}

Zumindest in Fällen wie (\mex{1}a) ist Paraphrase mit finitem Satz möglich (\mex{1}b):

\eal
\ex Otto macht [\sub{SC} [\sub{NP} den Tisch] [\sub{AP} sauber] ].
\ex Otto macht, [dass [\sub{NP} der Tisch] [\sub{AP} sauber] wird].
\zl

\emph{machen} selegiert in (\mex{0}a--b) 
ein Argument mit Theta-Rolle PROPOSITION, das je nachdem als Small Clause oder als finiter Nebensatz 
realisiert wird.

Small Clauses sind nicht unproblematisch (\citew[Kapitel~7.4]{Mueller2002b} und \compare{sc-probleme}{SC-Probleme})\\
und es gibt Alternativen (auch im GB-Framework).


}

\subsubsection{Präpositionalphrasen}

\frame{
\frametitle{Präpositionalphrasen: Komplemente}

Sieht man von als P kategorisierten Adverbien \compare{slide-lex-kat-gb}{Kreuzklassifikation} ab,
haben Präpositionen immer ein Komplement.

\vfill

\centerline{\begin{tabular}{cc}
\multicolumn{2}{c}{\node{dp}{PP}~~~~}\\[4ex]
\multicolumn{2}{c}{\node{ds}{P'}~~~~}\\[4ex]
\node{d}{P} & \node{np}{NP}\\[6ex]
\node{mit}{mit} & de\node{dk}{n Kinde}rn\\
\end{tabular}%
\nodeconnect{dp}{ds}%
\nodeconnect{ds}{d}\nodeconnect{ds}{np}%
\nodetriangle{np}{dk}%
\nodeconnect{d}{mit}%
%
\hspace{4cm}%
\begin{tabular}{cc}
\multicolumn{2}{c}{\node{pp}{PP}~~~~}\\[4ex]
\multicolumn{2}{c}{\node{ps}{P'}~~~~}\\[4ex]
\node{np2}{NP} & \node{p}{P}\\[6ex]
de\node{dk2}{n Kinde}rn & \node{zuliebe}{zuliebe}\\
\end{tabular}%
\nodeconnect{pp}{ps}%
\nodeconnect{ps}{p}\nodeconnect{ps}{np2}%
\nodetriangle{np2}{dk2}%
\nodeconnect{p}{zuliebe}%
}

\vfill

Manchmal unterteilt man in Prä- und Postpositionen,\\
manchmal faßt man alles unter Präposition zusammen.

}

\frame{
\frametitle{Spezifikatoren der PP und Verwendeung von PPen}

\begin{itemize}
\item Als Spezifikatoren lassen sich Gradausdrücke auffassen,\\
\zb NPs im Akkusativ (=~adverbialer Akkusativ) oder Adverbphrasen: 

\eal
\ex {}[\sub{PP} [\sub{NP} einen Tag] [\sub{P'}  vor [\sub{NP} der Abreise]]]
\ex {}[\sub{PP} [\sub{AdvP} kurz] [\sub{P'}  vor [\sub{NP} der Abreise]]] 
\zl
\pause
\item Präpositionalphrasen, die von Verben und Adjektiven abhängen,\\
können als Objekte, Prädikative oder Adverbialien fungieren;\\
in diesen Funktionen können sie vom Verb oder Adjektiv verlangte Komplemente sein 
oder als Adjunkte die Verbalphrase bzw.\ den Satz modifizieren.
\end{itemize}

}

}%\end{gb-intro}

